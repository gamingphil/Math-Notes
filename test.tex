\documentclass{article}
\usepackage[ngerman]{babel} % |
\usepackage[utf8]{inputenc} % | Language and special characters
\usepackage[T1]{fontenc} %    |
\usepackage{amsmath} % |
\usepackage{amssymb} % | Math symbols and environments
\usepackage{amsthm} %  |
\usepackage{siunitx} %  Typesetting (SI) units
\DeclareSIUnit\litre{l}
%\usepackage{physics} % Advanced Mathematics of Physics, conflicts with siunitx
\usepackage{multirow} % Tables with cells that span more than one row
\usepackage{fancyhdr} % Custom page layout
\usepackage{graphicx} % Insert images
\usepackage{gensymb} % Generic units of measurement in math and text typeface
\usepackage{xfrac} % Split-level fractions
\usepackage{xcolor} % Handling colors
\usepackage{float} % Improved floating objects such as figures and tables
\usepackage{tikz} % Graphics and figures
\usetikzlibrary{arrows} % -> customizable arrow tips
\usepackage{pgfplots} % Plotting functions and data 
\usepgfplotslibrary{fillbetween} % -> filling areas in pgfplots
\pgfplotsset{compat=newest}
\usepackage{systeme} % Systems of equations
\usepackage{romanbar} % Roman numbers with bars
\usepackage{titlepic} % Putting picture on the title page
\usepackage{ragged2e} % Commands and environments for setting ragged text
\usepackage{multicol} % Multicolumn formating
\usepackage[a4paper, ignoreheadfoot, left=2.5cm, right=2.5cm, top=2cm, bottom=3.5cm, headsep=1cm]{geometry} % Margins etc.
\usepackage{cancel} % Cancel terms in equations
\usepackage[version=4]{mhchem} % Write chemical equations
\usepackage{wrapfig} % Figures with text wrapped around them
\usepackage{hyperref} % Extensive support for hypertext
\usepackage{sidecap} % Typeset captions sideways
\sidecaptionvpos{figure}{c} % -> Vertical alignment of caption
\usepackage{pdfpages} % Include PDF documents
\usepackage{setspace} % Set space between lines 
% (\singlespacing, \onehalfspacing, \doublespacing)
\usepackage{subcaption}
\usepackage[scale=0.96]{XCharter} % Use the XCharter text font
% \usepackage[xcharter]{newtxmath} % Set the math font

% Hyperref setup:
\hypersetup{
    colorlinks=true,
    linkcolor=black,
    filecolor=magenta,      
    urlcolor=blue,
    citecolor=black
}


\definecolor{page}{HTML}{FFFFFF} % white
\definecolor{text}{HTML}{000000} % black
\definecolor{primary}{HTML}{019875}
\definecolor{contrastColour}{HTML}{E8F1F2} % white
\definecolor{tertiary}{HTML}{C47238} % yellow
\definecolor{secondary}{HTML}{C6B53C} % orange
\definecolor{quaternary}{HTML}{BD3E4C} % red
\definecolor{alternativePrimary}{HTML}{13293D} % blue
\colorlet{infoBulleBackground}{page!98!text}
% Shades
\definecolor{LightGrey}{HTML}{a5b1c2}
\definecolor{Grey}{HTML}{778ca3}
\definecolor{DarkGrey}{HTML}{4b6584}
\definecolor{Black}{HTML}{231F20}
% Primary Colours
\definecolor{Red}{HTML}{eb3b47}
\definecolor{LightRed}{HTML}{fc5c65}
\definecolor{DarkRed}{HTML}{d91c38}

\definecolor{Yellow}{HTML}{f7b731}
\definecolor{LightYellow}{HTML}{fed330}
\definecolor{DarkYellow}{HTML}{f0a132}

\definecolor{Blue}{HTML}{3867d6}
\definecolor{LightBlue}{HTML}{4b7bec}
\definecolor{DarkBlue}{HTML}{2654bf}
% tertiary Colours
\definecolor{Green}{HTML}{20bf6b}
\definecolor{LightGreen}{HTML}{26de81}
\definecolor{DarkGreen}{HTML}{1ba155}

\definecolor{Orange}{HTML}{fa8231}
\definecolor{LightOrange}{HTML}{fd9644}
\definecolor{DarkOrange}{HTML}{f76b20}

\definecolor{Purple}{HTML}{8854d0}
\definecolor{LightPurple}{HTML}{a55eea}
\definecolor{DarkPurple}{HTML}{6e49b8}
% secondary+ Colours
% \definecolor{Brown}{HTML}{}
\definecolor{Cyan}{HTML}{0fb9b1}
\definecolor{LightCyan}{HTML}{2bcbba}
\definecolor{DarkCyan}{HTML}{00a8a8}


\usetikzlibrary{calc}

\usepackage[many]{tcolorbox}

\newtcolorbox{boxx}[2][]{
    enhanced,
    fonttitle=\fontsize{11}{13.2}\selectfont\bfseries,
    coltitle=text,
    colback=infoBulleBackground,
    colbacktitle=infoBulleBackground,
    colframe=infoBulleBackground,
    toprule=4pt,titlerule=0pt,bottomrule=0pt,rightrule=0pt,leftrule=0pt,
    segmentation hidden,
    sharpish corners,
    overlay={\draw[line width=2.5pt,#1] (frame.north west)--(frame.south west);},
    % margins & padding
    before skip=\baselineskip,
    after skip=\baselineskip,
    boxsep=2pt,
    top=6pt,
    bottom=4pt,
    left=8pt,
    right=6pt,
    breakable,
    title=#2,
    extras middle and last={%
        %top=20pt,
        overlay={%
            \draw[line width=2.5pt,#1] (frame.north west)--(frame.south west);
            % \draw[line width=0.5pt,dashed,#1] (frame.north west)--(frame.north east);
            % \draw[] {([yshift=-12pt,xshift=4.5pt]frame.north west)} node[anchor=west] {#2};
            % \draw[] {([yshift=-12pt,xshift=-4.5pt]frame.north east)} node[anchor=east] {eg};
        }
    },
}

\definecolor{codegreen}{HTML}{369432}
\definecolor{codegray}{HTML}{3a3d41}
\definecolor{codepurple}{HTML}{c586c0}
\definecolor{backcolour}{HTML}{1e1e1e}
\definecolor{standartcolor}{HTML}{cccccc}
\definecolor{codeorange}{HTML}{f48771}

\setlength{\parindent}{0pt}
\definecolor{bg}{rgb}{0.13, 0.13, 0.13}
%\pagecolor{bg}
%\color{white}
\begin{document}
%---COLORS---
\definecolor{pRed}{RGB}{255, 55, 55}
\definecolor{pOrange}{RGB}{219, 132, 36}

%\pagestyle{fancy}
%\fancyhf{}
%\lhead{Moathekuchenaufgabe}
%\rhead{Lösung}
%\cfoot{\thepage}

\subsection[]{Tangente und Normale von Außen}

Gegeben ist eine Funktion $f$ und sowie ein Punkt $P$, der nicht auf $f$ liegt.

Bestimme die Gleichung(en) der Tangente(n) durch $P$ an $f$.

\begin{boxx}[LightGreen]{Allgemeine Tangentengleichung}
    \begin{align*}
        \frac{\Delta y}{\Delta x} = \frac{t(x)-f(u)}{x-u} &= f'(x) & &|\;\cdot (x-u) \\
        t(x) - f(u) &= f'(u) \cdot (x-u) & & |\; + f(u) \\
        t(x) &= f'(u)\cdot(x-u)+f(u)
    \end{align*}
\end{boxx}

\begin{figure}[H]
    \centering
    \begin{tikzpicture}
    \begin{axis}[
        height = 10cm,
        width = 13cm,
        axis lines = middle,
        ymin = -4,
        axis equal,
        xlabel = {\(x\)},
        ylabel = {\(y\)},
    ]
    \addplot[
        line width = 1pt,
        color = DarkBlue,
        samples = 100,
        domain = -3:3
    ]{x^2} node[anchor= east, pos = 0.98] {\(f(x)\)};
    \addplot[Green, domain = -1:4]{2*2^(0.5)*x - 2} node[anchor= west, pos = 0.98] {\(t_1(x)\)};
    \addplot[Green, domain = -4:1]{-2*2^(0.5)*x - 2} node[anchor= east, pos = 0.03] {\(t_2(x)\)};
    \addplot[] coordinates {(2.47, 5) (3.54,5)} node[anchor = north, pos = 0.5] {\(\Delta x\)};
    \addplot[] coordinates {(3.54, 5) (3.54,8)} node[anchor = west, pos = 0.5] {\(\Delta y\)};
    \addplot[Red, domain = -6:6]{(-1/(2*2^(0.5)))*x + 2.5} node[anchor= south west, pos = 0.98] {\(n_1(x)\)};
    \addplot[Red, domain = -6:6]{(1/(2*2^(0.5)))*x + 2.5} node[anchor= south east, pos = 0.02] {\(n_2(x)\)};
    \node[label={10:{\(Q(u,f(u))\)}},circle,fill,inner sep=1.5pt] at (axis cs:1.41,2) {};
    \node[label={0:{\(P(0,-2)\)}},circle,fill,inner sep=1.5pt] at (axis cs:0,-2) {};
    \end{axis}
    \end{tikzpicture}
    
\end{figure}


\begin{enumerate}
    \item Terme von $f(u)$ bzw. $f'(u)$ bestimmen.
    \begin{align*}
        f(u)=u^2;\;f'(u)=2u
    \end{align*}
    \item $P$ in $t(x)$ einsetzen.
    \begin{align*}
        -2 &= 2u\cdot (0-u)+u^2 \\
        -2 &= -2u^2 +u^2 \\
        -2 &= -u^2 \\
        u &= \pm \sqrt{2}
    \end{align*}
    \item $u_1$ / $u_2$ in $t(x)$ einsetzen.
    \begin{align*}
        &u_1: & t_1(x) &= 2\sqrt{2} \cdot (x-\sqrt{2}) + \left(\sqrt{u}\right)^2  &&\\
        & & &= 2\sqrt{2} x -2 &&\\
        & u_2: & t_2(x) &= -2\sqrt{2} x -2 &&
    \end{align*}
\end{enumerate}
\begin{boxx}[Red]{Normale}
    Die Gerade, die die Tangente orthogonal schneidet heißt \textbf{Normale}.

    Für die Steigung der Normalen gilt:
    \begin{align*}
        m_t \cdot m_n &= -1 \\
        m_n &= -\frac{1}{m_t}
    \end{align*}
    \[n(x) = -\frac{1}{f'(u)}\cdot (x-u)+f(u)\]
\end{boxx}

\end{document}