\documentclass{article}
\usepackage[ngerman]{babel} % |
\usepackage[utf8]{inputenc} % | Language and special characters
\usepackage[T1]{fontenc} %    |
\usepackage{amsmath} % |
\usepackage{amssymb} % | Math symbols and environments
\usepackage{amsthm} %  |
\usepackage{siunitx} %  Typesetting (SI) units
\DeclareSIUnit\litre{l}
%\usepackage{physics} % Advanced Mathematics of Physics, conflicts with siunitx
\usepackage{multirow} % Tables with cells that span more than one row
\usepackage{fancyhdr} % Custom page layout
\usepackage{graphicx} % Insert images
\usepackage{gensymb} % Generic units of measurement in math and text typeface
\usepackage{xfrac} % Split-level fractions
\usepackage{xcolor} % Handling colors
\usepackage{float} % Improved floating objects such as figures and tables
\usepackage{tikz} % Graphics and figures
\usetikzlibrary{arrows} % -> customizable arrow tips
\usepackage{pgfplots} % Plotting functions and data 
\usepgfplotslibrary{fillbetween} % -> filling areas in pgfplots
\pgfplotsset{compat=newest}
\usepackage{systeme} % Systems of equations
\usepackage{romanbar} % Roman numbers with bars
\usepackage{titlepic} % Putting picture on the title page
\usepackage{ragged2e} % Commands and environments for setting ragged text
\usepackage{multicol} % Multicolumn formating
\usepackage[a4paper, ignoreheadfoot, left=2.5cm, right=2.5cm, top=2cm, bottom=3.5cm, headsep=1cm]{geometry} % Margins etc.
\usepackage{cancel} % Cancel terms in equations
\usepackage[version=4]{mhchem} % Write chemical equations
\usepackage{wrapfig} % Figures with text wrapped around them
\usepackage{hyperref} % Extensive support for hypertext
\usepackage{sidecap} % Typeset captions sideways
\sidecaptionvpos{figure}{c} % -> Vertical alignment of caption
\usepackage{pdfpages} % Include PDF documents
\usepackage{setspace} % Set space between lines 
% (\singlespacing, \onehalfspacing, \doublespacing)
\usepackage{subcaption}
\usepackage[scale=0.96]{XCharter} % Use the XCharter text font
% \usepackage[xcharter]{newtxmath} % Set the math font

% Hyperref setup:
\hypersetup{
    colorlinks=true,
    linkcolor=black,
    filecolor=magenta,      
    urlcolor=blue,
    citecolor=black
}


\definecolor{page}{HTML}{FFFFFF} % white
\definecolor{text}{HTML}{000000} % black
\definecolor{primary}{HTML}{019875}
\definecolor{contrastColour}{HTML}{E8F1F2} % white
\definecolor{tertiary}{HTML}{C47238} % yellow
\definecolor{secondary}{HTML}{C6B53C} % orange
\definecolor{quaternary}{HTML}{BD3E4C} % red
\definecolor{alternativePrimary}{HTML}{13293D} % blue
\colorlet{infoBulleBackground}{page!98!text}
% Shades
\definecolor{LightGrey}{HTML}{a5b1c2}
\definecolor{Grey}{HTML}{778ca3}
\definecolor{DarkGrey}{HTML}{4b6584}
\definecolor{Black}{HTML}{231F20}
% Primary Colours
\definecolor{Red}{HTML}{eb3b47}
\definecolor{LightRed}{HTML}{fc5c65}
\definecolor{DarkRed}{HTML}{d91c38}

\definecolor{Yellow}{HTML}{f7b731}
\definecolor{LightYellow}{HTML}{fed330}
\definecolor{DarkYellow}{HTML}{f0a132}

\definecolor{Blue}{HTML}{3867d6}
\definecolor{LightBlue}{HTML}{4b7bec}
\definecolor{DarkBlue}{HTML}{2654bf}
% tertiary Colours
\definecolor{Green}{HTML}{20bf6b}
\definecolor{LightGreen}{HTML}{26de81}
\definecolor{DarkGreen}{HTML}{1ba155}

\definecolor{Orange}{HTML}{fa8231}
\definecolor{LightOrange}{HTML}{fd9644}
\definecolor{DarkOrange}{HTML}{f76b20}

\definecolor{Purple}{HTML}{8854d0}
\definecolor{LightPurple}{HTML}{a55eea}
\definecolor{DarkPurple}{HTML}{6e49b8}
% secondary+ Colours
% \definecolor{Brown}{HTML}{}
\definecolor{Cyan}{HTML}{0fb9b1}
\definecolor{LightCyan}{HTML}{2bcbba}
\definecolor{DarkCyan}{HTML}{00a8a8}


\usetikzlibrary{calc}

\usepackage[many]{tcolorbox}

\newtcolorbox{boxx}[2][]{
    enhanced,
    fonttitle=\fontsize{11}{13.2}\selectfont\bfseries,
    coltitle=text,
    colback=infoBulleBackground,
    colbacktitle=infoBulleBackground,
    colframe=infoBulleBackground,
    toprule=4pt,titlerule=0pt,bottomrule=0pt,rightrule=0pt,leftrule=0pt,
    segmentation hidden,
    sharpish corners,
    overlay={\draw[line width=2.5pt,#1] (frame.north west)--(frame.south west);},
    % margins & padding
    before skip=\baselineskip,
    after skip=\baselineskip,
    boxsep=2pt,
    top=6pt,
    bottom=4pt,
    left=8pt,
    right=6pt,
    breakable,
    title=#2,
    extras middle and last={%
        %top=20pt,
        overlay={%
            \draw[line width=2.5pt,#1] (frame.north west)--(frame.south west);
            % \draw[line width=0.5pt,dashed,#1] (frame.north west)--(frame.north east);
            % \draw[] {([yshift=-12pt,xshift=4.5pt]frame.north west)} node[anchor=west] {#2};
            % \draw[] {([yshift=-12pt,xshift=-4.5pt]frame.north east)} node[anchor=east] {eg};
        }
    },
}

\definecolor{codegreen}{HTML}{369432}
\definecolor{codegray}{HTML}{3a3d41}
\definecolor{codepurple}{HTML}{c586c0}
\definecolor{backcolour}{HTML}{1e1e1e}
\definecolor{standartcolor}{HTML}{cccccc}
\definecolor{codeorange}{HTML}{f48771}

\setlength{\parindent}{0pt}
\definecolor{bg}{rgb}{0.13, 0.13, 0.13}
%\pagecolor{bg}
%\color{white}
\begin{document}
%---COLORS---
\definecolor{pRed}{RGB}{255, 55, 55}
\definecolor{pOrange}{RGB}{219, 132, 36}

%\pagestyle{fancy}
%\fancyhf{}
%\lhead{Moathekuchenaufgabe}
%\rhead{Lösung}
%\cfoot{\thepage}

\setcounter{section}{3}
\section{Funktionen und ihre Graphen}
\subsection{Strecken, verschieben, spiegeln}
Gegeben sei der Graph der Funktion $f$. 
Der in \color{Green}$x$-Richtung verschobene\color{black}, in \color{Yellow}$y$-Richtung verschobene \color{black}
und in \color{Purple}$y$-Richtung \color{black} gestreckte Graph der Funktion $g$ 
besitzt den Funktionsterm:
\begin{align*}
    g(x) = \color{Purple} a \color{black}\cdot f(x \color{Green}- c\color{black}) \color{Yellow} + d
\end{align*}
Bei den Spiegelungen von $f$ gilt:
\begin{itemize}
    \item $g(x) = f(-x)\;\;\;$ Spiegelung an der \textbf{\emph{y}-Achse}
    \item $g(x) = -f(x)\;\;\;$ Spiegelung an der \textbf{\emph{x}-Achse}
    \item $g(x) = -f(-x)\;\;\;$ Spiegelung am \textbf{Ursprung}
\end{itemize}
\begin{boxx}[DarkBlue]{Beispiel}
    Skizziere die Graphen von $f$ und $g$.\\

    $a)$\hspace{3mm} $\displaystyle f(x) = - \sqrt{x+2} - 3;\;\;\;\; x\geq -2$ \\
    \begin{figure}[H]
        \centering
        \begin{tikzpicture}
        \begin{axis}[
            width = 12cm,
            height = 7cm,
            axis lines = middle,
            xlabel = {\(x\)},
            ylabel = {\(y\)},
            xmin = -3,
            xmax = 8,
            ymin = -7,
            ymax= 3,
        ]
        \addplot[dashed, thick, Blue, domain=-0:6,samples=100] {x^0.5} node[anchor = north west, pos = 1] {\(y = \sqrt{x}\)};
        \addplot[dashed, thick, Green, domain=-2:3,samples=100] {(x+2)^0.5} node[anchor = south east, pos = 0.3] {\(y = \sqrt{x+2}\)};
        \addplot[dashed, thick, Purple, domain=-2:3,samples=100] {-(x+2)^0.5} node[anchor = west, pos = 1] {\(y = -\sqrt{x+2}\)};
        \addplot[thick, smooth, Yellow, domain=-2:7.5,samples=100] {-(x+2)^0.5-3} node[anchor = south west, pos = 0.7] {\(y = -\sqrt{x+2}-3\)};
        \end{axis}
        \end{tikzpicture}
    \end{figure}
    $b)$\hspace{3mm} $\displaystyle g(x) = \frac{2}{x+1}- \frac{1}{2};\;\;\;\; x \not = -1$
    \begin{figure}[H]
        \centering
        \begin{tikzpicture}
        \begin{axis}[
            width = 11cm,
            height = 9cm,
            axis lines = middle,
            xlabel = {\(x\)},
            ylabel = {\(y\)},
            xmin = -7,
            xmax = 7,
            ymin = -7,
            ymax= 7,
            restrict y to domain=-12:12
        ]
        \addplot[dashed, thick, Blue, domain=-7:7,samples=200] {1/x} node[anchor = west, pos = 0.7] {\(\displaystyle y = \frac{1}{x}\)};
        \addplot[dashed, thick, Green, domain=-7:7,samples=200] {1/(x+1)} node[anchor = east, pos = 0.67] {\(\displaystyle y = \frac{1}{x+1}\)};
        \addplot[dashed, thick, Purple, domain=-7:7,samples=200] {2/(x+1)}node[anchor = south, pos = 0.92] {\(\displaystyle y = \frac{2}{x+1}\)};
        \addplot[thick, Yellow, domain=-7:7,samples=200] {2/(x+1)-0.5} node[anchor = north east, pos = 0.15] {\(\displaystyle y = \frac{2}{x+1} - \frac{1}{2}\)};
        \end{axis}
        \end{tikzpicture}
    \end{figure}
    Zeige, dass die Graphen von $f_k$ mit $\displaystyle f_k(x) = k x e^{x^2};\; k \in \mathbb{R}$
    punktsymmetrisch zum Ursprung sind.
    \begin{align*}
        f(-x) &= k \cdot (-x) \cdot e^{(-x)^2} \\
        &= - k x e^{x^2} \\
        &= -f(x) 
    \end{align*}
    \qed
\end{boxx}
\subsection{Linearfaktordarstellung - mehrfache Nullstellen}
\begin{boxx}[LightGreen]{Satz 1}
    Hat eine ganzrationale Funktion vom Grad $n$ eine Nullstelle
    $x_0$, so gilt:
    \begin{align*}
        f(x) &= (x - x_0)\cdot g(x) & &\text{wobei } g \text{ vom Grad } n-1 \text{ ist.}
    \end{align*}
    $(x- x_0)$ nennt man \textbf{Linearfaktor}.
\end{boxx}
\begin{boxx}[LightGreen]{Satz 2}
    Eine ganzrationale Funktion $n$-ten Grades besitzt höchstens $n$ Nullstellen.
\end{boxx}
\begin{boxx}[LightGreen]{Satz 3}
    Sei $\displaystyle f(x) = (x-a)^k \cdot g(x)$.
    \begin{itemize}
        \item Für $k = 1:$ Schnittstelle von $f$ mit der $x$-Achse.
        \item Für $k = 2:$ Berührstelle von $f$ an der $x$-Achse.
        \item Für $k = 3:$ Sattelstelle von $f$ an der $x$-Achse.
    \end{itemize}
\end{boxx}
\begin{boxx}[DarkBlue]{Beispiel}
    Skizziere den Graph von $f$ mit $\displaystyle f(x) = x^3(x-2)^2(x+1)$.
    \begin{figure}[H]
        \centering
        \begin{tikzpicture}
        \begin{axis}[
            width = 10cm,
            height = 7cm,
            axis lines = middle,
            xlabel = {\(x\)},
            ylabel = {\(y\)},
            restrict y to domain=-8:8,
            xmax = 2.75,
            xmin = -1.75,
            ymax = 5,
            ymin = -1,
            ytick = {\empty},
        ]
            \addplot[thick, DarkBlue, samples = 100, domain = -1.75:2.75]{x^3*(x-2)^2*(x+1)} node[anchor = east, pos = 0.85] {\(G_f\)};
        \end{axis}
        \end{tikzpicture}
    \end{figure}
\end{boxx}
\end{document}