\documentclass{article}
\usepackage[ngerman]{babel} % |
\usepackage[utf8]{inputenc} % | Language and special characters
\usepackage[T1]{fontenc} %    |
\usepackage{amsmath} % |
\usepackage{amssymb} % | Math symbols and environments
\usepackage{amsthm} %  |
\usepackage{siunitx} %  Typesetting (SI) units
\DeclareSIUnit\litre{l}
%\usepackage{physics} % Advanced Mathematics of Physics, conflicts with siunitx
\usepackage{multirow} % Tables with cells that span more than one row
\usepackage{fancyhdr} % Custom page layout
\usepackage{graphicx} % Insert images
\usepackage{gensymb} % Generic units of measurement in math and text typeface
\usepackage{xfrac} % Split-level fractions
\usepackage{xcolor} % Handling colors
\usepackage{float} % Improved floating objects such as figures and tables
\usepackage{tikz} % Graphics and figures
\usetikzlibrary{arrows} % -> customizable arrow tips
\usetikzlibrary{tikzmark}
\usepackage{pgfplots} % Plotting functions and data 
\usepgfplotslibrary{fillbetween} % -> filling areas in pgfplots
\pgfplotsset{compat=newest}
\usepackage{systeme} % Systems of equations
\usepackage{romanbar} % Roman numbers with bars
\usepackage{titlepic} % Putting picture on the title page
\usepackage{ragged2e} % Commands and environments for setting ragged text
\usepackage{multicol} % Multicolumn formating
\usepackage[a4paper, ignoreheadfoot, left=2.5cm, right=2.5cm, top=2cm, bottom=3.5cm, headsep=1cm]{geometry} % Margins etc.
\usepackage{cancel} % Cancel terms in equations
\usepackage[version=4]{mhchem} % Write chemical equations
\usepackage{wrapfig} % Figures with text wrapped around them
\usepackage{hyperref} % Extensive support for hypertext
\usepackage{sidecap} % Typeset captions sideways
\sidecaptionvpos{figure}{c} % -> Vertical alignment of caption
\usepackage{pdfpages} % Include PDF documents
\usepackage{setspace} % Set space between lines 
% (\singlespacing, \onehalfspacing, \doublespacing)
\usepackage{subcaption}
\usepackage[scale=0.96]{XCharter} % Use the XCharter text font
% \usepackage[xcharter]{newtxmath} % Set the math font

% Hyperref setup:
\hypersetup{
    colorlinks=true,
    linkcolor=black,
    filecolor=magenta,      
    urlcolor=blue,
    citecolor=black
}


\definecolor{page}{HTML}{FFFFFF} % white
\definecolor{text}{HTML}{000000} % black
\definecolor{primary}{HTML}{019875}
\definecolor{contrastColour}{HTML}{E8F1F2} % white
\definecolor{tertiary}{HTML}{C47238} % yellow
\definecolor{secondary}{HTML}{C6B53C} % orange
\definecolor{quaternary}{HTML}{BD3E4C} % red
\definecolor{alternativePrimary}{HTML}{13293D} % blue
\colorlet{infoBulleBackground}{page!98!text}
% Shades
\definecolor{LightGrey}{HTML}{a5b1c2}
\definecolor{Grey}{HTML}{778ca3}
\definecolor{DarkGrey}{HTML}{4b6584}
\definecolor{Black}{HTML}{231F20}
% Primary Colours
\definecolor{Red}{HTML}{eb3b47}
\definecolor{LightRed}{HTML}{fc5c65}
\definecolor{DarkRed}{HTML}{d91c38}

\definecolor{Yellow}{HTML}{f7b731}
\definecolor{LightYellow}{HTML}{fed330}
\definecolor{DarkYellow}{HTML}{f0a132}

\definecolor{Blue}{HTML}{3867d6}
\definecolor{LightBlue}{HTML}{4b7bec}
\definecolor{DarkBlue}{HTML}{2654bf}
% tertiary Colours
\definecolor{Green}{HTML}{20bf6b}
\definecolor{LightGreen}{HTML}{26de81}
\definecolor{DarkGreen}{HTML}{1ba155}

\definecolor{Orange}{HTML}{fa8231}
\definecolor{LightOrange}{HTML}{fd9644}
\definecolor{DarkOrange}{HTML}{f76b20}

\definecolor{Purple}{HTML}{8854d0}
\definecolor{LightPurple}{HTML}{a55eea}
\definecolor{DarkPurple}{HTML}{6e49b8}
% secondary+ Colours
% \definecolor{Brown}{HTML}{}
\definecolor{Cyan}{HTML}{0fb9b1}
\definecolor{LightCyan}{HTML}{2bcbba}
\definecolor{DarkCyan}{HTML}{00a8a8}


\usetikzlibrary{calc}

\usepackage[many]{tcolorbox}

\newtcolorbox{boxx}[2][]{
    enhanced,
    fonttitle=\fontsize{11}{13.2}\selectfont\bfseries,
    coltitle=text,
    colback=infoBulleBackground,
    colbacktitle=infoBulleBackground,
    colframe=infoBulleBackground,
    toprule=4pt,titlerule=0pt,bottomrule=0pt,rightrule=0pt,leftrule=0pt,
    segmentation hidden,
    sharpish corners,
    overlay={\draw[line width=2.5pt,#1] (frame.north west)--(frame.south west);},
    % margins & padding
    before skip=\baselineskip,
    after skip=\baselineskip,
    boxsep=2pt,
    top=6pt,
    bottom=4pt,
    left=8pt,
    right=6pt,
    breakable,
    title=#2,
    extras middle and last={%
        %top=20pt,
        overlay={%
            \draw[line width=2.5pt,#1] (frame.north west)--(frame.south west);
            % \draw[line width=0.5pt,dashed,#1] (frame.north west)--(frame.north east);
            % \draw[] {([yshift=-12pt,xshift=4.5pt]frame.north west)} node[anchor=west] {#2};
            % \draw[] {([yshift=-12pt,xshift=-4.5pt]frame.north east)} node[anchor=east] {eg};
        }
    },
}

\definecolor{codegreen}{HTML}{369432}
\definecolor{codegray}{HTML}{3a3d41}
\definecolor{codepurple}{HTML}{c586c0}
\definecolor{backcolour}{HTML}{1e1e1e}
\definecolor{standartcolor}{HTML}{cccccc}
\definecolor{codeorange}{HTML}{f48771}

\setlength{\parindent}{0pt}
\definecolor{bg}{rgb}{0.13, 0.13, 0.13}
%\pagecolor{bg}
%\color{white}
\begin{document}
%---COLORS---
\definecolor{pRed}{RGB}{255, 55, 55}
\definecolor{pOrange}{RGB}{219, 132, 36}

%\pagestyle{fancy}
%\fancyhf{}
%\lhead{Moathekuchenaufgabe}
%\rhead{Lösung}
%\cfoot{\thepage}

\setcounter{section}{4}
\section{Lineare Gleichungssysteme}
\subsection{Der Gauß-Algorithmus}
Mehrere Gleichungen mit gemeinsamen (linearen) Variablen bilden ein
\textbf{lineares Gleichungssystem (LGS).}

Ein LGS mit 3 oder mehr Variablen löst man meistens mit dem 
\textbf{Gauß-Algorithmus} am sinnvollsten. Dabei werden durch
\begin{enumerate}
    \item vertauschen von Gleichungen,
    \item Multiplikation von einer oder mehrerer Gleichungen mit einer Zahl $\not= 0$,
    \item Addition mehrerer Gleichungen und
    \item Einsetzen einer Variablen in eine andere Gleichung
\end{enumerate}
so lange Variablen eliminiert und dadurch bestimmt bis eine sogenannte
\textbf{Stufenform} vorliegt.
\[\left( \begin{array}{ccc|c} x & x & x & x \\ \tikzmark{d1}0 & x & x  & x \\ \tikzmark{d2}0 & 0\tikzmark{d3} & x & x \end{array} \right) \begin{array}{cc}
    &\\
    \longleftarrow \text{Matrix-Schreibweise} &\\
    &
\end{array}\]
\begin{tikzpicture}[overlay, remember picture]
    \draw ([xshift=-1pt,yshift=11pt]{pic cs:d1}) -- ([xshift=-1pt,yshift=-1pt]{pic cs:d2});
    \draw ([xshift=-1pt,yshift=-1pt]{pic cs:d2}) -- ([xshift=6pt,yshift=-1pt]{pic cs:d3});
    \draw ([xshift=6pt,yshift=-1pt]{pic cs:d3}) -- ([xshift=-1pt,yshift=11pt]{pic cs:d1});
\end{tikzpicture}
\begin{boxx}[DarkBlue]{Beispiel}
    Löse das LGS.
    \begin{align*}
        &&&&2x_1 - 4x_2 + 5x_3 &= 3 &&&&|\; \cdot (-3) &&|\; \cdot (-2) &&&&\\
        &&&&3x_1 + 3x_2 + 7x_3 &= 13 &&&&|\; \cdot 2 &&&&\\
        &&&&4x_1 - 2x_2 - 3x_3 &= -1 &&&&&&|\; \cdot 1&&&&
    \end{align*}
    \begin{align*}
        &\sim \left( \begin{array}{ccc|c} 2 & -4 & 5 & 3 \\ 0 & 18 & -1  & 17 \\ 0 & 6 & -13 & -7 \end{array} \right) \begin{array}{ll}
             &  \\
             |\cdot 1& \\
             |\cdot (-3)&
        \end{array} \\
        &\sim \left( \begin{array}{ccc|c} 2 & -4 & 5 & 3 \\ 0 & 18 & -1  & 17 \\ 0 & 0 & 38 & 38 \end{array} \right) \begin{array}{ll}
             &  \\
             & \\
             \Rightarrow x_3 = 1&
        \end{array} 
    \end{align*}
    \begin{align*}
        18x_2 - 1 &= 17 \\
        x_2 &= 1
    \end{align*}
    \begin{align*}
        2x_1 - 4 + 5 &= 3 \\
        2x_1 &= 2 \\
        x_1 &= 1
    \end{align*}
    \[\Rightarrow \mathbb{L} = \lbrace(1;\;1;\;1)\rbrace\]
\end{boxx}
\newpage
\subsection{Lösungsmengen von LGS}
Man bringt ein LGS wie gewohnt un Stufenform und
erkennt dann schnell, dass ein LGS entweder
\begin{itemize}
    \item \textbf{keine}
    \item \textbf{eine} oder
    \item \textbf{unendlich} viele Lösungen haben kann.
\end{itemize}
Bei keiner Lösung erhält man eine Zeile der Form
$\displaystyle 0 \cdot x_3 = 1$, bei unendlich vielen
Lösungen erhält man bspw. $0 \cdot x_3 = 0$.
Dann wählt man für $x_3$ einen beliebigen Parameter und gibt die anderen Variablen in Abhängigkeit von diesem an.
\begin{boxx}[DarkBlue]{Beispiel}
    Bestimme die Lösungsmenge des LGS. \\
    $a)$\hspace{3mm}
    \begin{align*}
        &\left( \begin{array}{ccc|c} 1 & 2 & 2 & 3 \\ 2 & 4 & 4  & 6 \\ -1 & 1 & 5 & 9 \end{array} \right) \begin{array}{ll}
             |\cdot (-2) & |\cdot 1 \\
             |\cdot 1& \\
             &|\cdot 1
        \end{array} \\
        \sim&\left( \begin{array}{ccc|c} 1 & 2 & 2 & 3 \\ 0 & 0 & 0 & 0 \\ 0 & 3 & 7 & 12 \end{array} \right) \begin{array}{ll}
             & \\
             \Rightarrow x_3 = t& \\
             &
        \end{array} \\
    \end{align*}
    \begin{align*}
        3 x_2 + 7t &= 12 \\
        x_2 &= - \frac{7}{3}t + 4
    \end{align*}
    \begin{align*}
        x_1 + 2\left(-\frac{7}{3}t + 4\right) + 2t &= 3 \\
        x_1 - \frac{14}{3}t + 8 + 2t &= 3
        x_1 &= \frac{8}{3}t - 5
    \end{align*}
    \[\mathbb{L} = \biggl\{\left(\frac{8}{3}t - 5;\; -\frac{7}{3}t + 4;\; t\right)\biggl\}\]
    $b)$\hspace{3mm}
    \begin{align*}
        &\left( \begin{array}{ccc|c} -2 & 1 & 4 & 3 \\ -4 & 2 & 8  & 8 \\ -3 & 2 & 5 & 9 \end{array} \right) \begin{array}{ll}
             |\cdot (-2) & \\
             |\cdot 1& \\
             &
        \end{array} \\
        \sim&\left( \begin{array}{ccc|c} -2 & 1 & 4 & 3 \\ 0 & 0 & 0 & 2 \\ -3 & 2 & 5 & 9 \end{array} \right) \begin{array}{ll}
             & \\
             \Rightarrow \mathbb{L} = \{\}& \\
             &
        \end{array} \\
    \end{align*}
\end{boxx}
\end{document}