\documentclass{article}
\usepackage[ngerman]{babel} % |
\usepackage[utf8]{inputenc} % | Language and special characters
\usepackage[T1]{fontenc} %    |
\usepackage{amsmath} % |
\usepackage{amssymb} % | Math symbols and environments
\usepackage{amsthm} %  |
\usepackage{siunitx} %  Typesetting (SI) units
\DeclareSIUnit\litre{l}
%\usepackage{physics} % Advanced Mathematics of Physics, conflicts with siunitx
\usepackage{multirow} % Tables with cells that span more than one row
\usepackage{fancyhdr} % Custom page layout
\usepackage{graphicx} % Insert images
\usepackage{gensymb} % Generic units of measurement in math and text typeface
\usepackage{xfrac} % Split-level fractions
\usepackage{xcolor} % Handling colors
\usepackage{float} % Improved floating objects such as figures and tables
\usepackage{tikz} % Graphics and figures
\usetikzlibrary{arrows} % -> customizable arrow tips
\usetikzlibrary{tikzmark}
\usepackage{pgfplots} % Plotting functions and data 
\usepgfplotslibrary{fillbetween} % -> filling areas in pgfplots
\pgfplotsset{compat=newest}
\usepackage{tkz-base,tkz-euclide}
\usepackage{systeme} % Systems of equations
\usepackage{romanbar} % Roman numbers with bars
\usepackage{titlepic} % Putting picture on the title page
\usepackage{ragged2e} % Commands and environments for setting ragged text
\usepackage{multicol} % Multicolumn formating
\usepackage[a4paper, ignoreheadfoot, left=2.5cm, right=2.5cm, top=2cm, bottom=3.5cm, headsep=1cm]{geometry} % Margins etc.
\usepackage{cancel} % Cancel terms in equations
\usepackage[version=4]{mhchem} % Write chemical equations
\usepackage{wrapfig} % Figures with text wrapped around them
\usepackage{hyperref} % Extensive support for hypertext
\usepackage{sidecap} % Typeset captions sideways
\sidecaptionvpos{figure}{c} % -> Vertical alignment of caption
\usepackage{pdfpages} % Include PDF documents
\usepackage{setspace} % Set space between lines 
% (\singlespacing, \onehalfspacing, \doublespacing)
\usepackage{subcaption}
\usepackage[scale=0.96]{XCharter} % Use the XCharter text font
% \usepackage[xcharter]{newtxmath} % Set the math font
\usepackage{esvect} % Nicer vectors (\vv{})
\usepackage[linguistics]{forest}
% Hyperref setup:
\hypersetup{
    colorlinks=true,
    linkcolor=black,
    filecolor=magenta,      
    urlcolor=blue,
    citecolor=black
}


\definecolor{page}{HTML}{FFFFFF} % white
\definecolor{text}{HTML}{000000} % black
\definecolor{primary}{HTML}{019875}
\definecolor{contrastColour}{HTML}{E8F1F2} % white
\definecolor{tertiary}{HTML}{C47238} % yellow
\definecolor{secondary}{HTML}{C6B53C} % orange
\definecolor{quaternary}{HTML}{BD3E4C} % red
\definecolor{alternativePrimary}{HTML}{13293D} % blue
\colorlet{infoBulleBackground}{page!98!text}
% Shades
\definecolor{LightGrey}{HTML}{a5b1c2}
\definecolor{Grey}{HTML}{778ca3}
\definecolor{DarkGrey}{HTML}{4b6584}
\definecolor{Black}{HTML}{231F20}
% Primary Colours
\definecolor{Red}{HTML}{eb3b47}
\definecolor{LightRed}{HTML}{fc5c65}
\definecolor{DarkRed}{HTML}{d91c38}

\definecolor{Yellow}{HTML}{f7b731}
\definecolor{LightYellow}{HTML}{fed330}
\definecolor{DarkYellow}{HTML}{f0a132}

\definecolor{Blue}{HTML}{3867d6}
\definecolor{LightBlue}{HTML}{4b7bec}
\definecolor{DarkBlue}{HTML}{2654bf}
% tertiary Colours
\definecolor{Green}{HTML}{20bf6b}
\definecolor{LightGreen}{HTML}{26de81}
\definecolor{DarkGreen}{HTML}{1ba155}

\definecolor{Orange}{HTML}{fa8231}
\definecolor{LightOrange}{HTML}{fd9644}
\definecolor{DarkOrange}{HTML}{f76b20}

\definecolor{Purple}{HTML}{8854d0}
\definecolor{LightPurple}{HTML}{a55eea}
\definecolor{DarkPurple}{HTML}{6e49b8}
% secondary+ Colours
% \definecolor{Brown}{HTML}{}
\definecolor{Cyan}{HTML}{0fb9b1}
\definecolor{LightCyan}{HTML}{2bcbba}
\definecolor{DarkCyan}{HTML}{00a8a8}


\usetikzlibrary{calc}

\usepackage[many]{tcolorbox}

\newtcolorbox{boxx}[2][]{
    enhanced,
    fonttitle=\fontsize{11}{13.2}\selectfont\bfseries,
    coltitle=text,
    colback=infoBulleBackground,
    colbacktitle=infoBulleBackground,
    colframe=infoBulleBackground,
    toprule=4pt,titlerule=0pt,bottomrule=0pt,rightrule=0pt,leftrule=0pt,
    segmentation hidden,
    sharpish corners,
    overlay={\draw[line width=2.5pt,#1] (frame.north west)--(frame.south west);},
    % margins & padding
    before skip=\baselineskip,
    after skip=\baselineskip,
    boxsep=2pt,
    top=6pt,
    bottom=4pt,
    left=8pt,
    right=6pt,
    breakable,
    title=#2,
    extras middle and last={%
        %top=20pt,
        overlay={%
            \draw[line width=2.5pt,#1] (frame.north west)--(frame.south west);
            % \draw[line width=0.5pt,dashed,#1] (frame.north west)--(frame.north east);
            % \draw[] {([yshift=-12pt,xshift=4.5pt]frame.north west)} node[anchor=west] {#2};
            % \draw[] {([yshift=-12pt,xshift=-4.5pt]frame.north east)} node[anchor=east] {eg};
        }
    },
}

\definecolor{codegreen}{HTML}{369432}
\definecolor{codegray}{HTML}{3a3d41}
\definecolor{codepurple}{HTML}{c586c0}
\definecolor{backcolour}{HTML}{1e1e1e}
\definecolor{standartcolor}{HTML}{cccccc}
\definecolor{codeorange}{HTML}{f48771}

\setlength{\parindent}{0pt}
\definecolor{bg}{rgb}{0.13, 0.13, 0.13}
%\pagecolor{bg}
%\color{white}
\begin{document}
%---COLORS---
\definecolor{pRed}{RGB}{255, 55, 55}
\definecolor{pOrange}{RGB}{219, 132, 36}

%\pagestyle{fancy}
%\fancyhf{}
%\lhead{Moathekuchenaufgabe}
%\rhead{Lösung}
%\cfoot{\thepage}

\setcounter{section}{5}
\section{Geraden und Ebenen}
\subsection{Vektoren im Raum}
Vektoren kommen hauptsächlich auf folgende 3 Arten und Weisen vor:
\begin{figure}[H]
    \centering
    \begin{subfigure}{.33\textwidth}
        \centering
        \begin{tikzpicture}
            \tkzInit[xmax=4,ymax=3]
            \tkzDrawX
            \tkzDrawY
            \tkzDefPoint(1,1){A}
            \tkzDefPoint(2,2){B}
            \tkzDefPoint(3,1){C}
            \tkzDefPoint(4,2){D}
            \tkzDrawSegment[-Triangle](A,B)
            \tkzDrawSegment[-Triangle](C,D)
            \tkzLabelPoint[below](1,0){$1$}
            \tkzLabelPoint[left](0,1){$1$}
            \tkzLabelSegment[right=1ex,pos=.4](A,B){$\vec{a}$}
            \tkzLabelSegment[right=1ex,pos=.4](C,D){$\vec{a}$}
        \end{tikzpicture}
        \caption{Als Pfeil (mit beliebigem Anfang)}
    \end{subfigure}%
    \begin{subfigure}{.33\textwidth}
        \centering
        \begin{tikzpicture}
            \tkzInit[xmax=4,ymax=3]
            \tkzDrawX
            \tkzDrawY
            \tkzDefPoint(1.5,1){A}
            \tkzDefPoint(4,3){B}
            \tkzDrawPoints[size=3,shape=cross out](A, B)
            \tkzDrawSegment[-Triangle](A,B)
            \tkzLabelPoint[below](1,0){$1$}
            \tkzLabelPoint[left](0,1){$1$}
            \tkzLabelPoint[below](A){$A(1.5,1)$}
            \tkzLabelPoint[left=1ex](B){$B(4,3)$}
            \tkzLabelSegment[right=2ex,pos=.35](A,B){$\vv{AB}$}
        \end{tikzpicture}
        \caption{Als Pfeil (zwischen 2 Punkten)}
    \end{subfigure}%
    \begin{subfigure}{.33\textwidth}
        \centering
        \begin{tikzpicture}
            \tkzInit[xmax=4,ymax=3]
            \tkzDrawX
            \tkzDrawY
            \tkzDefPoint(0,0){A}
            \tkzDefPoint(3,1){B}
            \tkzDrawPoint[size=3,shape=cross out](B)
            \tkzDrawSegment[-Triangle](A,B)
            \tkzLabelPoint[below](1,0){$1$}
            \tkzLabelPoint[left](0,1){$1$}
            \tkzLabelPoint[right=1ex](B){$A(3,1)$}
            \tkzText(2.5,2.8){\small$\displaystyle \vv{OA}=\begin{pmatrix}3\\1\end{pmatrix}$}
            \tkzText(2.5,2.2){\small\color{Red}$\text{Ortsvektor von }A$}
        \end{tikzpicture}
        \caption{Als Punkt}
    \end{subfigure}%
\end{figure}

\begin{boxx}[Red]{Gegenvektor}
    Gegenvektor eines Vektors $\vec{a}$ ist der Vektor $-\vec{a}$.
\end{boxx}
\begin{boxx}[DarkBlue]{Beispiel}
    Bestimme den Gegenvektor zum Vektor $\displaystyle \vv{AB} = \begin{pmatrix}3 \\ 1 \\ -2 \end{pmatrix}$.
    \[\vv{BA} = -\vv{AB} = \begin{pmatrix}-3 \\ -1 \\ 2\end{pmatrix}\]
\end{boxx}
\begin{boxx}[Red]{Mittelpunkt}
    Der Mittelpunkt $M$ zweier Punkte $A(a_1,a_2,a_3)$ und $B(b_1,b_2,b_3)$ ergibt sich wiefolt:
    \[M\left(\frac{a_1+b_2}{2}, \frac{a_2+b_2}{2}, \frac{a_3 + b_3}{2}\right)\]
\end{boxx}
\begin{boxx}[DarkBlue]{Beispiel}
    Bestimme den Mittelpunkt $M$ der Punkte $A(2,3,3)$ und $B(4,1,2)$.
    \[\Rightarrow M(3,2,2.5)\]
\end{boxx}
\begin{boxx}[Red]{Betrag}
    Der Betrag eines Vektors $\vec{a}$ ist geometrisch die Länge des zugehörigen Pfeils.
    Er lässt sich mit dem Satz des Pythagoras berechnen:
    \[|\vec{a}| = \left|\begin{pmatrix}
        a_1 \\ a_2 \\ a_3
    \end{pmatrix}\right| = \sqrt{a_1^2 + a_2^2 + a_3^2}\]
\end{boxx}
\begin{boxx}[DarkBlue]{Beispiel}
    Berechne den Betrag des Vektors $\displaystyle \vec{a} = \begin{pmatrix}3 \\ 2 \\ 6\end{pmatrix}$.
    \[|\vec{a}| = \sqrt{9 + 4 + 36} = \sqrt{49} = 7\]
\end{boxx}
\begin{boxx}[Red]{Einheitsvektor}
    Der Einheitsvektor $\vec{a}_0$ ist der Vektor, der in dieselbe Richtung wie $\vec{a}$ zeigt, und den Betrag 1 hat.
    
    Er errechnet sich mit:
    \[\vec{a}_0 = \frac{1}{|\vec{a}|} \cdot \vec{a}\]
\end{boxx} 
\begin{boxx}[DarkBlue]{Beispiel}
    Bestimme den Einheitsvektor $\vec{a}_0$ des Vektors $\vec{a} = \begin{pmatrix}3 \\ 2 \\ 4\end{pmatrix}$.
    \[\vec{a}_0 = \frac{1}{7} \cdot \begin{pmatrix}3\\2\\4\end{pmatrix} \\\]
\end{boxx}
\begin{boxx}[DarkBlue]{Beispiel}
    Gegeben ist der Vektor $\displaystyle \vv{AB} = \begin{pmatrix}3\\3\\3\end{pmatrix}$.
    Bestimme jeweils den fehlenden Punkt. \\
    $a)$\hspace{3mm} $A(0,-1,2)$
    \[\vv{OB} = \vv{OA} + \vv{AB} = \begin{pmatrix}0\\-1\\2\end{pmatrix} + \begin{pmatrix}3\\3\\3\end{pmatrix} = \begin{pmatrix}3\\2\\5\end{pmatrix}\]
    \[\Rightarrow B(3,2,5)\]
    $b)$\hspace{3mm} $B(2,0,3)$
    \[\vv{OA} = \vv{OB} - \vv{AB} = \begin{pmatrix}2\\0\\3\end{pmatrix} - \begin{pmatrix}3\\3\\3\end{pmatrix} = \begin{pmatrix}-1\\-3\\0\end{pmatrix}\]
    \[\Rightarrow A(-1,-3,0)\]
\end{boxx}
\newpage
\subsection{Geraden im Raum}
% Parametergleichung einer Geraden:
% \[\vec{x} = \underbrace{\begin{pmatrix}1\\2\\3\end{pmatrix}}_{\text{\makebox[0pt]{Stützvektor}}} + \;r \cdot \underbrace{\begin{pmatrix}1\\1\\1\end{pmatrix}}_{\text{\makebox[0pt]{Richtungsvektor}}}\]
\begin{boxx}[Red]{Allgemeine Parametergleichung einer Geraden}
    \[g: \vec{x} = \color{Red}\vec{p}\color{black} + t\cdot\color{Yellow}\vec{u}\]
    \begin{figure}[H]
        \centering
    \begin{tikzpicture}
        \tkzDefPoint(0,0){A}
        \tkzDefPoint(4,2){B}
        \tkzDefPoint(1,0.5){P}
        \tkzDefPoint(2,1){U}
        \tkzDefPoint(3,1.5){E}
        \tkzDrawSegment(A,B)
        \tkzDrawPoint[size=3,shape=cross out,color=Red](P)
        \tkzDrawPoint[size=3,shape=cross out](E)
        \tkzLabelPoint[above](P){$P$}
        \tkzDrawSegment[-Triangle,color=Yellow,thick](P,U)
        \tkzLabelSegment[below,color=Yellow](P,U){$\vec{u}$}
        \tkzLabelSegment[below,pos=0.97](A,B){$g$}
    \end{tikzpicture}
    \end{figure}
    $\color{Red} \vec{p}\color{black}:$ Stützvektor

    $\color{Yellow} \vec{u}\color{black}:$ Richtungsvektor
\end{boxx}
\begin{boxx}[Red]{Gegenseite Lage von Geraden}
    Es gibt vier mögliche gegenseitige Lagen zweier Geraden:
    \begin{itemize}
        \item parallel und verschieden (echt parallel)
        \item identisch
        \item sie schneiden sich in einem Punkt
        \item windschief
    \end{itemize}
    \begin{figure}[H]
        \centering
        \begin{forest}
            [Sind die Richtungsvektoren Vielfache?
                [\textbf{ja}: parallel oder identisch
                    [Haben sie gemeinsame Punkte?
                        [\textbf{ja}: identisch]
                        [\textbf{nein}: parallel]
                    ]
                ]
                [\textbf{nein}: schneiden sich oder sind windschief
                    [Haben sie gemeinsame Punkte?
                        [\textbf{ja}: schneiden sich]
                        [\textbf{nein}: windschief]
                    ]
                ]
            ]
        \end{forest}
    \end{figure}
\end{boxx}
\begin{boxx}[DarkBlue]{Beispiel}
    Untersuche die gegenseitige Lage der Geraden $g$ und $h$.\\\\
    $\displaystyle g:\vec{x} = \begin{pmatrix}1\\-1\\1\end{pmatrix} + r \cdot \begin{pmatrix}2\\3\\3\end{pmatrix};\; h: \vec{x} = \begin{pmatrix}1\\1\\0\end{pmatrix} + s \cdot \begin{pmatrix}1\\-1\\1\end{pmatrix}$ \\\\
    Die Richtungsvektoren sind keine Vielfachen $\rightarrow$ schneiden sich oder sind windschief
    \begin{align*}
        &&&&&g\cap h: &\begin{pmatrix}1\\-1\\1\end{pmatrix} + r \cdot \begin{pmatrix}2\\3\\3\end{pmatrix} = \begin{pmatrix}1\\1\\0\end{pmatrix} + s \cdot \begin{pmatrix}1\\-1\\1\end{pmatrix}&&&&&&
    \end{align*}
    \begin{align*}
        1+2r &= 1+s \\
        -1+3r &= 1-s \\
        1+ 3r &= s
    \end{align*}
    \begin{align}
        2r - s &= 0 \label{geraden_eq1}\\
        3r + s &= 2 \label{geraden_eq2}\\
        3r - s &= -1 \label{geraden_eq3}
    \end{align}
    \begin{align*}
        &&&&&\text{(\ref{geraden_eq2})} + \text{(\ref{geraden_eq3})}:& 6r &= 1 &&&&&&\\
        &&&&&& r &= \frac{1}{6} \\
        &&&&& r = \frac{1}{6} \text{ in (\ref{geraden_eq2})}:& 3 \cdot \frac{1}{6} + s &= 2 \\
        &&&&&& s &= 1.5 \\
        &&&&& r = \frac{1}{6};\; s = 1.5\text{ in (\ref{geraden_eq1})}:& \frac{1}{3} - 1.5 &\not= 0 \rightarrow \text{keine Schnittpunkte}
    \end{align*}
    \[\Rightarrow \text{windschief}\]

\end{boxx}
\end{document}