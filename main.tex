\documentclass{article}
\usepackage[ngerman]{babel} % |
\usepackage[utf8]{inputenc} % | Language and special characters
\usepackage[T1]{fontenc} %    |
\usepackage{amsmath} % |
\usepackage{amssymb} % | Math symbols and environments
\usepackage{amsthm} %  |
\usepackage{siunitx} %  Typesetting (SI) units
\DeclareSIUnit\litre{l}
%\usepackage{physics} % Advanced Mathematics of Physics, conflicts with siunitx
\usepackage{multirow} % Tables with cells that span more than one row
\usepackage{fancyhdr} % Custom page layout
\usepackage{graphicx} % Insert images
\usepackage{gensymb} % Generic units of measurement in math and text typeface
\usepackage{xfrac} % Split-level fractions
\usepackage{xcolor} % Handling colors
\usepackage{float} % Improved floating objects such as figures and tables
\usepackage{tikz} % Graphics and figures
\usetikzlibrary{arrows} % -> customizable arrow tips
\usepackage{pgfplots} % Plotting functions and data 
\usepgfplotslibrary{fillbetween} % -> filling areas in pgfplots
\pgfplotsset{compat=newest}
\usepackage{systeme} % Systems of equations
\usepackage{romanbar} % Roman numbers with bars
\usepackage{titlepic} % Putting picture on the title page
\usepackage{ragged2e} % Commands and environments for setting ragged text
\usepackage{multicol} % Multicolumn formating
\usepackage[a4paper, ignoreheadfoot, left=2.5cm, right=2.5cm, top=2cm, bottom=3.5cm, headsep=1cm]{geometry} % Margins etc.
\usepackage{cancel} % Cancel terms in equations
\usepackage[version=4]{mhchem} % Write chemical equations
\usepackage{wrapfig} % Figures with text wrapped around them
\usepackage{hyperref} % Extensive support for hypertext
\usepackage{sidecap} % Typeset captions sideways
\sidecaptionvpos{figure}{c} % -> Vertical alignment of caption
\usepackage{pdfpages} % Include PDF documents
\usepackage{setspace} % Set space between lines 
% (\singlespacing, \onehalfspacing, \doublespacing)
\usepackage{subcaption}
\usepackage[scale=0.96]{XCharter} % Use the XCharter text font
% \usepackage[xcharter]{newtxmath} % Set the math font

% Hyperref setup:
\hypersetup{
    colorlinks=true,
    linkcolor=black,
    filecolor=magenta,      
    urlcolor=blue,
    citecolor=black
}


\definecolor{page}{HTML}{FFFFFF} % white
\definecolor{text}{HTML}{000000} % black
\definecolor{primary}{HTML}{019875}
\definecolor{contrastColour}{HTML}{E8F1F2} % white
\definecolor{tertiary}{HTML}{C47238} % yellow
\definecolor{secondary}{HTML}{C6B53C} % orange
\definecolor{quaternary}{HTML}{BD3E4C} % red
\definecolor{alternativePrimary}{HTML}{13293D} % blue
\colorlet{infoBulleBackground}{page!98!text}
% Shades
\definecolor{LightGrey}{HTML}{a5b1c2}
\definecolor{Grey}{HTML}{778ca3}
\definecolor{DarkGrey}{HTML}{4b6584}
\definecolor{Black}{HTML}{231F20}
% Primary Colours
\definecolor{Red}{HTML}{eb3b47}
\definecolor{LightRed}{HTML}{fc5c65}
\definecolor{DarkRed}{HTML}{d91c38}

\definecolor{Yellow}{HTML}{f7b731}
\definecolor{LightYellow}{HTML}{fed330}
\definecolor{DarkYellow}{HTML}{f0a132}

\definecolor{Blue}{HTML}{3867d6}
\definecolor{LightBlue}{HTML}{4b7bec}
\definecolor{DarkBlue}{HTML}{2654bf}
% tertiary Colours
\definecolor{Green}{HTML}{20bf6b}
\definecolor{LightGreen}{HTML}{26de81}
\definecolor{DarkGreen}{HTML}{1ba155}

\definecolor{Orange}{HTML}{fa8231}
\definecolor{LightOrange}{HTML}{fd9644}
\definecolor{DarkOrange}{HTML}{f76b20}

\definecolor{Purple}{HTML}{8854d0}
\definecolor{LightPurple}{HTML}{a55eea}
\definecolor{DarkPurple}{HTML}{6e49b8}
% secondary+ Colours
% \definecolor{Brown}{HTML}{}
\definecolor{Cyan}{HTML}{0fb9b1}
\definecolor{LightCyan}{HTML}{2bcbba}
\definecolor{DarkCyan}{HTML}{00a8a8}


\usetikzlibrary{calc}

\usepackage[many]{tcolorbox}

\newtcolorbox{boxx}[2][]{
    enhanced,
    fonttitle=\fontsize{11}{13.2}\selectfont\bfseries,
    coltitle=text,
    colback=infoBulleBackground,
    colbacktitle=infoBulleBackground,
    colframe=infoBulleBackground,
    toprule=4pt,titlerule=0pt,bottomrule=0pt,rightrule=0pt,leftrule=0pt,
    segmentation hidden,
    sharpish corners,
    overlay={\draw[line width=2.5pt,#1] (frame.north west)--(frame.south west);},
    % margins & padding
    before skip=\baselineskip,
    after skip=\baselineskip,
    boxsep=2pt,
    top=6pt,
    bottom=4pt,
    left=8pt,
    right=6pt,
    breakable,
    title=#2,
    extras middle and last={%
        %top=20pt,
        overlay={%
            \draw[line width=2.5pt,#1] (frame.north west)--(frame.south west);
            % \draw[line width=0.5pt,dashed,#1] (frame.north west)--(frame.north east);
            % \draw[] {([yshift=-12pt,xshift=4.5pt]frame.north west)} node[anchor=west] {#2};
            % \draw[] {([yshift=-12pt,xshift=-4.5pt]frame.north east)} node[anchor=east] {eg};
        }
    },
}

\definecolor{codegreen}{HTML}{369432}
\definecolor{codegray}{HTML}{3a3d41}
\definecolor{codepurple}{HTML}{c586c0}
\definecolor{backcolour}{HTML}{1e1e1e}
\definecolor{standartcolor}{HTML}{cccccc}
\definecolor{codeorange}{HTML}{f48771}

\setlength{\parindent}{0pt}
\definecolor{bg}{rgb}{0.13, 0.13, 0.13}
%\pagecolor{bg}
%\color{white}
\begin{document}
%---COLORS---
\definecolor{pRed}{RGB}{255, 55, 55}
\definecolor{pOrange}{RGB}{219, 132, 36}

%\pagestyle{fancy}
%\fancyhf{}
%\lhead{Moathekuchenaufgabe}
%\rhead{Lösung}
%\cfoot{\thepage}

\setcounter{section}{3}
\section{Funktionen und ihre Graphen}
\subsection{Strecken, verschieben, spiegeln}
Gegeben sei der Graph der Funktion $f$. 
Der in \color{Green}$x$-Richtung verschobene\color{black}, in \color{Yellow}$y$-Richtung verschobene \color{black}
und in \color{Purple}$y$-Richtung \color{black} gestreckte Graph der Funktion $g$ 
besitzt den Funktionsterm:
\begin{align*}
    g(x) = \color{Purple} a \color{black}\cdot f(x \color{Green}- c\color{black}) \color{Yellow} + d
\end{align*}
Bei den Spiegelungen von $f$ gilt:
\begin{itemize}
    \item $g(x) = f(-x)\;\;\;$ Spiegelung an der \textbf{\emph{y}-Achse}
    \item $g(x) = -f(x)\;\;\;$ Spiegelung an der \textbf{\emph{x}-Achse}
    \item $g(x) = -f(-x)\;\;\;$ Spiegelung am \textbf{Ursprung}
\end{itemize}
\begin{boxx}[DarkBlue]{Beispiel}
    Skizziere die Graphen von $f$ und $g$.\\

    $a)$\hspace{3mm} $\displaystyle f(x) = - \sqrt{x+2} - 3;\;\;\;\; x\geq -2$ \\
    \begin{figure}[H]
        \centering
        \begin{tikzpicture}
        \begin{axis}[
            width = 12cm,
            height = 7cm,
            axis lines = middle,
            xlabel = {\(x\)},
            ylabel = {\(y\)},
            xmin = -3,
            xmax = 8,
            ymin = -7,
            ymax= 3,
        ]
        \addplot[dashed, thick, Blue, domain=-0:6,samples=100] {x^0.5} node[anchor = north west, pos = 1] {\(y = \sqrt{x}\)};
        \addplot[dashed, thick, Green, domain=-2:3,samples=100] {(x+2)^0.5} node[anchor = south east, pos = 0.3] {\(y = \sqrt{x+2}\)};
        \addplot[dashed, thick, Purple, domain=-2:3,samples=100] {-(x+2)^0.5} node[anchor = west, pos = 1] {\(y = -\sqrt{x+2}\)};
        \addplot[thick, smooth, Yellow, domain=-2:7.5,samples=100] {-(x+2)^0.5-3} node[anchor = south west, pos = 0.7] {\(y = -\sqrt{x+2}-3\)};
        \end{axis}
        \end{tikzpicture}
    \end{figure}
    $b)$\hspace{3mm} $\displaystyle g(x) = \frac{2}{x+1}- \frac{1}{2};\;\;\;\; x \not = -1$
    \begin{figure}[H]
        \centering
        \begin{tikzpicture}
        \begin{axis}[
            width = 11cm,
            height = 9cm,
            axis lines = middle,
            xlabel = {\(x\)},
            ylabel = {\(y\)},
            xmin = -7,
            xmax = 7,
            ymin = -7,
            ymax= 7,
            restrict y to domain=-12:12
        ]
        \addplot[dashed, thick, Blue, domain=-7:7,samples=200] {1/x} node[anchor = west, pos = 0.7] {\(\displaystyle y = \frac{1}{x}\)};
        \addplot[dashed, thick, Green, domain=-7:7,samples=200] {1/(x+1)} node[anchor = east, pos = 0.67] {\(\displaystyle y = \frac{1}{x+1}\)};
        \addplot[dashed, thick, Purple, domain=-7:7,samples=200] {2/(x+1)}node[anchor = south, pos = 0.92] {\(\displaystyle y = \frac{2}{x+1}\)};
        \addplot[thick, Yellow, domain=-7:7,samples=200] {2/(x+1)-0.5} node[anchor = north east, pos = 0.15] {\(\displaystyle y = \frac{2}{x+1} - \frac{1}{2}\)};
        \end{axis}
        \end{tikzpicture}
    \end{figure}
    Zeige, dass die Graphen von $f_k$ mit $\displaystyle f_k(x) = k x e^{x^2};\; k \in \mathbb{R}$
    punktsymmetrisch zum Ursprung sind.
    \begin{align*}
        f(-x) &= k \cdot (-x) \cdot e^{(-x)^2} \\
        &= - k x e^{x^2} \\
        &= -f(x) 
    \end{align*}
    \qed
\end{boxx}
\subsection{Linearfaktordarstellung - mehrfache Nullstellen}
\begin{boxx}[LightGreen]{Satz 1}
    Hat eine ganzrationale Funktion vom Grad $n$ eine Nullstelle
    $x_0$, so gilt:
    \begin{align*}
        f(x) &= (x - x_0)\cdot g(x) & &\text{wobei } g \text{ vom Grad } n-1 \text{ ist.}
    \end{align*}
    $(x- x_0)$ nennt man \textbf{Linearfaktor}.
\end{boxx}
\begin{boxx}[LightGreen]{Satz 2}
    Eine ganzrationale Funktion $n$-ten Grades besitzt höchstens $n$ Nullstellen.
\end{boxx}
\begin{boxx}[LightGreen]{Satz 3}
    Sei $\displaystyle f(x) = (x-a)^k \cdot g(x)$.
    \begin{itemize}
        \item Für $k = 1:$ Schnittstelle von $f$ mit der $x$-Achse.
        \item Für $k = 2:$ Berührstelle von $f$ an der $x$-Achse.
        \item Für $k = 3:$ Sattelstelle von $f$ an der $x$-Achse.
    \end{itemize}
\end{boxx}
\begin{boxx}[DarkBlue]{Beispiel}
    $a)$\hspace{3mm} Skizziere den Graph von $f$ mit $\displaystyle f(x) = x^3(x-2)^2(x+1)$.
    \begin{figure}[H]
        \centering
        \begin{tikzpicture}
        \begin{axis}[
            width = 10cm,
            height = 7cm,
            axis lines = middle,
            xlabel = {\(x\)},
            ylabel = {\(y\)},
            restrict y to domain=-8:8,
            xmax = 2.75,
            xmin = -1.75,
            ymax = 5,
            ymin = -1,
            ytick = {\empty},
        ]
            \addplot[thick, DarkBlue, samples = 100, domain = -1.75:2.75]{x^3*(x-2)^2*(x+1)} node[anchor = east, pos = 0.85] {\(G_f\)};
        \end{axis}
        \end{tikzpicture}
    \end{figure}
    \newpage
    $b)$\hspace{3mm} Bestimme die Funktionsgleichung des folgenden Graphen.
    \begin{figure}[H]
        \centering
        \begin{tikzpicture}
        \begin{axis}[
            width = 10cm,
            height = 7cm,
            axis lines = middle,
            xlabel = {\(x\)},
            ylabel = {\(y\)},
            restrict y to domain = -3:3,
            ymin = -2.5,
            ymax = 3.2,
            xtick = {-1,1,2}
        ]
            \addplot[thick, DarkBlue, domain = -2:3, samples = 100]{0.5*(x+1)^3*(x-1)*(x-2)^2} node[anchor = south west, pos = 0.1] {\(G_g\)};
        \end{axis}
        \end{tikzpicture}
    \end{figure}
    \begin{align*}
        g(x) &= a(x+1)^3 (x-1) (x-2)^2 \\
        \text{mit } a &= 1: \;g(0) = -4 \Rightarrow a = \frac{1}{2} \\
        g(x) &= \frac{1}{2} (x+1)^3 (x-1) (x-2)^2
    \end{align*}
\end{boxx}

\subsection{Lösen von Gleichungen}
Folgende Strategien zum Lösen von diversen Gleichungen sind
zielführend:
\subsubsection*{Betragsgleichungen}
Führe eine Fallunterscheidung durch:
\begin{itemize}
    \item für positive Beträge kann man den Betrag weglassen und
    die Gleichung wie gewohnt lösen.
    \item für negative Beträge wird eine Seite der Gleichung mit
    $-1$ multipliziert.
\end{itemize}
\begin{boxx}[DarkBlue]{Beispiel}
    \begin{align*}
        && \left|\frac{10}{e^x-1}\right| &= 2 \\
        &\text{Fall 1:} & \frac{10}{e^x-1} &= 2 & &|\; \cdot e^x - 1 \\ 
        && 10 &= 2e^x - 2 & &|\; + 2 ;\; \cdot \frac{1}{2} \\
        && 6 &= e^x  & &|\; \ln\\
        && x &= \ln 6 \\
        &\text{Fall 2:} & -\frac{10}{e^x-1} &= 2 & &|\; \cdot e^x - 1 \\
        && -10 &= 2e^x - 2 & &|\; + 2 ;\; \cdot \frac{1}{2} \\
        && -4 &= e^x \Rightarrow \text{keine Lösung} \\
        &\text{Probe:} & \left|\frac{10}{e^{\ln 6}-1}\right| &= \left|\frac{10}{5}\right| = 2 \\
        && &\Rightarrow \mathbb{L} = \{\ln 6\}
    \end{align*}
\end{boxx}
\subsubsection*{Wurzelgleichungen}
\begin{itemize}
    \item isoliere die Wurzel
    \item quadriere beide Seiten der Gleichung
\end{itemize}
\begin{boxx}[DarkBlue]{Beispiel}
    \begin{align*}
        \sqrt{20 - 2x} + 6 &= x & &|\; -6 \\
        \sqrt{20 - 2x} &= x-6 & &|\; ()^2 \\
        20 - 2x &= (x-6)^2 \\
        20 - 2x &= x^2 - 12x + 36 & &|\; -20 + 2x \\
        x^2 - 10x + 16 &= 0 \\
        (x-2)(x-8) &= 0 \\
        x_1 = 2;\; x_2 &= 8
    \end{align*}
    \begin{align*}
        &\text{Probe:} &  \sqrt{20 - 2\cdot 2} + 6 &= 2 \Leftrightarrow 4 + 6 \not = 2 &&\\
        & &  \sqrt{20 - 2\cdot 8} + 6 &= 8 \Leftrightarrow 2 + 6  = 8 &&\\
        & & \Rightarrow \mathbb{L} &= \{8\}
    \end{align*}
\end{boxx}
\subsubsection*{Bruchgleichungen}
\begin{itemize}
    \item Bestimme den Hauptnenner
    \item Beide Seiten mit dem Hauptnenner durchmultiplizieren
\end{itemize}
\begin{boxx}[DarkBlue]{Beispiel}
    \begin{align*}
        \frac{6}{x^4} - \frac{5}{x^2} &= -1 & &|\; \cdot x^4 \\
        6 - 5x^2 &= -x^4 & &|\; + x^4 \\
        x^4 - 5x^2 + 6 &= 0  & &|\; u = x^2 \\
        u^2 - 5u +6 &= 0 \\
        (u-2)(u-3) &= 0 \\
        u_1 = 2;\; u_2 &= 3 & &|\; x^2 = u
    \end{align*}
    \begin{align*}
        x^2 &= 2 & x^2 &= 3\\
        x_1 &= \pm \sqrt{2} &  x_2 &= \pm \sqrt{3}
    \end{align*}
    \begin{center}
        $x^4 \not = 0$ und $x^2 \not = 0$ für $x = \pm \sqrt{2}$ oder $x = \pm \sqrt{3}$
    \end{center}
    \[\Rightarrow \mathbb{L} = \{\sqrt{2};\,-\sqrt{2};\,\sqrt{3};\,-\sqrt{3}\}\]
\end{boxx}
\newpage
\subsubsection*{Ungleichungen}
Entweder: Mit Vergleichszeichen auflösen und aufpassen bei
Mulitplikation oder Division mit negativen Zahlen.
\\\\
Oder: Eine Gleichung lösen und Werte größer und kleiner als
die Lösung testen.
\begin{boxx}[DarkBlue]{Beispiel}
    \begin{align*}
        1 - \left(\frac{1}{2}\right)^x &< 0.05 & &|\; - 1 \\
        - \left(\frac{1}{2}\right)^x &< -0.95 & &|\; \cdot (-1) \\
        \left(\frac{1}{2}\right)^x &> 0.95 & &|\; \log \\
        x \log 0.5 &> \log 0.95 & &|\; \cdot \frac{1}{\log 0.5} \\
        x &< \frac{\log 0.95}{\log 0.5} \approx 0.074
    \end{align*}
\end{boxx}

\subsection{Trigonometrische Funktionen}

Gegeben sei die Funktion $f$ mit $f(x) = \sin x$.
Der Graph von der Funktion $g$ mit
\begin{align*}
    g(x) = a \sin\left(b(x-c)\right) + d
\end{align*}
ist gegenüber dem Graph von $f$
\begin{itemize}
    \item um $|a|$ Einheiten in $y$-Richtung gestreckt,
    \item um $d$ Einheiten in y-Richtung verschoben,
    \item besitzt die Periode 
    $\displaystyle p = \frac{2 \pi}{b}$ 
    (Streckung in $x$-Richtung) und
    \item um $c$ Einheiten in $x$-Richtung verschoben.
\end{itemize}

Für $a<0$ wird der Graph zusätzlich an der $x$-Achse
gespiegelt.

\begin{boxx}[DarkBlue]{Beispiel}
    $a)$\hspace{3mm} Gib im Intervall $I=[0;2\pi]$
    zwei Lösungen der Gleichung $\sin x = -0.6$ an.
    \begin{figure}[H]
        \centering
        \begin{tikzpicture}
        \begin{axis}[
            clip = false,
            height=5cm,
            width=11cm,
            axis lines = middle,
            xlabel = {\(x\)},
            ylabel = {\(y\)},
            xtick={-6.28318, -4.7123889, -3.14159, -1.5708, 1.5708, 3.14159, 4.7123889, 6.28318},
            xticklabels={$-2\pi$, $-\frac{3\pi}{2}$,$-\pi$, $-\frac{\pi}{2}$, 
             $\frac{\pi}{2}$,$\pi$, $\frac{3\pi}{2}$, $2\pi$},
            xmin=-4.5,
            xmax=8,
            ymin=-1.2,
            ymax=1.2, 
        ]
        \addplot[thick, DarkBlue, domain=-3.14159:6.28318, samples =100]{sin(deg(x))};
        \addplot[LightRed, thick, domain=-4:7]{-0.6} node[anchor=west, pos = 1] {\(y=-0.6\)};
        \node[label={315:{\(x_2\)}},circle,fill,inner sep=1.5pt] at (axis cs:5.64,-0.6) {};
        \node[label={225:{\(x_1\)}},circle,fill,inner sep=1.5pt] at (axis cs:3.78,-0.6) {};
        \addplot[] coordinates {
            (-0.64,0) (-0.64,-0.6)
        } node[anchor = south, pos = 0] {\(-0.64\)};
        \end{axis}
        \end{tikzpicture}
    \end{figure}
    \begin{align*}
        \sin^{-1} (-0.6) &\approx -0.64 \\
        x_2 &= -0.64 + 2\pi \approx 5.64 \\ 
        x_1 &= \pi + 0.64 \approx 3.78
    \end{align*}

    $b)$\hspace{3mm} Skizziere den Graphen von
    $f(x) = 2\sin\left(2\pi(x - 1)\right) - 1$.
    \begin{figure}[H]
        \centering
        \begin{tikzpicture}
        \begin{axis}[
            height=6cm,
            width=12cm,
            axis lines = middle,
            xlabel = {\(x\)},
            ylabel = {\(y\)},
            ytick = {1, -1, -2, -3},
            xmin=-3.5,
            xmax=3.5,
            ymin=-3.5,
            ymax=1.6,
            grid = major,
            clip = false
        ]
        \addplot[thick, DarkBlue, domain=-3.2:3.2, samples=200, smooth]{2*sin(deg(2*pi*(x-1)))-1} node[pos=0.01, anchor = east] {\(G_f\)};
        \end{axis}
        \end{tikzpicture}
    \end{figure}
\end{boxx}

\subsection{Senkrechte und waagerechte Asymptoten}

Gegeben sind die ganzrationalen Funktionen $g(x)$ und
$h(x)$. Die Funktion $f$ mit $\displaystyle f(x) = \frac{g(x)}{h(x)}\;\;\;(h(x)\not= 0)$
nennt man \textbf{gebrochenrationale Funktion}.

Wenn $g\left(x_0\right) \not= 0$ und $h\left(x_0\right) = 0$ gilt,
dann ist $x_0$ eine \textbf{Polstelle} von $f$ und die Gerade mit
$x=x_0$ ist eine \textbf{senkrechte Asymptote} des Graphen von $f$.

Gilt $g\left(x_0\right) = 0$ und $h\left(x_0\right) = 0$,
dann liegt keine senkrechte Asymptote, sondern eine \textbf{hebbare Definitionslücke} vor.

Für die waagerechte Asymptote gilt:
\begin{enumerate}
    \item Zählergrad > Nennergrad: keine waagerechte Asymptote
    \item Zählergrad = Nennergrad: höchste Potenz von $x$ ausklammern und 
    $\displaystyle \lim_{x\to\pm\infty} f(x)$ bilden. $\displaystyle \left(y = \frac{a}{b}\right)$
    \item Zählergrad < Nennergrad: waagerechte Asymptote bei $y = 0$.
\end{enumerate}
\begin{boxx}[DarkBlue]{Beispiel}
    $a)$\hspace{3mm} Bestimme die senkrechte und waagerechte Asymptote.\\\\
    $\displaystyle f(x) = \frac{6x^2+3}{5x^2- \sfrac{1}{2}}$ 
    \begin{align*}
        5x^2 - \frac{1}{2} = 0 \Leftrightarrow x &= \pm\frac{1}{\sqrt{10}} & &\rightarrow \text{Senk. Asymp. bei } x = \pm\frac{1}{\sqrt{10}}\\
        \lim_{x\to\pm\infty} \frac{\cancel{x^2}\left(6 + \frac{3}{x^2}\right)}{\cancel{x^2}\left(5 - \frac{\sfrac{1}{2}}{x^2}\right)} &= \frac{6}{5}
        & &\rightarrow \text{Wag. Asymp. bei } y = \frac{6}{5}
    \end{align*}
    $\displaystyle g(x) = \frac{7x}{x^2 - 1}$
    \begin{align*}
        x^2 - 1 = 0 \Leftrightarrow x &= \pm 1 & &\rightarrow \text{Senk. Asymp. bei } x = \pm 1\\
        & & &\rightarrow \text{Wag. Asymp. bei } y = 0
    \end{align*}
    $\displaystyle h(x) = \frac{3x^2 + 2}{7x - 2}$ \\
    \begin{align*}
        7x - 2 = 0 \Leftrightarrow x &= \frac{2}{7} & &\rightarrow \text{Senk. Asymp. bei } x = \frac{2}{7} \\
        & & &\rightarrow \text{keine wag. Asymp.}
    \end{align*}
    \newpage
    $b)$\hspace{3mm} Skizziere den Graphen von $t$ mit 
    $\displaystyle t(x) = \frac{3x^2 + 1}{x^2 + 4x + 4}$.
    \begin{figure}[H]
        \centering
        \begin{tikzpicture}
        \begin{axis}[
            height=7cm,
            width=10cm,
            axis lines = middle,
            xlabel = {\(x\)},
            ylabel = {\(y\)},
            xmax= 15,
            xmin = -15,
            restrict y to domain = -20:20
        ]
        \addplot[thick, DarkBlue, domain=-20:20, samples = 500]{(3*x^2+1)/(x^2 +4*x + 4)} node[pos=0.1, anchor = south] {\(G_f\)};
        \addplot[thick,dashed, LightPurple, domain=-15:15]{3} node[pos=0.57, anchor=south]{\(y = 3\)};
        \addplot[thick,dashed, LightPurple, domain=-15:15] coordinates {(-2,20) (-2, 0)} node[pos=0.92,anchor=east] {\(x=-2\)};
        \end{axis}
        \end{tikzpicture}
    \end{figure}
\end{boxx}
\end{document}