\documentclass{article}
\usepackage[ngerman]{babel} % |
\usepackage[utf8]{inputenc} % | Language and special characters
\usepackage[T1]{fontenc} %    |
\usepackage{amsmath} % |
\usepackage{amssymb} % | Math symbols and environments
\usepackage{amsthm} %  |
\usepackage{siunitx} %  Typesetting (SI) units
\DeclareSIUnit\litre{l}
%\usepackage{physics} % Advanced Mathematics of Physics, conflicts with siunitx
\usepackage{multirow} % Tables with cells that span more than one row
\usepackage{fancyhdr} % Custom page layout
\usepackage{graphicx} % Insert images
\usepackage{gensymb} % Generic units of measurement in math and text typeface
\usepackage{xfrac} % Split-level fractions
\usepackage{xcolor} % Handling colors
\usepackage{float} % Improved floating objects such as figures and tables
\usepackage{tikz} % Graphics and figures
\usepackage{tikz-3dplot}
\usetikzlibrary{arrows} % -> customizable arrow tips
\usetikzlibrary{tikzmark}
\usepackage{pgfplots} % Plotting functions and data 
\usepgfplotslibrary{fillbetween} % -> filling areas in pgfplots
\pgfplotsset{compat=newest}
\usepackage{tkz-base,tkz-euclide}
\usepackage{systeme} % Systems of equations
\usepackage{romanbar} % Roman numbers with bars
\usepackage{titlepic} % Putting picture on the title page
\usepackage{ragged2e} % Commands and environments for setting ragged text
\usepackage{multicol} % Multicolumn formating
\usepackage[a4paper, ignoreheadfoot, left=2.5cm, right=2.5cm, top=2cm, bottom=3.5cm, headsep=1cm]{geometry} % Margins etc.
\usepackage{cancel} % Cancel terms in equations
\usepackage[version=4]{mhchem} % Write chemical equations
\usepackage{wrapfig} % Figures with text wrapped around them
\usepackage{hyperref} % Extensive support for hypertext
\usepackage{sidecap} % Typeset captions sideways
\sidecaptionvpos{figure}{c} % -> Vertical alignment of caption
\usepackage{pdfpages} % Include PDF documents
\usepackage{setspace} % Set space between lines 
% (\singlespacing, \onehalfspacing, \doublespacing)
\usepackage{subcaption}
\usepackage[scale=0.96]{XCharter} % Use the XCharter text font
% \usepackage[xcharter]{newtxmath} % Set the math font
\usepackage{esvect} % Nicer vectors (\vv{})
\usepackage[linguistics]{forest}
\usepackage{etoolbox}
\AtBeginEnvironment{align}{\setcounter{equation}{0}}
% Hyperref setup:
\hypersetup{
    colorlinks=true,
    linkcolor=black,
    filecolor=magenta,      
    urlcolor=blue,
    citecolor=black
}


\definecolor{page}{HTML}{FFFFFF} % white
\definecolor{text}{HTML}{000000} % black
\definecolor{primary}{HTML}{019875}
\definecolor{contrastColour}{HTML}{E8F1F2} % white
\definecolor{tertiary}{HTML}{C47238} % yellow
\definecolor{secondary}{HTML}{C6B53C} % orange
\definecolor{quaternary}{HTML}{BD3E4C} % red
\definecolor{alternativePrimary}{HTML}{13293D} % blue
\colorlet{infoBulleBackground}{page!98!text}
% Shades
\definecolor{LightGrey}{HTML}{a5b1c2}
\definecolor{Grey}{HTML}{778ca3}
\definecolor{DarkGrey}{HTML}{4b6584}
\definecolor{Black}{HTML}{231F20}
% Primary Colours
\definecolor{Red}{HTML}{eb3b47}
\definecolor{LightRed}{HTML}{fc5c65}
\definecolor{DarkRed}{HTML}{d91c38}

\definecolor{Yellow}{HTML}{f7b731}
\definecolor{LightYellow}{HTML}{fed330}
\definecolor{DarkYellow}{HTML}{f0a132}

\definecolor{Blue}{HTML}{3867d6}
\definecolor{LightBlue}{HTML}{4b7bec}
\definecolor{DarkBlue}{HTML}{2654bf}
% tertiary Colours
\definecolor{Green}{HTML}{20bf6b}
\definecolor{LightGreen}{HTML}{26de81}
\definecolor{DarkGreen}{HTML}{1ba155}

\definecolor{Orange}{HTML}{fa8231}
\definecolor{LightOrange}{HTML}{fd9644}
\definecolor{DarkOrange}{HTML}{f76b20}

\definecolor{Purple}{HTML}{8854d0}
\definecolor{LightPurple}{HTML}{a55eea}
\definecolor{DarkPurple}{HTML}{6e49b8}
% secondary+ Colours
% \definecolor{Brown}{HTML}{}
\definecolor{Cyan}{HTML}{0fb9b1}
\definecolor{LightCyan}{HTML}{2bcbba}
\definecolor{DarkCyan}{HTML}{00a8a8}


\usetikzlibrary{calc}

\usepackage[many]{tcolorbox}

\newtcolorbox{boxx}[2][]{
    enhanced,
    fonttitle=\fontsize{11}{13.2}\selectfont\bfseries,
    coltitle=text,
    colback=infoBulleBackground,
    colbacktitle=infoBulleBackground,
    colframe=infoBulleBackground,
    toprule=4pt,titlerule=0pt,bottomrule=0pt,rightrule=0pt,leftrule=0pt,
    segmentation hidden,
    sharpish corners,
    overlay={\draw[line width=2.5pt,#1] (frame.north west)--(frame.south west);},
    % margins & padding
    before skip=\baselineskip,
    after skip=\baselineskip,
    boxsep=2pt,
    top=6pt,
    bottom=4pt,
    left=8pt,
    right=6pt,
    breakable,
    title=#2,
    extras middle and last={%
        %top=20pt,
        overlay={%
            \draw[line width=2.5pt,#1] (frame.north west)--(frame.south west);
            % \draw[line width=0.5pt,dashed,#1] (frame.north west)--(frame.north east);
            % \draw[] {([yshift=-12pt,xshift=4.5pt]frame.north west)} node[anchor=west] {#2};
            % \draw[] {([yshift=-12pt,xshift=-4.5pt]frame.north east)} node[anchor=east] {eg};
        }
    },
}

\definecolor{codegreen}{HTML}{369432}
\definecolor{codegray}{HTML}{3a3d41}
\definecolor{codepurple}{HTML}{c586c0}
\definecolor{backcolour}{HTML}{1e1e1e}
\definecolor{standartcolor}{HTML}{cccccc}
\definecolor{codeorange}{HTML}{f48771}

\setlength{\parindent}{0pt}
\definecolor{bg}{rgb}{0.13, 0.13, 0.13}
%\pagecolor{bg}
%\color{white}
\begin{document}
%---COLORS---
\definecolor{pRed}{RGB}{255, 55, 55}
\definecolor{pOrange}{RGB}{219, 132, 36}

%\pagestyle{fancy}
%\fancyhf{}
%\lhead{Moathekuchenaufgabe}
%\rhead{Lösung}
%\cfoot{\thepage}

\setcounter{section}{5}
\section{Geraden und Ebenen}
\subsection{Vektoren im Raum}
Vektoren kommen hauptsächlich auf folgende 3 Arten und Weisen vor:
\begin{figure}[H]
    \centering
    \begin{subfigure}{.33\textwidth}
        \centering
        \begin{tikzpicture}
            \tkzInit[xmax=4,ymax=3]
            \tkzDrawX
            \tkzDrawY
            \tkzDefPoint(1,1){A}
            \tkzDefPoint(2,2){B}
            \tkzDefPoint(3,1){C}
            \tkzDefPoint(4,2){D}
            \tkzDrawSegment[-Triangle](A,B)
            \tkzDrawSegment[-Triangle](C,D)
            \tkzLabelPoint[below](1,0){$1$}
            \tkzLabelPoint[left](0,1){$1$}
            \tkzLabelSegment[right=1ex,pos=.4](A,B){$\vec{a}$}
            \tkzLabelSegment[right=1ex,pos=.4](C,D){$\vec{a}$}
        \end{tikzpicture}
        \caption{Als Pfeil (mit beliebigem Anfang)}
    \end{subfigure}%
    \begin{subfigure}{.33\textwidth}
        \centering
        \begin{tikzpicture}
            \tkzInit[xmax=4,ymax=3]
            \tkzDrawX
            \tkzDrawY
            \tkzDefPoint(1.5,1){A}
            \tkzDefPoint(4,3){B}
            \tkzDrawPoints[size=3,shape=cross out](A, B)
            \tkzDrawSegment[-Triangle](A,B)
            \tkzLabelPoint[below](1,0){$1$}
            \tkzLabelPoint[left](0,1){$1$}
            \tkzLabelPoint[below](A){$A(1.5,1)$}
            \tkzLabelPoint[left=1ex](B){$B(4,3)$}
            \tkzLabelSegment[right=2ex,pos=.35](A,B){$\vv{AB}$}
        \end{tikzpicture}
        \caption{Als Pfeil (zwischen 2 Punkten)}
    \end{subfigure}%
    \begin{subfigure}{.33\textwidth}
        \centering
        \begin{tikzpicture}
            \tkzInit[xmax=4,ymax=3]
            \tkzDrawX
            \tkzDrawY
            \tkzDefPoint(0,0){A}
            \tkzDefPoint(3,1){B}
            \tkzDrawPoint[size=3,shape=cross out](B)
            \tkzDrawSegment[-Triangle](A,B)
            \tkzLabelPoint[below](1,0){$1$}
            \tkzLabelPoint[left](0,1){$1$}
            \tkzLabelPoint[right=1ex](B){$A(3,1)$}
            \tkzText(2.5,2.8){\small$\displaystyle \vv{OA}=\begin{pmatrix}3\\1\end{pmatrix}$}
            \tkzText(2.5,2.2){\small\color{Red}$\text{Ortsvektor von }A$}
        \end{tikzpicture}
        \caption{Als Punkt}
    \end{subfigure}%
\end{figure}

\begin{boxx}[Red]{Gegenvektor}
    Gegenvektor eines Vektors $\vec{a}$ ist der Vektor $-\vec{a}$.
\end{boxx}
\begin{boxx}[DarkBlue]{Beispiel}
    Bestimme den Gegenvektor zum Vektor $\displaystyle \vv{AB} = \begin{pmatrix}3 \\ 1 \\ -2 \end{pmatrix}$.
    \[\vv{BA} = -\vv{AB} = \begin{pmatrix}-3 \\ -1 \\ 2\end{pmatrix}\]
\end{boxx}
\begin{boxx}[Red]{Mittelpunkt}
    Der Mittelpunkt $M$ zweier Punkte $A(a_1,a_2,a_3)$ und $B(b_1,b_2,b_3)$ ergibt sich wiefolt:
    \[M\left(\frac{a_1+b_2}{2}, \frac{a_2+b_2}{2}, \frac{a_3 + b_3}{2}\right)\]
\end{boxx}
\begin{boxx}[DarkBlue]{Beispiel}
    Bestimme den Mittelpunkt $M$ der Punkte $A(2,3,3)$ und $B(4,1,2)$.
    \[\Rightarrow M(3,2,2.5)\]
\end{boxx}
\begin{boxx}[Red]{Betrag}
    Der Betrag eines Vektors $\vec{a}$ ist geometrisch die Länge des zugehörigen Pfeils.
    Er lässt sich mit dem Satz des Pythagoras berechnen:
    \[|\vec{a}| = \left|\begin{pmatrix}
        a_1 \\ a_2 \\ a_3
    \end{pmatrix}\right| = \sqrt{a_1^2 + a_2^2 + a_3^2}\]
\end{boxx}
\begin{boxx}[DarkBlue]{Beispiel}
    Berechne den Betrag des Vektors $\displaystyle \vec{a} = \begin{pmatrix}3 \\ 2 \\ 6\end{pmatrix}$.
    \[|\vec{a}| = \sqrt{9 + 4 + 36} = \sqrt{49} = 7\]
\end{boxx}
\begin{boxx}[Red]{Einheitsvektor}
    Der Einheitsvektor $\vec{a}_0$ ist der Vektor, der in dieselbe Richtung wie $\vec{a}$ zeigt, und den Betrag 1 hat.
    
    Er errechnet sich mit:
    \[\vec{a}_0 = \frac{1}{|\vec{a}|} \cdot \vec{a}\]
\end{boxx} 
\begin{boxx}[DarkBlue]{Beispiel}
    Bestimme den Einheitsvektor $\vec{a}_0$ des Vektors $\vec{a} = \begin{pmatrix}3 \\ 2 \\ 4\end{pmatrix}$.
    \[\vec{a}_0 = \frac{1}{7} \cdot \begin{pmatrix}3\\2\\4\end{pmatrix} \\\]
\end{boxx}
\begin{boxx}[DarkBlue]{Beispiel}
    Gegeben ist der Vektor $\displaystyle \vv{AB} = \begin{pmatrix}3\\3\\3\end{pmatrix}$.
    Bestimme jeweils den fehlenden Punkt. \\
    $a)$\hspace{3mm} $A(0,-1,2)$
    \[\vv{OB} = \vv{OA} + \vv{AB} = \begin{pmatrix}0\\-1\\2\end{pmatrix} + \begin{pmatrix}3\\3\\3\end{pmatrix} = \begin{pmatrix}3\\2\\5\end{pmatrix}\]
    \[\Rightarrow B(3,2,5)\]
    $b)$\hspace{3mm} $B(2,0,3)$
    \[\vv{OA} = \vv{OB} - \vv{AB} = \begin{pmatrix}2\\0\\3\end{pmatrix} - \begin{pmatrix}3\\3\\3\end{pmatrix} = \begin{pmatrix}-1\\-3\\0\end{pmatrix}\]
    \[\Rightarrow A(-1,-3,0)\]
\end{boxx}
\newpage
\subsection{Geraden im Raum}
% Parametergleichung einer Geraden:
% \[\vec{x} = \underbrace{\begin{pmatrix}1\\2\\3\end{pmatrix}}_{\text{\makebox[0pt]{Stützvektor}}} + \;r \cdot \underbrace{\begin{pmatrix}1\\1\\1\end{pmatrix}}_{\text{\makebox[0pt]{Richtungsvektor}}}\]
\begin{boxx}[Red]{Allgemeine Parametergleichung einer Geraden}
    \[g: \vec{x} = \color{Red}\vec{p}\color{black} + t\cdot\color{Yellow}\vec{u}\]
    \begin{figure}[H]
        \centering
    \begin{tikzpicture}
        \tkzDefPoint(0,0){A}
        \tkzDefPoint(4,2){B}
        \tkzDefPoint(1,0.5){P}
        \tkzDefPoint(2,1){U}
        \tkzDefPoint(3,1.5){E}
        \tkzDrawSegment(A,B)
        \tkzDrawPoint[size=3,shape=cross out,color=Red](P)
        \tkzDrawPoint[size=3,shape=cross out](E)
        \tkzLabelPoint[above](P){$P$}
        \tkzDrawSegment[-Triangle,color=Yellow,thick](P,U)
        \tkzLabelSegment[below,color=Yellow](P,U){$\vec{u}$}
        \tkzLabelSegment[below,pos=0.97](A,B){$g$}
    \end{tikzpicture}
    \end{figure}
    $\color{Red} \vec{p}\color{black}:$ Stützvektor

    $\color{Yellow} \vec{u}\color{black}:$ Richtungsvektor
\end{boxx}
\begin{boxx}[Red]{Gegenseite Lage von Geraden}
    Es gibt vier mögliche gegenseitige Lagen zweier Geraden:
    \begin{itemize}
        \item parallel und verschieden (echt parallel)
        \item identisch
        \item sie schneiden sich in einem Punkt
        \item windschief
    \end{itemize}
    \begin{figure}[H]
        \centering
        \begin{forest}
            [Sind die Richtungsvektoren Vielfache?
                [\textbf{ja}: parallel oder identisch
                    [Haben sie gemeinsame Punkte?
                        [\textbf{ja}: identisch]
                        [\textbf{nein}: parallel]
                    ]
                ]
                [\textbf{nein}: schneiden sich oder sind windschief
                    [Haben sie gemeinsame Punkte?
                        [\textbf{ja}: schneiden sich]
                        [\textbf{nein}: windschief]
                    ]
                ]
            ]
        \end{forest}
    \end{figure}
\end{boxx}
\begin{boxx}[DarkBlue]{Beispiel}
    Untersuche die gegenseitige Lage der Geraden $g$ und $h$.\\\\
    $\displaystyle g:\vec{x} = \begin{pmatrix}1\\-1\\1\end{pmatrix} + r \cdot \begin{pmatrix}2\\3\\3\end{pmatrix};\; h: \vec{x} = \begin{pmatrix}1\\1\\0\end{pmatrix} + s \cdot \begin{pmatrix}1\\-1\\1\end{pmatrix}$ \\\\
    Die Richtungsvektoren sind keine Vielfachen $\rightarrow$ schneiden sich oder sind windschief
    \begin{align*}
        &&&&&g\cap h: &\begin{pmatrix}1\\-1\\1\end{pmatrix} + r \cdot \begin{pmatrix}2\\3\\3\end{pmatrix} = \begin{pmatrix}1\\1\\0\end{pmatrix} + s \cdot \begin{pmatrix}1\\-1\\1\end{pmatrix}&&&&&&
    \end{align*}
    \begin{align*}
        1+2r &= 1+s \\
        -1+3r &= 1-s \\
        1+ 3r &= s
    \end{align*}
    \begin{align}
        2r - s &= 0 \label{geraden_eq1}\\
        3r + s &= 2 \label{geraden_eq2}\\
        3r - s &= -1 \label{geraden_eq3}
    \end{align}
    \begin{align*}
        &&&&&\text{(\ref{geraden_eq2})} + \text{(\ref{geraden_eq3})}:& 6r &= 1 &&&&&&\\
        &&&&&& r &= \frac{1}{6} \\
        &&&&& r = \frac{1}{6} \text{ in (\ref{geraden_eq2})}:& 3 \cdot \frac{1}{6} + s &= 2 \\
        &&&&&& s &= 1.5 \\
        &&&&& r = \frac{1}{6};\; s = 1.5\text{ in (\ref{geraden_eq1})}:& \frac{1}{3} - 1.5 &\not= 0 \rightarrow \text{keine Schnittpunkte}
    \end{align*}
    \[\Rightarrow \text{windschief}\]
\end{boxx}
\subsection{Ebenen im Raum}
\begin{figure}[H]
    \centering
    \tdplotsetmaincoords{60}{9}
    \begin{tikzpicture}[tdplot_main_coords]
        \filldraw[
        draw=Yellow,%
        fill=Yellow!20,%
    ]          (-4,-3,4)
            -- (4,-3,4)
            -- (4,3,4)
            -- (-4,3,4)
            -- cycle;
        \draw[->] (0,0,0) -- (-2,-1,4) node[anchor=north east,pos=0.5]{$\vec{a}$};
        \draw[->] (0,0,0) -- (-1,1,4) node[anchor=south west,pos=0.5]{$\vec{b}$};
        \draw[->] (0,0,0) -- (2.5,-1.5,4) node[anchor=north west,pos=0.5]{$\vec{c}$};
        \filldraw[black] (-2,-1,4) circle (1.5pt) node[anchor=south east]{$A$};
        \filldraw[black] (-1,1,4) circle (1.5pt) node[anchor=south east]{$B$};
        \filldraw[black] (2.5,-1.5,4) circle (1.5pt) node[anchor=south west]{$C$};
        \draw[Green,thick,->] (-2,-1,4) -- (-1,1,4) node[anchor=south east,pos=0.5]{$\vec{v}$};
        \draw[Red,thick,->] (-2,-1,4) -- (2.5,-1.5,4) node[anchor=south west,pos=0.5]{$\vec{u}$};
        \draw[->] (0,0,0) -- (2.2,1.5,4) node[anchor=south west,pos=1]{$x$};
        \node[Yellow] at (-3.5,-2,4) {$E$};
        \node[anchor=north east] at (0,0,0) {$O$};
    \end{tikzpicture}
\end{figure}
\begin{align*}
    E:\; \vec{x} &= \vv{OA} + r\cdot \vv{AC} + s \cdot \vv{AB} \hspace*{10mm} r,s \in \mathbb{R} \\
    E:\; \vec{x} &= \vv{OA} + r\cdot \color{Red} \vec{u} \color{black} + s\cdot \color{Green} \vec{v}
\end{align*}
\begin{boxx}[Red]{Parametergleichung einer Ebene}
    Jede Ebene lässt sich durch eine Parametergleichung der Form 
    \[\vec{x} = \vec{p} + r \cdot \vec{u} + s \cdot \vec{v}\]
    beschreiben. 
    
    $\vec{u}$ und $\vec{v}$ sind die Spannvektoren. 
    Sie dürfen keine Veilfachen voneinader sein.
    $\vec{p}$ ist der Stützvektor.
\end{boxx}
\newpage
\begin{boxx}[DarkBlue]{Beispiel}
    $a)$\hspace{3mm} Bestimme die Parametergleichung der Ebene, die durch die Punkte
    $A$, $B$ und $C$ verläuft.

    \[A(1,0,1);\; B(1,1,0);\; C(0,0,1)\]
    \begin{align*}
        E:\; \vec{x} &= \vv{OA} + r \cdot \vv{AC} + s \cdot \vv{AB} \\
        E:\; \vec{x} &= \begin{pmatrix}1\\0\\1\end{pmatrix} + r \cdot \begin{pmatrix}-1\\1\\0\end{pmatrix} + s \cdot \begin{pmatrix}0\\1\\-1\end{pmatrix} \\
    \end{align*}
    $b)$\hspace{3mm} Gegeben ist die Ebene $E$ mit der Gleichung 
    $\displaystyle \vec{x} = \begin{pmatrix}2\\0\\1\end{pmatrix} + r \cdot \begin{pmatrix}1\\3\\5\end{pmatrix} + s \cdot \begin{pmatrix}2\\-1\\1\end{pmatrix}$.\\

    \hspace{6mm} Bestimme, ob die Punkte $A(7,5,4)$ und $B(7,1,8)$ auf der Ebene $E$ liegen.\\
    \begin{align*}
        &&&A(7,5,4):    & \begin{pmatrix}7\\5\\4\end{pmatrix} &= \begin{pmatrix}2\\0\\1\end{pmatrix} + r \cdot \begin{pmatrix}1\\3\\5\end{pmatrix} + s \cdot \begin{pmatrix}2\\-1\\1\end{pmatrix} &&&&\\
        &&&&\begin{pmatrix}5\\5\\3\end{pmatrix} &= r \cdot \begin{pmatrix}1\\3\\5\end{pmatrix} + s \cdot \begin{pmatrix}2\\-1\\1\end{pmatrix}
    \end{align*}
    \begin{align}
        5 &= r + 2s \label{ebenen_eq1}\\
        5 &= 3r - s \label{ebenen_eq2}\\
        3 &= 5r + s \label{ebenen_eq3}
    \end{align}
    \begin{align*}
        &&&&&&&\text{(\ref{ebenen_eq2})} + \text{(\ref{ebenen_eq3})}:& 8 &= 8r &&&&&&&&\\
        &&&&&&&& \rightarrow r &= 1 \\
        &&&&&&& r = 1 \text{ in (\ref{ebenen_eq1})}:& 5 &= 1 + 2s \\
        &&&&&&&& \rightarrow s &= 2 \\
        &&&&&&& r = 1;\; s = 2\text{ in (\ref{ebenen_eq2})}:& 5 &\not= 3 - 2 \Rightarrow A \not\in E
    \end{align*}

    \begin{align*}
        &&&B(7,1,8):    & \begin{pmatrix}7\\1\\8\end{pmatrix} &= \begin{pmatrix}2\\0\\1\end{pmatrix} + r \cdot \begin{pmatrix}1\\3\\5\end{pmatrix} + s \cdot \begin{pmatrix}2\\-1\\1\end{pmatrix} &&&&\\
        &&&&\begin{pmatrix}5\\1\\7\end{pmatrix} &= r \cdot \begin{pmatrix}1\\3\\5\end{pmatrix} + s \cdot \begin{pmatrix}2\\-1\\1\end{pmatrix}
    \end{align*}
    \begin{align}
        5 &= r + 2s \label{ebenen2_eq1}\\
        1 &= 3r - s \label{ebenen2_eq2}\\
        7 &= 5r + s \label{ebenen2_eq3}
    \end{align}
    \begin{align*}
        &&&&&&&\text{(\ref{ebenen2_eq2})} + \text{(\ref{ebenen2_eq3})}:& 8 &= 8r &&&&&&&&\\
        &&&&&&&& \rightarrow r &= 1 \\
        &&&&&&& r = 1 \text{ in (\ref{ebenen2_eq1})}:& 5 &= 1 + 2s \\
        &&&&&&&& \rightarrow s &= 2 \\
        &&&&&&& r = 1;\; s = 2\text{ in (\ref{ebenen2_eq2})}:& 1 &= 3 - 2 \Rightarrow B\in E
    \end{align*}
    \newpage
    $c)$\hspace{3mm} Überprüfe ob die Punkte $A$, $B$, $C$ und $D$ in einer Ebene liegen.
    \[A(0,1,-1);\; B(2,3,5);\; C(-1,3,-1);\; D(2,2,2)\]
    \[E:\; \vec{x} = \begin{pmatrix}0\\1\\-1\end{pmatrix} + r \cdot \begin{pmatrix}2\\2\\6\end{pmatrix} + s \cdot \begin{pmatrix}-1\\2\\0\end{pmatrix}\hspace*{5mm} \text{(Durch }A,B,C\text{)}\]
    \begin{align*}
        \begin{pmatrix}2\\2\\2\end{pmatrix} &= \begin{pmatrix}0\\1\\-1\end{pmatrix} + r \cdot \begin{pmatrix}2\\2\\6\end{pmatrix} + s \cdot \begin{pmatrix}-1\\2\\0\end{pmatrix} \\
        \begin{pmatrix}2\\1\\3\end{pmatrix} &= r \cdot \begin{pmatrix}2\\2\\6\end{pmatrix} + s \cdot \begin{pmatrix}-1\\2\\0\end{pmatrix}
    \end{align*}
    \begin{align}
        2 &= 2r - s \label{ebenen3_eq1}\\
        1 &= 2r + 2s \label{ebenen3_eq2}\\
        3 &= 6r \label{ebenen3_eq3}
    \end{align}
    \begin{align*}
        &&&&&&&\text{(\ref{ebenen3_eq3})}:& 3 &= 6r &&&&&&&&\\
        &&&&&&&& \rightarrow r &= \frac{1}{2} \\
        &&&&&&& r = \frac{1}{2} \text{ in (\ref{ebenen3_eq1})}:& 2 &= 1 - s \\
        &&&&&&&& \rightarrow s &= -1 \\
        &&&&&&& r = \frac{1}{2};\; s = -1\text{ in (\ref{ebenen3_eq2})}:& 1 &\not= 1 - 2 \Rightarrow D\not\in E
    \end{align*}
    \begin{center}
        $\Rightarrow$ $A$, $B$, $C$ und $D$ liegen nicht in einer Ebene.
    \end{center}
\end{boxx}
\subsection{Zueinander orthogonale Vektoren -- Skalarprodukt}
\begin{boxx}[Red]{Skalarprodukt}
    Gegeben sind zwei Vektoren $\displaystyle \vec{a} = \begin{pmatrix}a_1\\a_2\\a_3\end{pmatrix}$
    und $\displaystyle \vec{b} = \begin{pmatrix}b_1\\b_2\\b_3\end{pmatrix}$.

    Der Term 
    \[\vec{a}\cdot \vec{b} = a_1b_1 + a_2b_2 + a_3b_3\]
    heißt \textbf{Skalarprodukt} der Vektoren $\vec{a}$ und $\vec{b}$.\\

    Zwei Vektoren sind genau dann orthogonal Zueinander, wenn ihr Skalarprodukt $0$ ist.
    \[\vec{a} \cdot \vec{b} = 0\]
\end{boxx}
\begin{boxx}[Purple]{Beweis}
    \begin{align*}
        \vec{a} \cdot \vec{a} &= a_1^2 + a_2^2 + a_3^2 
        = \left(\sqrt{a_1^2 + a_2^2 + a_3^2 }\right)^2 
        = \left|\vec{a}\right|^2
    \end{align*}
\end{boxx}
\begin{boxx}[DarkBlue]{Beispiel}
    Bestimme, ob sich die Geraden $g$ und $h$ orthogonal schneiden.
    \[g:\;\vec{x} = \begin{pmatrix}8\\-9\\7\end{pmatrix} + s \cdot \begin{pmatrix}2\\13\\1\end{pmatrix};\hspace*{3mm}h:\;\vec{x} = \begin{pmatrix}8\\-9\\7\end{pmatrix} + s \cdot \begin{pmatrix}1\\-2\\1\end{pmatrix}\]
    \[g \cap h: P(8,-9,7)\]
    \[\begin{pmatrix}2\\13\\1\end{pmatrix} \cdot \begin{pmatrix}1\\-2\\1\end{pmatrix} = 2 - 26 +1 \not= 0\]
    \begin{center}
        $\Rightarrow$ Sie schneiden sich nicht orthogonal.
    \end{center}
\end{boxx}
\subsection{Normalen- und Koordinatengleichung einer Ebene}
Der Punkt $P$ ist ein beliebiger Punkt in der  Ebene $E$.

Der Vektor $\vec{n}$ steht orthogonal auf der Ebene $E$ und wird
\textbf{Normalvektor} der Ebene $E$ genannt.
\begin{figure}[H]
    \centering
    \tdplotsetmaincoords{60}{9}
    \begin{tikzpicture}[tdplot_main_coords]
        \filldraw[
        draw=Yellow,%
        fill=Yellow!20,%
    ]          (-2,-3,4)
            -- (6,-3,4)
            -- (6,3,4)
            -- (-2,3,4)
            -- cycle;
        \draw[->] (0,0,0) -- (2,0,4) node[anchor=north west,pos=0.5]{$\vec{p}$};
        \filldraw[black] (2,0,4) circle (1.5pt) node[anchor=north west]{$P$};
        \draw[->] (2,0,4) -- (2,0,7) node[anchor=west,pos=0.45]{$\vec{n}$};
        \draw[->] (0,0,0) -- (0.5,-1.5,4) node[anchor=east,pos=0.5]{$\vec{x}$};
        \filldraw[black] (0.5,-1.5,4) circle (1.5pt) node[anchor=south east]{$X$};
        \draw[->] (2,0,4) -- (0.5,-1.5,4) node[anchor=south east,pos=0.35]{$\vv{PX}$};
        \node[Yellow] at (5.5,-2,4) {$E$};
        \node[anchor=north east] at (0,0,0) {$O$};
    \end{tikzpicture}
\end{figure}
Die Vektoren $\vec{n}$ und $\vec{x} - \vec{p}$ sind orthogonal,
daher gilt $\displaystyle \left(\vec{x} - \vec{p}\right)\cdot \vec{n} = 0$.

Auch umgekehrt gilt, dass ein Punkt $X$, der die Gleichung $\displaystyle \left(\vec{x} - \vec{p}\right)\cdot \vec{n} = 0$
erfüllt, in der Ebene $E$ liegt.
\begin{boxx}[Red]{Normalengleichung einer Ebene}
    Eine Ebene $E$ mit dem Stützvektor $\vec{p}$ und dem Normalvektor $\vec{n}$ wird beschreiben durch die Gleichung:
    \[E:\; \left(\vec{x} - \vec{p}\right)\cdot \vec{n} = 0\]
    Alle Punkte $X$, die diese Gleichung erfüllen, liegen in $E$.
\end{boxx}
\begin{boxx}[DarkBlue]{Beispiel}
    Gib die Normalengleichung der Ebene mit dem Normalvektor $\displaystyle \vec{n} = \begin{pmatrix}4\\1\\-2\end{pmatrix}$,
    die auf $P(1,2,3)$ liegt.
    \[E:\; \left[\vec{x} - \begin{pmatrix}1\\2\\3\end{pmatrix}\right]\cdot \begin{pmatrix}4\\1\\-2\end{pmatrix} = 0\]
\end{boxx}
Durch Ausmultiplizieren der Normalengleichung erhält man eine weitere Gleichung, um die Ebene zu beschreiben.
\begin{align*}
    \left[\vec{x} - \begin{pmatrix}1\\2\\3\end{pmatrix}\right]\cdot \begin{pmatrix}\color{Orange}4\\ \color{Orange}1\\ \color{Orange}-2\end{pmatrix} &= 0 \\
    \vec{x}\cdot \begin{pmatrix}4\\1\\-2\end{pmatrix} - \begin{pmatrix}1\\2\\3\end{pmatrix} \cdot \begin{pmatrix}4\\1\\-2\end{pmatrix} &= 0 \\
    \begin{pmatrix}x_1\\x_2\\x_3\end{pmatrix} \begin{pmatrix}4\\1\\-2\end{pmatrix} - \begin{pmatrix}1\\2\\3\end{pmatrix} \cdot \begin{pmatrix}4\\1\\-2\end{pmatrix} &= 0 \\
    4x_1 + x_2 - 2x_3 - \left(4+2-6\right) &= 0 \\
    \color{Orange}4\color{black}x_1 + \color{Orange}1\color{black}x_2 \color{Orange}- 2\color{black}x_3 &= 0 \\
\end{align*}
\begin{boxx}[Red]{Koordinatengleichung einer Ebene}
    Jede Ebene $E$ lässt sich durch eine \textbf{Koordinatengleichung} der Form
    \[ax_1 + bx_2 + cx_3 = d\]
    beschreiben. Mindestens einer der Koeffizienten muss ungleich 0 sein. \\

    Der Normalvektor $\vec{n}$ einer Ebene $E$ mit der Koordinatengleichung $ax_1 + bx_2 + cx_3 = d$
    ist $\displaystyle \vec{n} = \begin{pmatrix}a\\b\\c\end{pmatrix}$.
\end{boxx}
\begin{boxx}[DarkBlue]{Beispiel}
    Begründe, dass die Ebenen $E_1$ und $E_2$ parallel sind.
    \[E_1:\; 2x_1 + 3x_2 + 4x_3 = 5\]
    \[E_2:\; 2x_1 + 3x_2 + 4x_3 = 9\]
    \begin{center}
        $E_1$ und $E_2$ haben den gleichen Normalvektor.\\
        $\Rightarrow E_1 \parallel E_2$
    \end{center}
\end{boxx}
Die Koordinatenebenen lassen sich mit folgenden Koordinatengleichungen beschreiben:
\begin{itemize}
    \item $x_1x_2$-Ebene: $x_3 = 0$
    \item $x_1x_3$-Ebene: $x_2 = 0$
    \item $x_2x_3$-Ebene: $x_1 = 0$
\end{itemize}
\subsection{Ebenengleichungen umformen -- das Kreuzprodukt}
\[E:\; \vec{x} = \begin{pmatrix}1\\3\\2\end{pmatrix} + r \cdot \begin{pmatrix}-2\\-1\\-1\end{pmatrix} + s \cdot \begin{pmatrix}2\\-5\\3\end{pmatrix}\]
Um die Ebene $E$ mit einer Normalen- oder Koordinatengleichung zu beschreiben, 

braucht man $\vec{n}$ mit 
$\displaystyle \vec{n} \cdot \begin{pmatrix}-2\\-1\\-1\end{pmatrix} = 0$ 
und $\displaystyle \vec{n} \cdot \begin{pmatrix}2\\-5\\-3\end{pmatrix} = 0$.

\begin{boxx}[Red]{Kreuzprodukt}
    Unter dem Kreuzprodukt zweier Vektor $\vec{a}$ und $\vec{b}$ im $\mathbf{R^3}$
    versteht man den Vektor $\vec{a} \times \vec{b}$ ("kreuz") mit:
    \begin{align*}
        \vec{a} \times \vec{b} = \begin{pmatrix}
            a_2b_3 - a_3b_2 \\
            a_3b_1 - a_1b_3 \\
            a_1b_2 - a_2b_1
        \end{pmatrix}
    \end{align*}
    $\vec{a} \times \vec{b}$ ist orthogonal zu $\vec{a}$ und $\vec{b}$, 
    falls $\vec{a}$ und $\vec{b}$ keine Vielfachen sind.
    $\vec{a} \times \vec{b} = \vec{0}$ genau dann, wenn $\vec{a}$ und $\vec{b}$ Vielfache voneinader sind.
\end{boxx}
\begin{boxx}[Green]{Rechenverfahren}
    \begin{figure}[H]
        \centering
        \begin{subfigure}[b]{.2\textwidth}
            \centering
            \begin{align*}
                \vec{a} \times \vec{b} &= \begin{pmatrix}a_1\\a_2\\a_3\end{pmatrix} \times \begin{pmatrix}b_1\\b_2\\b_3\end{pmatrix}=
            \end{align*}
        \end{subfigure}%
        \begin{subfigure}[b]{.15\textwidth}
            \centering
            \begin{tabular}{ccc}
                \tikzmark{a1_1}$a_1$ && $b_1$\tikzmark{b1_1} \\ 
                $a_2$\tikzmark{a2_1} && \tikzmark{b2_1}$b_2$ \\ 
                $a_3$\tikzmark{a3_1} && \tikzmark{b3_1}$b_3$ \\ 
                $a_1$\tikzmark{a1_2} && \tikzmark{b1_2}$b_1$ \\ 
                $a_2$\tikzmark{a2_2} && \tikzmark{b2_2}$b_2$ \\ 
                \tikzmark{a3_2}$a_3$ &&$b_3$\tikzmark{b3_2}
            \end{tabular}
            \begin{tikzpicture}[overlay, remember picture, shorten >=7pt, shorten <=7pt, transform canvas={yshift=.25\baselineskip}]
                \draw [Red] ([xshift=-5pt]{pic cs:a2_1}) -- ([xshift=5pt]{pic cs:b3_1});
                \draw [LightBlue] ([xshift=-5pt]{pic cs:a3_1}) -- ([xshift=5pt]{pic cs:b2_1});
                \draw [Red] ([xshift=-5pt]{pic cs:a3_1}) -- ([xshift=5pt]{pic cs:b1_2});
                \draw [LightBlue] ([xshift=-5pt]{pic cs:a1_2}) -- ([xshift=5pt]{pic cs:b3_1});
                \draw [Red] ([xshift=-5pt]{pic cs:a1_2}) -- ([xshift=5pt]{pic cs:b2_2});
                \draw [LightBlue] ([xshift=-5pt]{pic cs:a2_2}) -- ([xshift=5pt]{pic cs:b1_2});
                \draw ([xshift=-7pt]{pic cs:a1_1}) -- ([xshift=7pt]{pic cs:b1_1});
                \draw ([xshift=-7pt]{pic cs:a3_2}) -- ([xshift=7pt]{pic cs:b3_2});
            \end{tikzpicture}
        \end{subfigure}%\
        \begin{subfigure}[b]{.2\textwidth}
            \centering
            \begin{align*}
                = \begin{pmatrix}
                    \color{Red}a_2b_3 \color{black}- \color{LightBlue}a_3b_2 \\
                    \color{Red}a_3b_1 \color{black}- \color{LightBlue}a_1b_3 \\
                    \color{Red}a_1b_2 \color{black}- \color{LightBlue}a_2b_1
                \end{pmatrix}
            \end{align*}
        \end{subfigure}
    \end{figure}
\end{boxx}
\begin{boxx}[DarkBlue]{Beispiel}
    Gib die Koordinatengleichung der Ebene $E$ mit $\displaystyle \vec{x} = \begin{pmatrix}1\\3\\2\end{pmatrix} + r \cdot \begin{pmatrix}-2\\-1\\-1\end{pmatrix} + s \cdot \begin{pmatrix}2\\-5\\3\end{pmatrix}$ an.
    \begin{align*}
        \vec{n}:\hspace*{5mm} \begin{pmatrix}-2\\-1\\-1\end{pmatrix} \times \begin{pmatrix}2\\-5\\-3\end{pmatrix} &=  \begin{pmatrix}3-5\\-2-6\\10+2\end{pmatrix} =  \begin{pmatrix}-2\\-8\\12\end{pmatrix} \\
        \vec{n} &=  \begin{pmatrix}-1\\-4\\6\end{pmatrix}
    \end{align*}
    \[E:\; -x_1 - 4x_2 + 6x_3 = d\]
    \begin{align*}
        \text{mit }P(1,3,2):\hspace*{5mm} -1-4\cdot3+6\cdot2 = d = - 1
    \end{align*}
    \[\Rightarrow E:\; -x_1 - 4x_2 + 6x_3 = -1\]
\end{boxx}
\[E:\; 2x_1 - 3x_2 -x_3 = 8\]
Um die Ebene $E$ mit einer Parametergleichung zu beschreiben, gibt es zwei Vorgehensweisen:
\begin{enumerate}
    \item 3 Punkte finden, die die Gleichung erfüllen, also in der Ebene liegen, die nicht auf einer Geraden liegen.
    \begin{center}
        Wir wählen
    \end{center}
    \[P_1(0,0,-8);\;P_2(4,0,0);\; P_3(5,0,2)\]
    \begin{align*}
        E:\; \vec{x} &= \vv{OP_1} + r \cdot \vv{P_1P_2} + s \cdot \vv{P_1P_3} \\
        \vec{x} &= \begin{pmatrix}0\\0\\-8\end{pmatrix} + r \cdot \color{Red}\begin{pmatrix}4\\0\\8\end{pmatrix} \color{black}+ s \cdot \color{Red}\begin{pmatrix}5\\0\\10\end{pmatrix}
    \end{align*}
    \begin{center}
        $\begin{pmatrix}4\\0\\8\end{pmatrix}$ und $\begin{pmatrix}5\\0\\10\end{pmatrix}$ sind Vielfache von $\begin{pmatrix}1\\0\\2\end{pmatrix}$.
        Wir wählen einen weiteren Punkt $P_4(1,-2,0)$ statt $P_3$.
    \end{center}
    \[\Rightarrow E:\; \vec{x} = \begin{pmatrix}0\\0\\-8\end{pmatrix} + r \cdot \begin{pmatrix}1\\-2\\8\end{pmatrix} + s \cdot \begin{pmatrix}4\\0\\8\end{pmatrix}\]
    \item Auflösen nach einer Variable und die anderen den Parametern $r$ und $s$ gleichsetzen.
    Dann die Vektoren so wählen, dass die Zeilen mit den Gleichungen übereinstimmen.
    \[E:\; 2x_1 - 3x_2 -x_3 = 8\]
    \begin{align*}
        x_1 &= r \\
        x_2 &= s \\
        x_3 &= 2x_1 - 3x_2 - 8 \\
        &= 2r - 3s -8
    \end{align*}
    \[\Rightarrow E:\; \begin{pmatrix}x_1\\x_2\\x_3\end{pmatrix} = \begin{pmatrix}0\\0\\-8\end{pmatrix} + r \cdot \begin{pmatrix}1\\0\\2\end{pmatrix} + s \cdot \begin{pmatrix}0\\1\\3\end{pmatrix}\]
\end{enumerate}
\end{document}