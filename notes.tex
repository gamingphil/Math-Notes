\documentclass{article}
\usepackage[ngerman]{babel} % |
\usepackage[utf8]{inputenc} % | Language and special characters
\usepackage[T1]{fontenc} %    |
\usepackage{amsmath} % |
\usepackage{amssymb} % | Math symbols and environments
\usepackage{amsthm} %  |
\usepackage{siunitx} %  Typesetting (SI) units
\DeclareSIUnit\litre{l}
%\usepackage{physics} % Advanced Mathematics of Physics, conflicts with siunitx
\usepackage{multirow} % Tables with cells that span more than one row
\usepackage{fancyhdr} % Custom page layout
\usepackage{graphicx} % Insert images
\usepackage{gensymb} % Generic units of measurement in math and text typeface
\usepackage{xfrac} % Split-level fractions
\usepackage{xcolor} % Handling colors
\usepackage{float} % Improved floating objects such as figures and tables
\usepackage{tikz} % Graphics and figures
\usetikzlibrary{arrows} % -> customizable arrow tips
\usetikzlibrary{tikzmark}
\usepackage{pgfplots} % Plotting functions and data 
\usepgfplotslibrary{fillbetween} % -> filling areas in pgfplots
\pgfplotsset{compat=newest}
\usepackage{systeme} % Systems of equations
\usepackage{romanbar} % Roman numbers with bars
\usepackage{titlepic} % Putting picture on the title page
\usepackage{ragged2e} % Commands and environments for setting ragged text
\usepackage{multicol} % Multicolumn formating
\usepackage[a4paper, ignoreheadfoot, left=2.5cm, right=2.5cm, top=2cm, bottom=3.5cm, headsep=1cm]{geometry} % Margins etc.
\usepackage{cancel} % Cancel terms in equations
\usepackage[version=4]{mhchem} % Write chemical equations
\usepackage{wrapfig} % Figures with text wrapped around them
\usepackage{hyperref} % Extensive support for hypertext
\usepackage{sidecap} % Typeset captions sideways
\sidecaptionvpos{figure}{c} % -> Vertical alignment of caption
\usepackage{pdfpages} % Include PDF documents
\usepackage{setspace} % Set space between lines 
% (\singlespacing, \onehalfspacing, \doublespacing)
\usepackage{subcaption}
\usepackage[scale=0.96]{XCharter} % Use the XCharter text font
% \usepackage[xcharter]{newtxmath} % Set the math font

% Hyperref setup:
\hypersetup{
    colorlinks=true,
    linkcolor=black,
    filecolor=magenta,      
    urlcolor=blue,
    citecolor=black
}


\definecolor{page}{HTML}{FFFFFF} % white
\definecolor{text}{HTML}{000000} % black
\definecolor{primary}{HTML}{019875}
\definecolor{contrastColour}{HTML}{E8F1F2} % white
\definecolor{tertiary}{HTML}{C47238} % yellow
\definecolor{secondary}{HTML}{C6B53C} % orange
\definecolor{quaternary}{HTML}{BD3E4C} % red
\definecolor{alternativePrimary}{HTML}{13293D} % blue
\colorlet{infoBulleBackground}{page!98!text}
% Shades
\definecolor{LightGrey}{HTML}{a5b1c2}
\definecolor{Grey}{HTML}{778ca3}
\definecolor{DarkGrey}{HTML}{4b6584}
\definecolor{Black}{HTML}{231F20}
% Primary Colours
\definecolor{Red}{HTML}{eb3b47}
\definecolor{LightRed}{HTML}{fc5c65}
\definecolor{DarkRed}{HTML}{d91c38}

\definecolor{Yellow}{HTML}{f7b731}
\definecolor{LightYellow}{HTML}{fed330}
\definecolor{DarkYellow}{HTML}{f0a132}

\definecolor{Blue}{HTML}{3867d6}
\definecolor{LightBlue}{HTML}{4b7bec}
\definecolor{DarkBlue}{HTML}{2654bf}
% tertiary Colours
\definecolor{Green}{HTML}{20bf6b}
\definecolor{LightGreen}{HTML}{26de81}
\definecolor{DarkGreen}{HTML}{1ba155}

\definecolor{Orange}{HTML}{fa8231}
\definecolor{LightOrange}{HTML}{fd9644}
\definecolor{DarkOrange}{HTML}{f76b20}

\definecolor{Purple}{HTML}{8854d0}
\definecolor{LightPurple}{HTML}{a55eea}
\definecolor{DarkPurple}{HTML}{6e49b8}
% secondary+ Colours
% \definecolor{Brown}{HTML}{}
\definecolor{Cyan}{HTML}{0fb9b1}
\definecolor{LightCyan}{HTML}{2bcbba}
\definecolor{DarkCyan}{HTML}{00a8a8}


\usetikzlibrary{calc}

\usepackage[many]{tcolorbox}

\newtcolorbox{boxx}[2][]{
    enhanced,
    fonttitle=\fontsize{11}{13.2}\selectfont\bfseries,
    coltitle=text,
    colback=infoBulleBackground,
    colbacktitle=infoBulleBackground,
    colframe=infoBulleBackground,
    toprule=4pt,titlerule=0pt,bottomrule=0pt,rightrule=0pt,leftrule=0pt,
    segmentation hidden,
    sharpish corners,
    overlay={\draw[line width=2.5pt,#1] (frame.north west)--(frame.south west);},
    % margins & padding
    before skip=\baselineskip,
    after skip=\baselineskip,
    boxsep=2pt,
    top=6pt,
    bottom=4pt,
    left=8pt,
    right=6pt,
    breakable,
    title=#2,
    extras middle and last={%
        %top=20pt,
        overlay={%
            \draw[line width=2.5pt,#1] (frame.north west)--(frame.south west);
            % \draw[line width=0.5pt,dashed,#1] (frame.north west)--(frame.north east);
            % \draw[] {([yshift=-12pt,xshift=4.5pt]frame.north west)} node[anchor=west] {#2};
            % \draw[] {([yshift=-12pt,xshift=-4.5pt]frame.north east)} node[anchor=east] {eg};
        }
    },
}

\definecolor{codegreen}{HTML}{369432}
\definecolor{codegray}{HTML}{3a3d41}
\definecolor{codepurple}{HTML}{c586c0}
\definecolor{backcolour}{HTML}{1e1e1e}
\definecolor{standartcolor}{HTML}{cccccc}
\definecolor{codeorange}{HTML}{f48771}

\setlength{\parindent}{0pt}
\definecolor{bg}{rgb}{0.13, 0.13, 0.13}
%\pagecolor{bg}
%\color{white}
\begin{document}
%---COLORS---
\definecolor{pRed}{RGB}{255, 55, 55}
\definecolor{pOrange}{RGB}{219, 132, 36}

%\pagestyle{fancy}
%\fancyhf{}
%\lhead{Moathekuchenaufgabe}
%\rhead{Lösung}
%\cfoot{\thepage}

\section{Differenzialrechnung}

\subsection{Wiederholung Klasse 10}
\begin{boxx}[Red]{Definition: Funktion}
    Eine eindeutige Zuordnung, die jedem $x$-Wert (aus dem Definitionsbereich) 
    einen $y$-Wert zuordnet, nennt man \textbf{Funktion}.
\end{boxx}

\begin{boxx}[Red]{Definition: Graph}
    Die Menge aller Punkte, die einer gemeinsamen Zuordnungsvorschrift folgen 
    (z.B. $\displaystyle y = x^2$), bilden einen \textbf{Graphen}.
\end{boxx}

\begin{boxx}[Red]{Definition: Differenzenquotient}
    Gegeben sind zwei Punkte $\displaystyle P\left(x_0\;|\;f\left(x_0\right)\right)$ und
    $\displaystyle Q\left(x_0 + h\;|\;f\left(x_0 + h\right)\right)$.
    \\\\
    Der Quotient $\displaystyle \frac{f\left(x_0 + h\right)- f\left(x_0\right)}{h}$
    heißt Differenzenquotient und beschreibt im Sachzusammenhang die 
    \textbf{mittlere Änderungsrate} in einem Intervall.
    \\\\
    Anschaulich entspricht der Differenzenquotient der \textbf{Steigung der Sekante}
    durch die beiden Punkte $P$ und $Q$.
\end{boxx}

\begin{boxx}[Red]{Definition: Ableitung}
    Existiert der Grenzwert des Differenzenquotienten für $h \to 0$, so nennt man
    diesen Wert die \textbf{Ableitung von} $\mathbf{f}$ \textbf{an der Stelle} $\mathbf{x_0}$.

    \[f'\left(x_0\right) = \lim_{h \to 0} \frac{f\left(x_0 + h\right)- f\left(x_0\right)}{h}\]

    Die Ableitung beschreibt im Sachzusammenhang die \textbf{momentane Änderungsrate}
    und entspricht anschaulich der \textbf{Steigung der Tangente am Graphen von} $\mathbf{f}$ 
    \textbf{an der Stelle} $\mathbf{x_0}$. 
    \\\\
    Existiert der Grenzwert $\forall x$ aus dem Intevall, so nennt man $f$ differenzierbar.
\end{boxx}

\begin{boxx}[DarkBlue]{Beispiel}
    $a)$\hspace{3mm} Zeiche den Graph von $f$ in mit $\displaystyle f(x) = \frac{1}{2}x^2$
    in eine geeignetes Koordinatensystem.
    \begin{figure}[H]
        \centering
        \begin{tikzpicture}
        \begin{axis}[
            height = 6cm,
            width = 8cm,
            axis lines = middle,
            xlabel = {\(x\)},
            ylabel = {\(y\)},
        ]
        \addplot[
            thick,
            color = LightPurple,
            samples = 100,
            smooth,
        ] {0.5*x^2} node[anchor = east, pos = 0.95] {\(G_f\)};
        \end{axis}
        \end{tikzpicture}
    \end{figure}

    $b)$\hspace{3mm} Bestimme die mittlere Änderungsrate von $f$ in $\left[0;3\right]$.
    \begin{align*}
        \frac{f(3)-f(1)}{3-1} &= \frac{4.5 - 0.5}{2} = 2
    \end{align*}

    $c)$\hspace{3mm} Bestimme die momentane Änderungsrate an der Stelle $x_0 = 2$ zeichnerisch.

    \begin{figure}[H]
        \centering
        \begin{tikzpicture}
        \begin{axis}[
            clip = false,
            height = 7cm,
            width = 8cm,
            axis lines = middle,
            xtick = {-2,-1,1,2,3,4,5,6},
            xmin = -2,
            xmax = 6,
            ymin = -2,
            ymax = 10,
            xlabel = {\(x\)},
            ylabel = {\(y\)},
            restrict y to domain = -2:10,
        ]
        \addplot[
            thick,
            color = LightPurple,
            samples = 100,
            domain = -2:6,
            smooth,
        ] {0.5*x^2} node[anchor = east, pos = 0.95] {\(G_f\)};
        \addplot[
            color = Black,
            domain = 0.5:5,
        ]{2*x - 2};
        \addplot[Black] coordinates {
            (2, 2) (3, 2) (3, 4)
        };
        \node at (axis cs:2.5,1.5) {\(1\)};
        \node at (axis cs:3.25,3) {\(2\)};
        \node at (axis cs:6,3) {\(\displaystyle \frac{\Delta y}{\Delta x} = \frac{2}{1} = 2\)};
        \end{axis}
        \end{tikzpicture}
    \end{figure}
\end{boxx}

\subsection{Verkettete Funktionen}

Gegeben seien zwei Funktionen $g$ und $h$.

Die Funktion $\displaystyle f = g \circ h$ (''$g$ nach $h$'') 
ist die \textbf{verkettete Funktion} mit
$(g \circ h)(x) = g\left(h(x)\right)$
\\\\
Dabei wird der Funktionsterm von $h(x)$ für die Variable in $g(x)$ eingesetzt.

\begin{boxx}[DarkBlue]{Beispiel}
    $a)$\hspace{3mm} $g(x) = 3x + 1;\; h(x)= 2x^2$. 
    Bestimme $(g \circ h)(x)$ und $(h \circ g)(x)$.

    \begin{align*}
        (g \circ h)(x) &= 6x^2+1 \\
        (h \circ g)(x) &= 2(3x+1)^2 = 18x^2 + 12x + 2
    \end{align*}
    
    $b)$\hspace{3mm} Bestimme je 2 Funktionen für $g$ und $h$, für die gilt
    $g \circ h = f$ mit $\displaystyle f(x) = \frac{2}{\left(x^2-1\right)^3}$

    \begin{align*}
        &&&\text{1.} & g(x) &= \frac{2}{x^3} &&&&\\
        &&&& h(x) &= x^2-1 &&&&\\\\
        &&&\text{2.} & g(x) &= \frac{2}{x} &&&&\\
        &&&& h(x) &= \left(x^2 - 1\right)^3 &&&&
    \end{align*}
\end{boxx}

\newpage

\subsection{Die Kettenregel}

Gegeben sind zwei differenzierbare Funktionen $u$ und $v$.
Gesucht ist die Ableitungsfunktion $f'$ mit $f = u \circ v$. Es gilt:
\[f'(x) = u'\left(v(x)\right) \cdot v'(x)\]

\begin{boxx}[Purple]{Beweis}
    \begin{align*}
        f'\left(x_1\right) &= \lim_{x_2 \to x_1} \frac{f\left(x_2\right)-f\left(x_1\right)}{x_2 - x_1} & &|\;\text{Differenzenquotient}\\
        &= \lim_{x_2 \to x_1} \frac{u\left(v\left(x_2\right)\right)- u\left(v\left(x_1\right)\right)}{x_2 - x_1}& &|\;\text{Einsetzen} \\
        &= \lim_{x_2 \to x_1} \frac{u\left(v\left(x_2\right)\right)- u\left(v\left(x_1\right)\right)}{v\left(x_2\right)-v\left(x_1\right)} \cdot \frac{v\left(x_2\right)-v\left(x_1\right)}{x_2 - x_1}& &|\;\text{Multiplikation einer 1} \\
        &= \lim_{x_2 \to x_1} \frac{u\left(v\left(x_2\right)\right)- u\left(v\left(x_1\right)\right)}{v\left(x_2\right)-v\left(x_1\right)} \cdot \lim_{x_2 \to x_1} \frac{v\left(x_2\right)-v\left(x_1\right)}{x_2 - x_1}& &|\;\text{Grenzwerte separat berechnen} \\
        &\;\;\;\;\;\;\;\text{da } v \text{ stetig ist } \Rightarrow  \lim_{x_2 \to x_1} v\left(x_2\right) = v\left(x_1\right) \\
        &= u'\left(v\left(x_1\right)\right) \cdot v'\left(x_1\right)
    \end{align*}
    \qed
\end{boxx}

Bezeichnet man $u(x)$ und $v(x)$ als äußere bzw. innere Funktion, dann lautet die Kettenregel salopp:

''Äußere Ableitung mal innere Ableitung''

\begin{boxx}[DarkBlue]{Beispiel}
    Leite ab!

    $a)$\hspace{3mm} $\displaystyle f(x) = (2x-1)^5$
    \begin{align*}
        f'(x) &= 10(2x-1)^4
    \end{align*}
    $b)$\hspace{3mm} $\displaystyle g(x) = \frac{1}{\frac{1}{2}x^2-2x}$
    \begin{align*}
        g'(x) &= -\frac{x-2}{\left(\frac{1}{2}x^2-2x\right)^2}
    \end{align*}
    $c)$\hspace{3mm} $\displaystyle h(x) = 2\cos\left(\frac{1}{3}x^2+\frac{1}{2}x\right)$
    \begin{align*}
        h'(x) &= -\left(\frac{4}{3}x + 1\right)\sin\left(\frac{1}{3}x^2+\frac{1}{2}x\right)
    \end{align*}
\end{boxx}

\newpage

\subsection{Die Produktregel}

Gegeben sind die Funktionen $f$ und $g$ (differenzierbar). 
Für $h(x) = f(x) \cdot g(x)$ gilt für Ableitungsfunktion

\[h'(x) = f'(x) \cdot g(x) + f(x) \cdot g'(x)\]

Salopp: ''abgeleitet hingeschrieben plus hingeschrieben abgeleitet''

\begin{boxx}[DarkBlue]{Beispiel}
    Leite ab!

    $a)$\hspace{3mm} $\displaystyle f(x) = x^2 \cdot \cos x$

    \begin{align*}
        f'(x) &= -x^2\sin(x) + 2x\cos(x)
    \end{align*}

    $b)$\hspace{3mm} $\displaystyle g(x) = \sqrt{x} \cdot \left(3x-5\right)^4$

    \begin{align*}
        g'(x) &= \frac{\left(3x-5\right)^4}{2\sqrt{x}} + 12\sqrt{x}\left(3x-5\right)^3
    \end{align*}
\end{boxx}

\subsection{Monotonie und Krümmungsverhalten}
\begin{boxx}[Red]{Definition}
    Eine Funktion $f$ sei definiert auf einem Intervall $I$.
    \\\\
    $f$ heißt \textbf{streng monoton wachsend} wenn:
    \[\forall x_1;x_2 \in I: \; x_1 < x_2 \Rightarrow f\left(x_1\right)<f\left(x_2\right)\]

    $f$ heißt \textbf{streng monoton fallend} wenn: \\
    \[\forall x_1;x_2 \in I: \; x_1 < x_2 \Rightarrow f\left(x_1\right)>f\left(x_2\right)\]
\end{boxx}

\begin{boxx}[LightGreen]{Monotoniesatz}
    Die Funktion $f$ sei differenzierbar.
    \\\\
    Wenn $f'(x)>0 \; \forall \; x \in I$ dann ist $f$ in $I$ streng monoton wachsend.

    Entsprechend gilt für $f'(x)<0$, dass $f$ streng monoton fallend ist.
    \\\\
    \textbf{Anmerkung:} Die Umkehrung des Satzes gilt im Allgemeinen nicht.
    (siehe $\displaystyle f(x) = x^2$)
\end{boxx}

\begin{boxx}[Red]{Definition: Links-/Rechtskurve}
    Sei $f$ eine auf dem Intevall $I$ zweimal differenzierbare Funktion.

    \begin{itemize}
        \item $f''(x)>0$ in $I$ $\rightarrow$ Linkskurve, $f'$ ist streng monoton wachsend
        \item $f''(x)<0$ in $I$ $\rightarrow$ Rechtskurve, $f'$ ist streng monoton fallend
    \end{itemize}
\end{boxx}
\newpage
\begin{boxx}[DarkBlue]{Beispiel}
    $\displaystyle f(x) = -2x^3 - 6x^2 + 18x$. 
    Untersuche $f$ auf ihr Krümmungsverhalten.
        \[f'(x) = -6x^2-12x+18;\; f''(x) = -12x-12\]
    \begin{align*}
        f''(x) &= 0 \\
        \Rightarrow x_1 &= -1\\
        f''(0) = -12 < 0 &\Rightarrow \text{Rechtskurve} \\
        f''(-2) = 12 > 0 &\Rightarrow \text{Linkskurve} \\
    \end{align*}
    In $\displaystyle (-\infty; -1]$ beschreibt $f$ eine Linkskurve.

    In $\displaystyle [-1; \infty)$ beschreibt $f$ eine Rechtskurve.
\end{boxx}

\subsection{Extremstellen}
Geben ist eine zweimal differenzierbare Funktion $f$. 
Die \textbf{notwendige Bedingung} für Extremstellen in $f$ lautet:

\[f'(x)=0\]

\begin{boxx}[LightYellow]{Notwendigkeit}
    Notwendigkeit bedeutet:

    Liegt in $f$ eine Extremstelle vor, so ist $f'(x) = 0$.
    Die Umkehrung gilt im Allgemeinen \textbf{nicht}.

    \begin{align*}
        \text{Extremstelle} &\Rightarrow f'(x) = 0 \\
        f'(x) = 0 &\not\Rightarrow \text{Extremstelle} \\
    \end{align*}
\end{boxx}
Gilt zusätzlich, dass $f'$ an der Extremstelle einen \textbf{Vorzeichenwechsel (VZW)} aufweist,
so weiß man beim Übergang
\begin{itemize}
    \item von $+$ nach $-$ liegt eine \textbf{Maximumstelle} vor.
    \item von $-$ nach $+$ liegt eine \textbf{Minimumstelle} vor.
\end{itemize}
Gilt
\begin{align*}
    &&&& f'\left(x_e\right) &= 0 &&\text{und} & f''\left(x_e\right) &\not = 0 &&&&
\end{align*}
so liegt in $f$ am der Stelle $x_e$ eine Extremstelle vor.

\begin{align*}
    f''\left(x_e\right)<0 &\Rightarrow \text{Maximumstelle} \\
    f''\left(x_e\right)>0 &\Rightarrow \text{Minimumstelle}
\end{align*}

\newpage
\begin{boxx}[DarkBlue]{Beispiel}
    $\displaystyle f(x) = x^3 + 3x^2 - 9x$. Untersuche $f$ auf Extremstellen.
    \[f'(x) = 3x^2+6x-9;\; f''(x)=6x+6\]
    \begin{align*}
        f'(x) &= 0 \\
        3x^2 + 6x - 9 &= & &|\; \cdot \frac{1}{3} \\
        x^2 -2x -3 &= 0 & &|\; \text{Satz von Vieta} \\
        (x-3)(x+1) &= 0 \\
        \Rightarrow x_1= 3;\; x_2 &= -1 \\\\
        f''(3) = 12 > 0 &\rightarrow \text{Minimumstelle} \\
        f''(-1) = -12 < 0 &\rightarrow \text{Maximumstelle} \\
    \end{align*}
\end{boxx}

\subsection{Wendestellen}
Sei $f$ eine mindestens 3-mal differenzierbare Funktion. 
Ist $x_w$ eine Wendestelle von $f$, so gilt
\begin{align*}
    &&&& f''\left(x_w\right) &= 0 & &\text{(notwendige Bedingung)} &&\\\\
    &&&\text{sowie} & f'''\left(x_w\right) &\not = 0
\end{align*}
Ebenfalls gilt, dass $f''$ um $x_w$ einen VZW besitzt, 
bzw. in $f'$ eine Extremstelle mit VZW vorliegt. (notwendig und hinreichend)
\begin{boxx}[DarkBlue]{Beispiel}
    Berechne die Wendestellen von $f$ mit $\displaystyle f(x) = x^3\left(2 + x\right)$.
    \begin{align*}
        f'(x) &= 3x^2(2+x) + x^3 \\
        f''(x) &= 6x(2+x)+3x^2 + 3x^2 \\
        &= 12x^2 + 12x \\
        f'''(x) &= 24x + 12\\\\
        f''(x) &= 0 \\
        12x^2 + 12x &= 0 \\
        12x(x+1)&= 0 \\
        \Rightarrow x_1 &= 0;\; x_2 = -1 \\\\
        f'''\left(x_1\right) &= 12 \not = 0 \\
        f'''\left(x_2\right) &= -12 \not = 0 \\
    \end{align*}

    $f$ hat Wendestellen bei $x_1 = 0$ und $x_2 = -1$.
\end{boxx}

\newpage
\subsection{Tangente und Normale von Außen}
Gegeben ist eine Funktion $f$ und sowie ein Punkt $P$, der nicht auf $f$ liegt.

Bestimme die Gleichung(en) der Tangente(n) durch $P$ an $f$.

\begin{boxx}[LightGreen]{Allgemeine Tangentengleichung}
    \begin{align*}
        \frac{\Delta y}{\Delta x} = \frac{t(x)-f(u)}{x-u} &= f'(x) & &|\;\cdot (x-u) \\
        t(x) - f(u) &= f'(u) \cdot (x-u) & & |\; + f(u) \\
        t(x) &= f'(u)\cdot(x-u)+f(u)
    \end{align*}
\end{boxx}

\begin{figure}[H]
    \centering
    \begin{tikzpicture}
    \begin{axis}[
        height = 10cm,
        width = 13cm,
        axis lines = middle,
        ymin = -4,
        axis equal,
        xlabel = {\(x\)},
        ylabel = {\(y\)},
    ]
    \addplot[
        line width = 1pt,
        color = DarkBlue,
        samples = 100,
        domain = -3:3
    ]{x^2} node[anchor= east, pos = 0.98] {\(f(x)\)};
    \addplot[Green, domain = -1:4]{2*2^(0.5)*x - 2} node[anchor= west, pos = 0.98] {\(t_1(x)\)};
    \addplot[Green, domain = -4:1]{-2*2^(0.5)*x - 2} node[anchor= east, pos = 0.03] {\(t_2(x)\)};
    \addplot[] coordinates {(2.47, 5) (3.54,5)} node[anchor = north, pos = 0.5] {\(\Delta x\)};
    \addplot[] coordinates {(3.54, 5) (3.54,8)} node[anchor = west, pos = 0.5] {\(\Delta y\)};
    \addplot[Red, domain = -6:6]{(-1/(2*2^(0.5)))*x + 2.5} node[anchor= south west, pos = 0.98] {\(n_1(x)\)};
    \addplot[Red, domain = -6:6]{(1/(2*2^(0.5)))*x + 2.5} node[anchor= south east, pos = 0.02] {\(n_2(x)\)};
    \node[label={10:{\(Q(u,f(u))\)}},circle,fill,inner sep=1.5pt] at (axis cs:1.41,2) {};
    \node[label={0:{\(P(0,-2)\)}},circle,fill,inner sep=1.5pt] at (axis cs:0,-2) {};
    \end{axis}
    \end{tikzpicture}
    
\end{figure}


\begin{enumerate}
    \item Terme von $f(u)$ bzw. $f'(u)$ bestimmen.
    \begin{align*}
        f(u)=u^2;\;f'(u)=2u
    \end{align*}
    \item $P$ in $t(x)$ einsetzen.
    \begin{align*}
        -2 &= 2u\cdot (0-u)+u^2 \\
        -2 &= -2u^2 +u^2 \\
        -2 &= -u^2 \\
        u &= \pm \sqrt{2}
    \end{align*}
    \item $u_1$ / $u_2$ in $t(x)$ einsetzen.
    \begin{align*}
        &u_1: & t_1(x) &= 2\sqrt{2} \cdot (x-\sqrt{2}) + \left(\sqrt{u}\right)^2  &&\\
        & & &= 2\sqrt{2} x -2 &&\\
        & u_2: & t_2(x) &= -2\sqrt{2} x -2 &&
    \end{align*}
\end{enumerate}
\begin{boxx}[Red]{Normale}
    Die Gerade, die die Tangente orthogonal schneidet heißt \textbf{Normale}.

    Für die Steigung der Normalen gilt:
    \begin{align*}
        m_t \cdot m_n &= -1  & m_n &= -\frac{1}{m_t}
    \end{align*}
    \[n(x) = -\frac{1}{f'(u)}\cdot (x-u)+f(u)\]
\end{boxx}
\subsection{Extremwertaufgaben mit Nebenbedingungen (Minimax)}

Beim Lösen von Minimax-Aufgaben ist folgendes Verfahren empfehlenswert:
\begin{enumerate}
    \item Aufgabe abchecken, ggf. Skizze machen
    \item \textbf{Hauptbedingung} formulieren. Diese hängt von zwei Variablen ab.
    \item In der Aufgabe wird die zu maximierende bzw. minimierende Größe 
    aus der Hauptbedingung durch irgendeine zusätzliche Größe begrenzt.
    Diese wird durch die sogenannte \textbf{Nebenbedingung} beschrieben.
    Die Nebenbedingung hängt auch von zwei Variablen ab.
    \item Jetzt stellt man die Nebenbedingung nach einer der Variablen um
    und setzt diese in die Hauptbedingung ein.
    Man erhält eine Gleichung, die nur noch von einer Variable abhängt und
    \textbf{Zielfunktion} genannt wird.
    \item Zielfunktion $Z(x)$ ableiten und notwendige und hinreichende Bedingungen
    für Extremstellen aufstellen:
    \begin{align*}
        Z'(x) &= 0 \\
        Z''(x) &\lessgtr 0
    \end{align*}
    \item Theoretisch Ränder überprüfen.
    \item Die Wert für die Variablen berechnen und einen Antwortsatz formulieren.
\end{enumerate}

\begin{boxx}[DarkBlue]{Beispiel}
    Die Summe zweier natürlicher Zahlen beträgt 21.
    Berechne das größtmögliche Produkt der beiden Zahlen.

    \begin{align*}
        P(a,b) &= a\cdot b \\
        a + b = 21 &\Leftrightarrow a = 21 - b \\
        P(b) &= b(21-b) \\
        &= -b^2 + 21b \\\\
        P'(b) = -2b &+ 21;\;P''(b) = -2 \\\\
        P'(b) &= 0 \\
        -2b + 21 &= 0 \\
        b &= 10.5 \Rightarrow 10 \\
        a &= 21-10 = 11\\\\
        P''(b) &< 0 \Rightarrow \text{Maximumstelle}\\\\
        \Rightarrow a\cdot b &= 110
    \end{align*}
    Das größtmögliche Produkt beträgt 110 und ist Produkt der Zahlen 11 und 10.
\end{boxx}

\newpage

\section{Exponentialfunktionen}

Bei einer beliebigen Exponentialfunktion $f$ mit $f(x) = a^x;\; a \in \mathbb{R}$ gilt,
dass die Zunahme des ''Bestands'' proportional zum Bestand ist.

\[f'(x) \sim f(x)\]

\subsection{Die natürliche Exponentialfunktion zur Basis \emph{e}}

Es gibt eine Zahl, für die die Ableitung der Exponentialfunktion exakt gleich wie
die Ausgangsfunktion ist.
\\\\
Für diese \textbf{Eulersche Zahl \emph{e}} $\mathbf{\approx 2.7182}$ gilt also
\[f(x) = e^x = f'(x) = e^x\] 

\begin{boxx}[DarkBlue]{Beispiel}
    Leite ab! \\
    $a)$\hspace{3mm} $f(x) = \sin x \cdot e^x$
    \begin{align*}
        f'(x) &= e^x\cos x + e^x\sin x = e^x(\cos x + \sin x)
    \end{align*}

    $b)$\hspace{3mm} $g(x) = 4e^{3x-1}$
    \begin{align*}
        f'(x) &= 12e^{3x-1}
    \end{align*}
\end{boxx}

\subsection{Exponentialgleichungen mit \emph{e}}

Gegeben ist eine Zahl $b > 0$. Die Lösung der Gleichung
\[e^x = b\]
heißt \textbf{natürlicher Logarithmus} von $b$.

\begin{boxx}[LightGreen]{Merke}
    \[e^{\ln k} = k\]

    Die natürliche Exponentialfunktion $e^x$ hat dementsprechend 
    die Logarithmusfunktion $f(x) = \ln x$ als \textbf{Umkehrfunktion}.
\end{boxx}

\begin{boxx}[DarkBlue]{Beispiel}
    \begin{align*}
        e^x + \frac{7}{e^x} &= 8 & &|\; -8 \\
        e^x + \frac{7}{e^x} -8 &= 0 & &|\; \cdot e^x \\
        e^{2x} + 7 - 8e^x &= 0 & &|\; u = e^x \\
        u^2 - 8u + 7 &= 0 \\
        (u-1)(u-7) &= 0 \\
        u_1 = 1;\;&u_2 = 7 \\
        \Rightarrow x_1 = 0;\;&x_2 = \ln 7
    \end{align*}
\end{boxx}

\newpage

\subsection{Kurvendiskussion bei \emph{e}-Funktionen}
Zur Erinnerung:
\begin{enumerate}
    \item \textbf{Nullstellen bestimmen:}
    \begin{align*}
        f(x_n) &= 0 & e^0 =1 \\
        & & \ln 0 \text{ ist nicht definiert.} 
    \end{align*}
    \item \textbf{Extremstellen bestimmen:} 
    \begin{align*}
        f'(x_e) &= 0\\
        f''(x_e) &\lessgtr 0
    \end{align*}
    \item \textbf{Wendepunkte bestimmen:}
    \begin{align*}
        f''(x_w) &= 0\\
        f'''(x_w) &\not= 0
    \end{align*}
\end{enumerate}

\begin{boxx}[Red]{Verhalten für $x \to \pm \infty$ und Asymptoten bestimmen}
    Eine waagerechte \textbf{Asymptote} ist eine Gerade der Form $y=a$,
    wobei $a$ ein endlicher Wert des Grenzwerts von $f$ für $x \to \pm \infty$ ist.
    
    Der Graph von $f$ nähert sich dann infinitesimal an die Asymptote an.
\end{boxx}

\textbf{btw: Symmetrie}
\begin{itemize}
    \item achsensymmetrisch zur $y$-Achse: $f(x)=f(-x)$
    \item punktsymmetrisch zum Ursprung: $f(x)=-f(-x)$
\end{itemize}

\begin{boxx}[DarkBlue]{Beispiel}
    Skizziere den Graph von $f$ mit $f(x)= x^5\cdot e^{-0.1x}$
    \begin{figure}[H]
        \centering
        \begin{tikzpicture}
        \begin{axis}[
            axis lines = middle,
            xlabel = {\(x\)},
            ylabel = {\(f(x)\)},
            xmax = 80,
            xmin = -30,
            ymax = 2400000,
            restrict y to domain = -2200000:2400000
        ]
        \addplot [
            smooth,
            domain = -30:80, 
            samples = 100, 
            color = LightPurple,
            thick
        ]
        {x^5*e^(-0.1*x)};
        \end{axis}
        \end{tikzpicture}
        \end{figure}
\end{boxx}

\newpage

\subsection{Funktionenscharen von \emph{e}-Funktionen}
\begin{boxx}[Red]{Definition: Funktionenschar}
    Eine Funktion $f_t$ mit einem \textbf{Parameter} $t$,
    ordnet jedem $x$ einen Funktionswert $f_t(x)$ zu.
    \\\\
    Dabei ist ein Parameter ein beliebiger Wert, der wenn er einmal festgelegt wurde,
    seinen Wert beibehält.
    \\\\
    Die Graphen von $f_t$ bilden eine sogenannte \textbf{Funktionenschar} von (Exponential-)Funktionen.
\end{boxx}


Die Graphen von $f_t$ können verschoben und/oder gestreckt werden:

\begin{align*}
    f_t(x)&=e^{x+t} & &\text{ist gegenüber } e^x \text{ um } t \text{ Einheiten} \\
    & & &\text{nach links verschoben für } t>0 \\
    & & &\text{und nach rechts für } t<0 \\\\
    f_k(x)&=ke^{x} & &\text{ist in } y \text{-Richtung gestaucht  bzw. gestreckt}\\\\
    f_b(x)&=e^{x}+b & &\text{ist in } y \text{-Richtung nach oben bzw.}\\
    & & &\text{nach unten verschoben} \\\\
    f_w(x)&=e^{wx} & &\text{ist in } x \text{-Richtung gestreckt bzw. gestaucht.}\\
\end{align*}

\begin{boxx}[DarkBlue]{Beispiel}
    Gegeben ist $f_t(x)=e^{x+t}-1$. \\\\
    $a)$\hspace{3mm} Bestimme $f_{-2}(3)$.
    \begin{align*}
        f_{-2}(3) &= e^{3+(-2)} -1 = e - 1 \approx 1.718
    \end{align*}
    $b)$\hspace{3mm} Beschreibe die Wirkung des Parameters $t$ auf $f_t$.\\\\
    Der Parameter $t$ verschriebt den Graphen in $x$-Richtung. 
    Bei Erhöhung verschiebt sich der Graph nach links.
    \\\\
    $c)$\hspace{3mm} Gib die Koordinaten der Schnittpunkte mit den Koordinatenachsen an.
    \begin{align*}
        f_t(0) &= e^{0+t} -1 = e^t - 1 \\
        &\Rightarrow S_t\left(0 \;|\; e^t-1\right)
    \end{align*}
    \begin{align*}
        && e^{x+t}-1 &= 0 & &|\; +1 && \\
        && e^{x+t} &= 1 & &|\; \ln && \\
        && x + t &= \ln 1 && \\
        && x + t &= 0 && \\
        && x &= -t && \\
        && &\Rightarrow N_t\left(-t \;|\;0\right)
    \end{align*}
\end{boxx}

\newpage
\subsection{Die Umkehrfunktion}
\begin{boxx}[Red]{Definition}
    Sei $f$ eine Funktion mit Definitionsmenge $D_f$ und Wertemenge $W_f$.

    $f$ heißt \textbf{umkehrbar}, wenn zu jedem $y \in W_f$ genau ein
    $x \in D_f$ mit $f(x) = y$ existiert.
    \\\\
    Bei einer umkehrbaren Funktion $f$ heißt die Funktion $\bar{f}$ \emph{(f quer)} mit
    $\bar{f}(y)=x$ die \textbf{Umkehrfunktion} von $f$.
    \\\\
    Es gilt
    \begin{align*}
        &&&& \bar{f}\left(f(x)\right) &= x & & \forall x \in D_f &&\text{und} && \\
        &&&& f\left(\bar{f}(x)\right) &= x & & \forall x \in D_{\bar{f}} = W_f &&&&
    \end{align*}
    \[\left(\text{z.B.}\;\;\; \sin^{-1}(\sin x)= x \;\;\;\text{bzw.}\;\;\;\sin^{-1}(\sin 30\degree)= 30\degree\right)\]
\end{boxx}

Ist $f$ streng monoton wachsend, so ist sie umkehrbar und es existiert eine Umkehrfunktion $\bar{f}$ von $f$.

Also überprüft man ob $f'(x) > 0$ gilt. 
\\\\
Gleiches gilt auch wenn $f$ durchgehend streng monoton fallend ist.
\\\\
\textbf{Berechnung von } $\mathbf{\bar{f}:}$

\begin{enumerate}
    \item $y = f(x)$ setzen
    \item Gleichung nach $x$ auflösen
    \item $x$ und $y$ vertauschen
    \item $y$ durch $\bar{f}(x)$ ersetzen
\end{enumerate}

\begin{boxx}[DarkBlue]{Beispiel}
    Gegeben ist $f$ mit $f(x)= \sqrt{2x-4} - 1$.\\
    $a)$\hspace{3mm} Gib $D_f$ und $W_f$ an.
    \begin{align*}
        D_f &= [2; \infty) \\
        W_f &= [-1; \infty) \\
    \end{align*}
    $b)$\hspace{3mm} Zeige, dass $f$ umkehrbar ist.
    \begin{align*}
        && f'(x) &= \frac{1}{\sqrt{2x-4}} \\
        && f'(x) &= 0 \\
        && \frac{1}{\sqrt{2x-4}} &= 0 & &\Rightarrow \text{keine Lösung} &&
    \end{align*}
    $c)$\hspace{3mm} Bestimme $\bar{f}$.
    \begin{align*}
        y &= \sqrt{2x-4} -1 & &|\;+1 \\
        y+1 &= \sqrt{2x-4}  & &|\;()^2 \\
        (y+1)^2 &= 2x-4  & &|\;+4 \\
        (y+1)^2 +4 &= 2x  & &|\;\cdot \frac{1}{2} \\
        x&=\frac{(y+1)^2 + 4}{2} \\
        y &=\frac{(x+1)^2}{2} +2 \\\\
        \Rightarrow \bar{f}(x) &= \frac{(x+1)^2}{2} +2
        \end{align*}
\end{boxx}

\subsection{Der natürliche Logarithmus und seine Ableitungsfunktion}

Von der natürlichen Exponentialfunktion $f(x) = e^x$ ist der natürliche Logarithmus $\bar{f}(x)=\ln x$
die Umkehrfunktion.

\begin{figure}[H]
    \centering
    \begin{tikzpicture}
        \begin{axis}[
            height = 6cm,
            width = 10cm,
            axis lines = middle,
            xlabel = {\(x\)},
            ylabel = {\(y\)},
            ymax = 2,
            ymin = -2,
        ]
            \addplot[
                color = Blue,
                thick,
                domain = 0:7,
                samples = 100,
            ] {ln(x)} node[anchor = north, pos = 0.95] {\(\ln x\)};
            % \node[anchor = north, color=Blue] at (rel axis cs:0.75,0.75) {\(\ln x\)};
            \node[label={150:{\((1,0)\)}},circle,fill,inner sep=1.5pt] at (axis cs:1,0) {};
            \node[label={150:{\((e,1)\)}},circle,fill,inner sep=1.5pt] at (axis cs:2.718,1) {};
        \end{axis}
    \end{tikzpicture}
\end{figure}
\begin{align*}
    D_{\bar{f}} &= (0;\infty) \\
    W_{\bar{f}} &= (-\infty;\infty) \\
\end{align*}

Für die Ableitung von $\ln x$ gilt:
\begin{align*}
    && x &= e^{\ln x} & &|\; ()' &&\\
    && 1 &= \left(e^{\ln x}\right)' &&\\
    && 1 &= \left(\ln x\right)' \cdot e^{\ln x} & &|\; e^{\ln x} = x &&\\
    && 1 &= \left(\ln x\right)' \cdot x & &|\; \frac{1}{x}\;\;(x \not= 0) &&\\
    && \left(\ln x\right)' &= \frac{1}{x}
\end{align*}
\newpage

\begin{boxx}[DarkBlue]{Beispiel}
    $a)$\hspace{3mm} $f(x) = x \ln(3x)$. Leite ab!
    \begin{align*}
        f'(x) &= \ln(3x) + \frac{x}{x} = \ln(3x) + 1
    \end{align*}
    $b)$\hspace{3mm} $\displaystyle g(x) = \frac{1}{2} \ln(3x-6)$. Bestimme $\bar{g}, D_g, D_{\bar{g}}, W_g$ und $W_{\bar{g}}$.
    \begin{align*}
        \bar{g}(x) &= \frac{1}{3} e^{2x} + 2 \\
        D_g = (2;\infty) &= W_{\bar{g}} = (2;\infty) \\
        D_{\bar{g}} = (-\infty;\infty) &= W_{g} = (-\infty;\infty) \\
    \end{align*}
\end{boxx}

\subsection{Exponentielles Wachstum}
Aus Klasse 9:
\begin{align*}
    f(t) &= f(0) \cdot a^t
\end{align*}
\begin{center}
    Für $a>1 \rightarrow$ exponentielle Zunahme \\
    Für $a<1 \rightarrow$ exponentieller Zerfall \\
\end{center}
\vspace{3mm}
Um die Exponentialfunktion ableiten zu können, ist es sinnvoll
sie zur Basis $e$ zu schreiben.
\[\Rightarrow f(t)=f(0)\cdot e^{\ln (a) \cdot t}\]
\begin{align*}
    &&&& \ln(a)&=k & &\text{Für }k>0 \rightarrow \text{Zunahme}&&&& \\
    &&&& & & &\text{Für }k<0 \rightarrow \text{Zerfall}&&&& \\
\end{align*}

Für die Verdopplungszeit $T_V$ gilt:
\[T_V = \frac{\ln 2}{k}\]

Für die Halbwertszeit $T_H$ gilt:
\[T_H = \frac{\ln 0.5}{k}\]

\begin{boxx}[DarkBlue]{Beispiel}
    Gegeben ist $f(t) = 228 \cdot 1.04^t$  $t$ in $y$. \\
    $a)$\hspace{3mm} Bestimme den Anfangsbestand.
    \begin{align*}
        f(0) &= 228
    \end{align*}

    $b)$\hspace{3mm} Handelt es sich um eine Zu- oder Abnahme? Begründe.

    Es handelt sich um eine Zunahme, da $1.04 > 1$ \\\\

    $c)$\hspace{3mm} Schreibe $f$ zur Basis $e$.
    \begin{align*}
        f(t) &= 228 \cdot e^{\ln(1.04)\cdot t}
    \end{align*}

    $d)$\hspace{3mm} Berechne $T_V/T_H$.
    \begin{align*}
        T_V&= \frac{\ln 2}{\ln 1.04} \approx 17.67 & T_H&= \frac{\ln 0.5}{\ln 1.04} \approx -17.77
    \end{align*}
\end{boxx}

\newpage

\section{Integralrechnung}
\subsection{Bestimmen der Gesamtänderung - orientierter Flächeninhalt}

Um von einer Größe die momentane Änderung zu berechnen, muss man ableiten.
Will man umgekehrt von der momentanen Änderung einer Größe auf die Größe selbst schließen, 
muss man den \textbf{orientierten Flächeninhalt} zwischen dem Graph der Änderungsrate und der $x$-Achse bestimmen.
\\
\begin{boxx}[LightYellow]{Anmerkung}
    Ein Flächeninhalt ist stets positiv, ein orientierter Flächeninhalt kann negativ sein.
\end{boxx}

\begin{boxx}[DarkBlue]{Beispiel}
    Ein Tank ist anfangs leer. 
    Über eine Leitung kann ihm Wasser hinzugefügt oder entfernt werden.
    Bestimme aus der folgenden Zufluss-/Abflussmenge den Wasserinhalt nach 12 Sekunden.

    \begin{align*}
        V &= \left(4\cdot3+\frac{1}{2}\cdot2\cdot 4-\frac{1}{2}\cdot 1 \cdot 2 - 4 \cdot 2 + 2\cdot 3\right) \unit{\liter} = \qty{13}{\liter}
    \end{align*}
\end{boxx}

\subsection{Das Integral als orientierter Flächeninhalt}
Um den Flächeninhalt von krummlinigen Graphen zu bestimmen, 
kann man die \textbf{Ober-} bzw. \textbf{Untersummen} zwischen Graph und $x$-Achse betrachten.
\\\\
\textbf{Anschaulich:} $\displaystyle f(x) = \frac{1}{2} x^2 +1;\; I=[0;5];\; n=5$
\begin{table}[H]
    \begin{tabular}{c|ccccc}
        $\displaystyle x$ & 0 & 1 & 2 & 3 & 4 \\ \hline
        $\displaystyle f(x)$ & 1 & 1.5 & 3 & 5.5 & 9 \\
    \end{tabular}
\end{table}

\begin{figure}[H]
    \centering
    \begin{subfigure}{.5\textwidth}
        \centering
        \begin{tikzpicture}
            \begin{axis}[
                width = 7cm,
                height = 6cm,
                axis lines = middle,
                xlabel = \(x\),
                ylabel = \(f(x)\),
                xmax = 5.5,
                xmin = 0,
                ymin = 0,
            ]
                \draw[fill = Red, fill opacity = 0.3] (0,0) rectangle (1,1.5);
                \draw[fill = Red, fill opacity = 0.3] (1,0) rectangle (2,3);
                \draw[fill = Red, fill opacity = 0.3] (2,0) rectangle (3,5.5);
                \draw[fill = Red, fill opacity = 0.3] (3,0) rectangle (4,9);
                \draw[fill = Red, fill opacity = 0.3] (4,0) rectangle (5,13.5);
                \addplot[color=Black, line width = 1, samples = 100, domain = 0:5.5] {0.5*x^2+1};
            \end{axis}
        \end{tikzpicture}
        \caption{Obersumme}
    \end{subfigure}%
    \begin{subfigure}{.5\textwidth}
        \centering
        \begin{tikzpicture}
            \begin{axis}[
                width = 7cm,
                height = 6cm,
                axis lines = middle,
                xlabel = \(x\),
                ylabel = \(f(x)\),
                xmax = 5.5,
                xmin = 0,
                ymin = 0,
            ]
                \draw[fill = Blue, fill opacity = 0.3] (0,0) rectangle (1,1);
                \draw[fill = Blue, fill opacity = 0.3] (1,0) rectangle (2,1.5);
                \draw[fill = Blue, fill opacity = 0.3] (2,0) rectangle (3,3);
                \draw[fill = Blue, fill opacity = 0.3] (3,0) rectangle (4,5.5);
                \draw[fill = Blue, fill opacity = 0.3] (4,0) rectangle (5,9);
                \addplot[color=Black, line width = 1, samples = 100, domain = 0:5.5] {0.5*x^2+1};
            \end{axis}
        \end{tikzpicture}
        \caption{Untersumme}
    \end{subfigure}
\end{figure}

Haben für $n \to \infty$ die Ober- und Untersumme den gleichen Wert,
so nennt man $f$ \textbf{integrierbar}.
\newpage
\begin{boxx}[Red]{Definition: Integral}
    Ist eine Funktion $f$ über $[a;b]$ integrierbar,
    so nennt man den orientierten Flächeninhalt über $[a;b]$,
    den der Graph von $f$ mit der $x$-Achse einschließt,
    das (bestimmte) \textbf{Integral} von $f$ über $[a;b]$.

    Dabei nennt man $a$ die untere und $b$ die obere Grenze des Integrals.

    \begin{align*}
        \int_a^b f(x) \, dx
    \end{align*}

    Dabei nennt man $f(x)$ den \textbf{Integrand} und $dx$ die \textbf{Integrationsvariable}.
\end{boxx}

\begin{boxx}[LightGreen]{Integraladditivität}

    \[\int_a^b f(x) \, dx + \int_b^c f(x) \, dx = \int_a^c f(x) \, dx\]

    \[\int_a^b f(x) \, dx = - \int_b^a f(x) \, dx\]

    \[\int_a^a f(x) \, dx = 0\]

\end{boxx}

\begin{boxx}[DarkBlue]{Beispiel}
    Bestimme näherungsweise $\displaystyle \int_{-1}^2 x^3 \, dx$

    \begin{align*}
        \int_{-1}^2 x^3 \, dx &= \left[\frac{1}{4}x^4\right]_{-1}^2
        = \left(\frac{1}{4}\cdot 2^4\right) - \left(\frac{1}{4} \cdot (-1)^4\right)
        = 3.75
    \end{align*}
\end{boxx}


\subsection{Der Hauptsatz der Differential- und Integralrechnung}

\begin{boxx}[Red]{Definition: Stammfunktion}
    Eine Funktion $F$ heißt Stammfunktion von $f$ im Intervall $I$, wenn gilt:

    \begin{align*}
        &&&&&& F'(x) &= f(x) & &\forall x \in I &&&&&&
    \end{align*}
\end{boxx}

\begin{boxx}[Red]{Hauptsatz der Differential- und Integralrechnung}
    Sei $f$ integrierbar auf $[a;b]$ und $F$ eine Stammfunktion von $f$, so gilt:

    \[\int_a^b f(x) \, dx = F(b) - F(a)\]

    Man schreibt:

    \[\int_a^b f(x) \, dx = \left[F(x)\right]_a^b = F(b) - F(a)\]
\end{boxx}

\begin{boxx}[LightGreen]{Satz: Stammfunktion}
    Von der Funktion $f$ existieren undendlich viele Stammfunktionen,
    die sich um eine Konstante $c$ unterscheiden. Es gilt:

    \begin{align*}
        &&&&&& F(x) &= G(x) + c & &\forall x \in I &&&&&&
    \end{align*}

    \textbf{Beweis:}

    \[F'(x)=f(x)=\left(G(x)+c\right)' = G'(x)\]
\end{boxx}

\begin{boxx}[DarkBlue]{Beispiel}
    $a)$\hspace{3mm} Bestimme zwei Stammfunktionen von $\displaystyle f(x) = 3x^2$

    \begin{align*}
        F_1(x) &= x^3 \\
        F_2(x) &= x^3 + \pi
    \end{align*}
    
    $b)$\hspace{3mm} Berechne $\displaystyle \int_1^3 3x^2 \, dx$.

    \begin{align*}
        \int_1^3 3x^2 \, dx &= \left[x^3\right]_1^3 = 3^3 - 1^3 = 26
    \end{align*}
\end{boxx}

\begin{boxx}[Red]{Integralfunktion}
    Sei $f$ eine über $I$ integrierbare Funktion und $u \in I$.
    Die Funktion $J_u$ mit $\displaystyle J_u(x) = \int_u^x f(t) \, dt$ 
    heißt die \textbf{Integralfunktion} von $f$ zur unteren Grenze $u$.
    \\\\
    Beachte:
    \begin{itemize}
        \item Man sollte für die Integrationsvariable und die obere Grenze
        nicht den gleichen Buchstaben wählen.
        \item Es gilt: $\displaystyle J_u(u) = \int_u^u f(t) \, dt = 0$,
        d.h. die untere Grenze ist stets eine Nullstelle der Integralfunktion.
    \end{itemize}
    
    \textbf{Satz:}
    Die Integralfunktion $J_u$ ist eine Stammfunktion von $f$:
    \[J_u'(x)=f(x)\]
\end{boxx}

\begin{boxx}[DarkBlue]{Beispiel}
    $a)$\hspace{3mm} $\displaystyle f(t) = 3t^2$. 
    Bestimme $x$ so, dass gilt $\displaystyle \int_1^x f(t) \, dt = 26$.

    \begin{align*}
        \int_1^x 3t^2 \, dt &= 26 \\
        \left[t^3\right]_1^x &= 26 \\
        x^3 - 1 &= 26 \\
        x = \sqrt[3]{27} &= 3
    \end{align*}
    \newpage
    $b)$\hspace{3mm} Bestimme eine Integralfunktion von $\displaystyle f(t) = \frac{1}{4} e^{0.5t}$
    zur unteren Grenze $u=-1$.

    \begin{align*}
        J_{-1}(x) &= \int_{-1}^x \frac{1}{4}e^{0.5t} \, dt = \left[\frac{1}{2}e^{0.5t}\right]_{-1}^x \\
        &= \frac{1}{2} e^{0.5x} - \frac{1}{2\sqrt{e}}
    \end{align*}
\end{boxx}

\subsection{Bestimmen von Stammfunktionen}

Seien $G$ und $H$ Stammfunktion von $g$ und $h$.
\\\\
\textbf{Potenzregel:}
\begin{align*}
    &&&&&& f(x) &= x^n & F(x) &= \frac{x^{n+1}}{n+1} &&&&&&
\end{align*}
für $n=-1$:
\begin{align*}
    &&&&&& f(x) &= \frac{1}{x} = x^{-1}& F(x) &= \ln |x| &&&&&&
\end{align*}

\textbf{Faktorregel:}
\begin{align*}
    &&&&&& f(x) &= c \cdot g(x) & F(x) &= c \cdot G(x) &&&&&&
\end{align*}

\textbf{Summenregel:}
\begin{align*}
    &&&&&& f(x) &= g(x) + h(x) & F(x) &= G(x) + H(x) &&&&&&
\end{align*}

\textbf{Lineare Verkettung/Substitution:}
\begin{align*}
    &&&&&& f(x) &= g(ax + b) & F(x) &=\frac{1}{a}G(ax+b) &&&&&&
\end{align*}

\textbf{Faktor und Summenregel für Integrale:}

\[\int_a^b c \cdot f(x) \, dx = c \cdot \int_a^b f(x) \, dx \\\]

\[\int_a^b g(x) + h(x) \, dx = \int_a^b g(x) \, dx + \int_a^b h(x) \, dx\]

Beweise: HDI

\begin{boxx}[DarkBlue]{Beispiel}
    $a)$\hspace{3mm} Bestimme eine Stammfunktion von $\displaystyle f(x) = 3e^{2x}-2 \sin(\pi x)$

    \begin{align*}
        F(x) &= \frac{3}{2} e^{2x} + \frac{2}{\pi} \cos(\pi x)
    \end{align*}

    $b)$\hspace{3mm} Berechne das Integral $\displaystyle \int_{-4}^{-1} \frac{5}{x} \, dx$

    \begin{align*}
        \int_{-4}^{-1} \frac{5}{x} \, dx &= \left[5 \ln |x|\right]_{-4}^{-1} 
        = 5 \ln 1 - 5\ln 4 = -5\ln 4 \approx -6.93
    \end{align*}
\end{boxx}
\newpage

\subsection{Graphen von Stammfunktionen}
Nach dem Schema:

\begin{align*}
    &&N \;\; E \;\; &W && F &&\\
    &&N \;\; &E \;\; W && f &&\\
    &&&N \;\; E \;\; W && f' &&
\end{align*}

gilt für die Graphen von $F$ und $f$ folgendes:

\begin{table}[H]
    \centering
    \begin{tabular}{|c|c|c|c|}
        \hline
         \textbf{Nullstelle von \emph{f}} & mit VZW von + nach - & mit VZW mit - nach + & ohne VZW  \\
         \hline
         \textbf{(innere) Extremstelle von \emph{F}} & Maximumstellen & Minimumstelle & Sattelstelle von $F$  \\
         \hline
    \end{tabular}
\end{table}

\begin{table}[H]
    \centering
    \begin{tabular}{|c|c|}
        \hline
         \textbf{Funktion \emph{f}} & (innere) Extremstelle  \\
         \hline
         \textbf{Stammfunktion \emph{F}} & Wendestelle \\
         \hline
    \end{tabular}
\end{table}

\begin{boxx}[DarkBlue]{Beispiel}
    Gegeben ist $G_f$. Skizziere $G_F$. 
    \begin{figure}[H]
        \centering
        \begin{tikzpicture}
        \begin{axis}[
            axis lines = middle,
            xlabel = {\(x\)},
            ylabel = {\(y\)},
            ymin = -7.5,
            ymax = 7.5,
            xmax = 3,
            xmin = -3,
        ]
            \addplot[
                thick,
                color = Blue,
                domain = -3:3,
                samples = 100,
            ]
            {x*(x-2)*(x+2)} node[anchor = north west, pos = 0.6] {\(G_f\)};
        \end{axis}
        \end{tikzpicture}
    \end{figure}
    \begin{figure}[H]
        \centering
        \begin{tikzpicture}
        \begin{axis}[
            axis lines = middle,
            xlabel = {\(x\)},
            ylabel = {\(y\)},
            ymin = -7.5,
            ymax = 7.5,
            xmax = 3,
            xmin = -3,
        ]
            \addplot[
                thick,
                color = DarkPurple,
                domain = -3:3,
                samples = 100,
            ]
            {0.25*x^4-2*x^2+5} node[anchor = south west, pos = 0.1] {\(G_F\)};
            \addplot[
                dashed,
                thick,
                color = DarkPurple,
                domain = -3:3,
                samples = 100,
            ]
            {0.25*x^4-2*x^2-1};
        \end{axis}
        \end{tikzpicture}
    \end{figure}
\end{boxx}
\newpage
\subsection{Integral und Flächeninhalt}

Ein Integral kann einen negativen Wert haben,
ein Flächeninhalt nicht.

Das bedeutet, dass man beim Berechnen des Flächeninhalts
zwischen eines Graphen und der $x$-Achse wie folgt vorgehen
sollte:
\begin{enumerate}
    \item Bestimme die Nullstellen von $f$
    \item Berechne die Integrale der Teilintervalle
    \item Berechne die Teilflächen für $f(x)<0$ über 
    $\displaystyle \left|\int_{a}^{x_0} f(x) \, dx\right|$
    oder $\displaystyle - \int_{a}^{x_0} f(x) \, dx$
\end{enumerate}

Für den Flächeninhalt $A$ zwischen zwei Graphen gilt:

\[A = \int_a^b f(x) - g(x) \; dx\]

Salopp: ''Das Obere minus das Untere.''

\begin{boxx}[DarkBlue]{Beispiel}
    $\displaystyle f(x) = x^3;\;g(x) = x;\; I = [-2;5]$

    Berechne den Flächeninhalt zwischen den beiden Funktionsgraphen für $I$.

    \textbf{Skizze:}
    \begin{figure}[H]
    \centering
    \begin{tikzpicture}
    \begin{axis}[
        axis lines = middle,
        xlabel = {\(x\)},
        ylabel = {\(y\)},
        ymin = -3,
        ymax = 8.5
    ]
    \addplot[
        thick,
        color = LightRed,
        samples = 100,
        domain = -2:3,
    ]{x^3} node[anchor = east, pos = 0.45] {\(G_f\)};
    \addplot[
        name path = F,
        thick,
        color = LightRed,
        samples = 100,
        domain = -2:5,
    ]{x^3};
    \addplot[
        thick,
        color = LightBlue,
        samples = 100,
        domain = -3:6,
    ]{x}node[anchor = north west, pos = 0.8] {\(G_g\)};
    \addplot[
        name path = G,
        thick,
        color = LightBlue,
        samples = 100,
        domain = -2:5,
    ]{x};
    \addplot[LightYellow!30, domain = -2:5] fill between[of=G and F];
    \end{axis}
    \end{tikzpicture}    
    \end{figure}
    \begin{align*}
        f(x) &= g(x) \\
        x^3 &= x \\
        x^3 - x &= 0 \\
        x(x^2-1) &= 0 \\
        \Rightarrow x_1 = 0;\; x_2 &= 1;\; x_3 = -1
    \end{align*}
    \begin{align*}
        A &=  \int_{-2}^{-1} g(x) - f(x) \; dx + 
        \int_{-1}^{0} f(x) - g(x) \; dx + 
        \int_{0}^{1} g(x) - f(x) \; dx + 
        \int_{1}^{5} f(x) - g(x) \; dx \\
        &= \left[\frac{1}{2} x^2- \frac{1}{4} x^4\right]_{-2}^{-1} +
        \left[\frac{1}{4} x^4 - \frac{1}{2} x^2\right]_{-1}^{0} + 
        \left[\frac{1}{2} x^2- \frac{1}{4} x^4\right]_{0}^{1} + 
        \left[\frac{1}{4} x^4 - \frac{1}{2} x^2\right]_{1}^{5} \\
        &= \frac{9}{4} + \frac{1}{4} + \frac{1}{4} + 144 \\
        &= 146.75
    \end{align*}
\end{boxx}
\newpage

\subsection{Volumen von Rotationskörpern}
Sei $f$ eine auf $[a;b]$ integrierbare Funktion.
Lässt man die Fläche, die der Graph von $f$ mit der
$x$-Achse einschließt um die $x$-Achse rotiert,
so entsteht ein 3-dimensionaler Rotationskörper,
dessen Volumen man wie folgt berechnet:

\[V_{\text{Rot}} = \pi \int_a^b f(x)^2 \; dx\]

\begin{boxx}[DarkBlue]{Beispiel}
    Der rotierende Graph von $f$ mit $\displaystyle f(x) = \frac{3}{4}\sqrt{x}$ 
    erzeugt einen Rotationskörper in Form eines Sektglases.

    \begin{figure}[H]
        \centering
        \begin{tikzpicture}
        \begin{axis}[
            axis lines = left,
            % axis line style = {black!80},
            colormap/cool,
            grid = major,
            xlabel = {\(x\)},
            zlabel = {\(y\)},
            zlabel style={rotate=-90},
            xtick={2,4,6,8,10},
            xmin = 0, xmax = 10,
            ytick={-2.5,0,2.5},
            ymin = -2.5, ymax = 2.5,
            ztick={-2.5,0,2.5},
            zmin = -2.5, zmax = 2.5,
            view = {15}{30},
        ]
        \addplot3 [
            surf, 
            domain=0:10,
            domain y = -5:5,
            samples = 20,
            samples y = 30,
            fill = LightYellow!60,
            faceted color = Black!70,
        ] (x, {0.75*x^(0.5)*cos(deg(y))}, {0.75*x^(0.5)*sin(deg(y))});
        \addplot3[
            ultra thick,
            color = Red,
            domain=0:10,
            domain y = 0:0,
            samples = 50,
        ] {0.75*x^(0.5)}  node[anchor = south east, pos = 0.4] {\(G_f\)};
        \end{axis}
        \end{tikzpicture}
    \end{figure}

    $a)$\hspace{3mm} Wie viel Sekt passt in das $\qty{10}{\centi\meter}$ Hohe Sektglas?
    \begin{align*}
        V &= \pi \int_0^{10} f(x)^2 \, dx \\
        &= \pi \int_0^{10} \frac{16}{9} x \, dx \\
        &= \pi \cdot \left[\frac{8}{9} x^2\right]_0^{10} \\
        &= \pi \cdot \frac{800}{9} \approx 279.25
    \end{align*}
    $b)$\hspace{3mm} An welcher Stelle muss man den \qty{100}{\milli\liter} Eichstrich anbringen?
    \begin{align*}
        V &= 100 \\
        \pi \int_0^a f(x)^2 \, dx &= 100 \\
        \pi \cdot \left[\frac{8}{9}x^2\right]_0^a &= 100 \\
        \pi \cdot \frac{8}{9} a^2 &= 100 \\
        a^2 &= \frac{900}{8 \pi } \\
        a &= \sqrt{\frac{900}{8 \pi}} \approx 5.98
    \end{align*}
\end{boxx}

\newpage

\subsection{Uneigentliche Integrale}
\textbf{Problem:} \\
Sind die Flächen, die ein Graph mit z.B. der $x$-Achse einschließt endlich, wenn
\begin{itemize}
    \item eine Polstelle (senkrechte Asymptote) vorliegt,
    \item die Funktion für $x \to \pm \infty$ läuft und der
    Graph dabei eine waagerechte Asymptote besitzt?
\end{itemize}
Dabei sind die betrachteten Flächen \textbf{unbegrenzt}.
Das heißt aber nicht, dass automatisch der Flächeninhalt endlich
oder unendlich sein muss.
\\\\
Untersucht man den Flächeninhalt an seiner Polstelle $z$, berechnet man das Integral
\[\int_z^b f(x) \, dx\]
Nach dem Bilden der Stammfunktion setzt man $z$ ein und entscheidet, 
ob der Grenzwert des Rechenausdrucks einen endlichen Flächeninhalt
ergibt oder nicht.

Bei waagerechten Asymptoten verfährt man analog nur mit dem Grenzwert für $x \to \pm \infty$.
\\\\
Erhält man für den Grenzwert einen endlichen Wert, spricht man
von einem \textbf{uneigentlichen Integral}.

\begin{boxx}[DarkBlue]{Beispiel}
    Berechne! \\\\
    $a)$\hspace{3mm} $\displaystyle \int_1^\infty \frac{3}{x} \, dx$ \\
    \begin{align*}
        \int_1^z \frac{3}{x} \, dx = \left[3 \ln|x|\right]_1^z 
        &= \ln|z| \\
        \lim_{z \to \infty} \ln|z|  &= \infty 
    \end{align*}
    $b)$\hspace{3mm} $\displaystyle \int_{-\infty}^0 \frac{1}{2} e^{\frac{1}{2}x} \, dx$
    \begin{align*}
        \int_z^0 \frac{1}{2} e^{\frac{1}{2}x} \, dx = \left[e^{\frac{1}{2}x}\right]_z^0
        &= -e^{\frac{1}{2}z} + 1 \\
        \lim_{z \to -\infty} -e^{\frac{1}{2}z} + 1 &= 1 \\
        \Rightarrow \int_{-\infty}^0 \frac{1}{2} e^{\frac{1}{2}x} \, dx &= 1
    \end{align*}
\end{boxx}

\newpage

\subsection{Mittelwerte von Funktionen}
\begin{boxx}[Red]{Definition: Mittelwert}
    Die Zahl 
    \[\overline{m} = \frac{1}{b-a}\int_a^b f(x) \, dx\]
    heißt \textbf{Mittelwert} der Funktion $f$ über $[a;b]$.
\end{boxx}

Graphisch lässt sich der Mittelwert bestimmen, indem man die Fläche,
die der Graph von $f$ mit der $x$-Achse einschließt mit der 
Rechtecksfläsche, die durch die konstante Funktion $g(x) = m$
entsteht, vergleicht. Beide Flächen müssen gleich groß sein.

Alternativ müssen die beiden Teilflächen $A_1$ und $A_2$ ober-
bzw. unterhalb der Mittelwertslinie gleich groß sein. \\\\
\textbf{Anschaulich:}
\begin{figure}[H]
    \centering
    \begin{tikzpicture}
    \begin{axis}[
        clip = false,
        ticks = none,
        height = 7cm,
        width = 10cm,
        axis lines = middle,
        xmin=-1,
        xmax=2.5,
        ymin=0,
        ymax=2.5,
        xlabel = {\(x\)},
        ylabel = {\(y\)},
    ]
    \addplot[
        thick,
        color=DarkBlue,
        domain = -1:2.5,
        samples = 50,
    ]{0.012*(x+2.5)^3 + 0.5};
    \addplot[] coordinates {
        (0.4,0) (0.4,2)
    } node[anchor = north, pos = 0] {\(a\)};
    \addplot[] coordinates {
        (2.2,0) (2.2,2)
    } node[anchor = north, pos = 0] {\(b\)};
    \addplot[
        thick,
        color=DarkPurple,
        domain=0.3:2.3,
    ]{1.19} node[anchor = west, pos = 1] {\(\overline{m}\)};
    \addplot[
        name path = M,
        thick,
        color=DarkPurple,
        domain=0.4:2.2,
    ]{1.19};
    \addplot[
        name path = F,
        thick,
        color=DarkBlue,
        domain=0.4:2.2,
        samples = 50,
    ]{0.012*(x+2.5)^3 + 0.5};
    \addplot[LightYellow!30] fill between[of=F and M];
    \node[DarkYellow] at (axis cs:0.7,1.05) {\(A_1\)};
    \node[DarkYellow] at (axis cs:2,1.37) {\(A_2\)};
    \end{axis}
    \end{tikzpicture}
\end{figure}
\begin{boxx}[DarkBlue]{Beispiel}
    $a)$\hspace{3mm} $\displaystyle f(x)=-x^2+4;\;a=-3;\;b=4$. 
    Bestimme $\overline{m}$!
    \begin{align*}
        \overline{m} &= \frac{1}{4-(-3)}\int_{-3}^4 -x^2+4 \, dx \\
        &= \frac{1}{7} \left[-\frac{1}{3}x^3+4x\right]_{-3}^4 \\
        &= \frac{1}{7} \left(-\frac{4^3}{3}+16-\left(-\frac{(-3)^3}{3}-12\right)\right) \\
        &= -\frac{1}{3}
    \end{align*}

    $b)$\hspace{3mm} Bestimme graphisch den Mittelwert von $g$.
    \begin{figure}[H]
        \centering
        \begin{tikzpicture}
        \begin{axis}[
            height = 6cm,
            width = 11cm,
            axis lines = middle,
            xtick = {\empty},
            ytick = {1,2,3,4,5},
            xlabel = {\(x\)},
            ylabel = {\(y\)},
        ]
        \addplot[
            thick,
            color=DarkBlue,
            domain=-1.2:7.2,
            samples=100,
        ]{0.4*(((x^3)/(3))- 3*x^2+5*x)+3.7}
        node[anchor=south east,pos=0.92] {\(G_g\)};
        \addplot[] coordinates {
            (1,0) (1,5)
        } node[anchor = north, pos = 0] {\(a\)};
        \addplot[] coordinates {
            (5,0) (5,5)
        } node[anchor = north, pos = 0] {\(b\)};
        \addplot[thick, DarkPurple, domain = 1:5]{2.5}
        node[anchor = west, pos = 1] {\(\overline{m}\approx 2.5\)};
        \addplot[thick,dashed, DarkPurple] coordinates {
            (0,2.5) (1,2.5)
        };
        \end{axis}
        \end{tikzpicture}
    \end{figure}
\end{boxx}
\section{Funktionen und ihre Graphen}
\subsection{Strecken, verschieben, spiegeln}
Gegeben sei der Graph der Funktion $f$. 
Der in \color{Green}$x$-Richtung verschobene\color{black}, in \color{Yellow}$y$-Richtung verschobene \color{black}
und in \color{Purple}$y$-Richtung \color{black} gestreckte Graph der Funktion $g$ 
besitzt den Funktionsterm:
\begin{align*}
    g(x) = \color{Purple} a \color{black}\cdot f(x \color{Green}- c\color{black}) \color{Yellow} + d
\end{align*}
Bei den Spiegelungen von $f$ gilt:
\begin{itemize}
    \item $g(x) = f(-x)\;\;\;$ Spiegelung an der \textbf{\emph{y}-Achse}
    \item $g(x) = -f(x)\;\;\;$ Spiegelung an der \textbf{\emph{x}-Achse}
    \item $g(x) = -f(-x)\;\;\;$ Spiegelung am \textbf{Ursprung}
\end{itemize}
\begin{boxx}[DarkBlue]{Beispiel}
    Skizziere die Graphen von $f$ und $g$.\\

    $a)$\hspace{3mm} $\displaystyle f(x) = - \sqrt{x+2} - 3;\;\;\;\; x\geq -2$ \\
    \begin{figure}[H]
        \centering
        \begin{tikzpicture}
        \begin{axis}[
            width = 12cm,
            height = 7cm,
            axis lines = middle,
            xlabel = {\(x\)},
            ylabel = {\(y\)},
            xmin = -3,
            xmax = 8,
            ymin = -7,
            ymax= 3,
        ]
        \addplot[dashed, thick, Blue, domain=-0:6,samples=100] {x^0.5} node[anchor = north west, pos = 1] {\(y = \sqrt{x}\)};
        \addplot[dashed, thick, Green, domain=-2:3,samples=100] {(x+2)^0.5} node[anchor = south east, pos = 0.3] {\(y = \sqrt{x+2}\)};
        \addplot[dashed, thick, Purple, domain=-2:3,samples=100] {-(x+2)^0.5} node[anchor = west, pos = 1] {\(y = -\sqrt{x+2}\)};
        \addplot[thick, smooth, Yellow, domain=-2:7.5,samples=100] {-(x+2)^0.5-3} node[anchor = south west, pos = 0.7] {\(y = -\sqrt{x+2}-3\)};
        \end{axis}
        \end{tikzpicture}
    \end{figure}
    $b)$\hspace{3mm} $\displaystyle g(x) = \frac{2}{x+1}- \frac{1}{2};\;\;\;\; x \not = -1$
    \begin{figure}[H]
        \centering
        \begin{tikzpicture}
        \begin{axis}[
            width = 11cm,
            height = 9cm,
            axis lines = middle,
            xlabel = {\(x\)},
            ylabel = {\(y\)},
            xmin = -7,
            xmax = 7,
            ymin = -7,
            ymax= 7,
            restrict y to domain=-12:12
        ]
        \addplot[dashed, thick, Blue, domain=-7:7,samples=200] {1/x} node[anchor = west, pos = 0.7] {\(\displaystyle y = \frac{1}{x}\)};
        \addplot[dashed, thick, Green, domain=-7:7,samples=200] {1/(x+1)} node[anchor = east, pos = 0.67] {\(\displaystyle y = \frac{1}{x+1}\)};
        \addplot[dashed, thick, Purple, domain=-7:7,samples=200] {2/(x+1)}node[anchor = south, pos = 0.92] {\(\displaystyle y = \frac{2}{x+1}\)};
        \addplot[thick, Yellow, domain=-7:7,samples=200] {2/(x+1)-0.5} node[anchor = north east, pos = 0.15] {\(\displaystyle y = \frac{2}{x+1} - \frac{1}{2}\)};
        \end{axis}
        \end{tikzpicture}
    \end{figure}
    Zeige, dass die Graphen von $f_k$ mit $\displaystyle f_k(x) = k x e^{x^2};\; k \in \mathbb{R}$
    punktsymmetrisch zum Ursprung sind.
    \begin{align*}
        f(-x) &= k \cdot (-x) \cdot e^{(-x)^2} \\
        &= - k x e^{x^2} \\
        &= -f(x) 
    \end{align*}
    \qed
\end{boxx}
\subsection{Linearfaktordarstellung - mehrfache Nullstellen}
\begin{boxx}[LightGreen]{Satz 1}
    Hat eine ganzrationale Funktion vom Grad $n$ eine Nullstelle
    $x_0$, so gilt:
    \begin{align*}
        f(x) &= (x - x_0)\cdot g(x) & &\text{wobei } g \text{ vom Grad } n-1 \text{ ist.}
    \end{align*}
    $(x- x_0)$ nennt man \textbf{Linearfaktor}.
\end{boxx}
\begin{boxx}[LightGreen]{Satz 2}
    Eine ganzrationale Funktion $n$-ten Grades besitzt höchstens $n$ Nullstellen.
\end{boxx}
\begin{boxx}[LightGreen]{Satz 3}
    Sei $\displaystyle f(x) = (x-a)^k \cdot g(x)$.
    \begin{itemize}
        \item Für $k = 1:$ Schnittstelle von $f$ mit der $x$-Achse.
        \item Für $k = 2:$ Berührstelle von $f$ an der $x$-Achse.
        \item Für $k = 3:$ Sattelstelle von $f$ an der $x$-Achse.
    \end{itemize}
\end{boxx}
\begin{boxx}[DarkBlue]{Beispiel}
    $a)$\hspace{3mm} Skizziere den Graph von $f$ mit $\displaystyle f(x) = x^3(x-2)^2(x+1)$.
    \begin{figure}[H]
        \centering
        \begin{tikzpicture}
        \begin{axis}[
            width = 10cm,
            height = 7cm,
            axis lines = middle,
            xlabel = {\(x\)},
            ylabel = {\(y\)},
            restrict y to domain=-8:8,
            xmax = 2.75,
            xmin = -1.75,
            ymax = 5,
            ymin = -1,
            ytick = {\empty},
        ]
            \addplot[thick, DarkBlue, samples = 100, domain = -1.75:2.75]{x^3*(x-2)^2*(x+1)} node[anchor = east, pos = 0.85] {\(G_f\)};
        \end{axis}
        \end{tikzpicture}
    \end{figure}
    \newpage
    $b)$\hspace{3mm} Bestimme die Funktionsgleichung des folgenden Graphen.
    \begin{figure}[H]
        \centering
        \begin{tikzpicture}
        \begin{axis}[
            width = 10cm,
            height = 7cm,
            axis lines = middle,
            xlabel = {\(x\)},
            ylabel = {\(y\)},
            restrict y to domain = -3:3,
            ymin = -2.5,
            ymax = 3.2,
            xtick = {-1,1,2}
        ]
            \addplot[thick, DarkBlue, domain = -2:3, samples = 100]{0.5*(x+1)^3*(x-1)*(x-2)^2} node[anchor = south west, pos = 0.1] {\(G_g\)};
        \end{axis}
        \end{tikzpicture}
    \end{figure}
    \begin{align*}
        g(x) &= a(x+1)^3 (x-1) (x-2)^2 \\
        \text{mit } a &= 1: \;g(0) = -4 \Rightarrow a = \frac{1}{2} \\
        g(x) &= \frac{1}{2} (x+1)^3 (x-1) (x-2)^2
    \end{align*}
\end{boxx}

\subsection{Lösen von Gleichungen}
Folgende Strategien zum Lösen von diversen Gleichungen sind
zielführend:
\subsubsection*{Betragsgleichungen}
Führe eine Fallunterscheidung durch:
\begin{itemize}
    \item für positive Beträge kann man den Betrag weglassen und
    die Gleichung wie gewohnt lösen.
    \item für negative Beträge wird eine Seite der Gleichung mit
    $-1$ multipliziert.
\end{itemize}
\begin{boxx}[DarkBlue]{Beispiel}
    \begin{align*}
        && \left|\frac{10}{e^x-1}\right| &= 2 \\
        &\text{Fall 1:} & \frac{10}{e^x-1} &= 2 & &|\; \cdot e^x - 1 \\ 
        && 10 &= 2e^x - 2 & &|\; + 2 ;\; \cdot \frac{1}{2} \\
        && 6 &= e^x  & &|\; \ln\\
        && x &= \ln 6 \\
        &\text{Fall 2:} & -\frac{10}{e^x-1} &= 2 & &|\; \cdot e^x - 1 \\
        && -10 &= 2e^x - 2 & &|\; + 2 ;\; \cdot \frac{1}{2} \\
        && -4 &= e^x \Rightarrow \text{keine Lösung} \\
        &\text{Probe:} & \left|\frac{10}{e^{\ln 6}-1}\right| &= \left|\frac{10}{5}\right| = 2 \\
        && &\Rightarrow \mathbb{L} = \{\ln 6\}
    \end{align*}
\end{boxx}
\subsubsection*{Wurzelgleichungen}
\begin{itemize}
    \item isoliere die Wurzel
    \item quadriere beide Seiten der Gleichung
\end{itemize}
\begin{boxx}[DarkBlue]{Beispiel}
    \begin{align*}
        \sqrt{20 - 2x} + 6 &= x & &|\; -6 \\
        \sqrt{20 - 2x} &= x-6 & &|\; ()^2 \\
        20 - 2x &= (x-6)^2 \\
        20 - 2x &= x^2 - 12x + 36 & &|\; -20 + 2x \\
        x^2 - 10x + 16 &= 0 \\
        (x-2)(x-8) &= 0 \\
        x_1 = 2;\; x_2 &= 8
    \end{align*}
    \begin{align*}
        &\text{Probe:} &  \sqrt{20 - 2\cdot 2} + 6 &= 2 \Leftrightarrow 4 + 6 \not = 2 &&\\
        & &  \sqrt{20 - 2\cdot 8} + 6 &= 8 \Leftrightarrow 2 + 6  = 8 &&\\
        & & \Rightarrow \mathbb{L} &= \{8\}
    \end{align*}
\end{boxx}
\subsubsection*{Bruchgleichungen}
\begin{itemize}
    \item Bestimme den Hauptnenner
    \item Beide Seiten mit dem Hauptnenner durchmultiplizieren
\end{itemize}
\begin{boxx}[DarkBlue]{Beispiel}
    \begin{align*}
        \frac{6}{x^4} - \frac{5}{x^2} &= -1 & &|\; \cdot x^4 \\
        6 - 5x^2 &= -x^4 & &|\; + x^4 \\
        x^4 - 5x^2 + 6 &= 0  & &|\; u = x^2 \\
        u^2 - 5u +6 &= 0 \\
        (u-2)(u-3) &= 0 \\
        u_1 = 2;\; u_2 &= 3 & &|\; x^2 = u
    \end{align*}
    \begin{align*}
        x^2 &= 2 & x^2 &= 3\\
        x_1 &= \pm \sqrt{2} &  x_2 &= \pm \sqrt{3}
    \end{align*}
    \begin{center}
        $x^4 \not = 0$ und $x^2 \not = 0$ für $x = \pm \sqrt{2}$ oder $x = \pm \sqrt{3}$
    \end{center}
    \[\Rightarrow \mathbb{L} = \{\sqrt{2};\,-\sqrt{2};\,\sqrt{3};\,-\sqrt{3}\}\]
\end{boxx}
\newpage
\subsubsection*{Ungleichungen}
Entweder: Mit Vergleichszeichen auflösen und aufpassen bei
Mulitplikation oder Division mit negativen Zahlen.
\\\\
Oder: Eine Gleichung lösen und Werte größer und kleiner als
die Lösung testen.
\begin{boxx}[DarkBlue]{Beispiel}
    \begin{align*}
        1 - \left(\frac{1}{2}\right)^x &< 0.05 & &|\; - 1 \\
        - \left(\frac{1}{2}\right)^x &< -0.95 & &|\; \cdot (-1) \\
        \left(\frac{1}{2}\right)^x &> 0.95 & &|\; \log \\
        x \log 0.5 &> \log 0.95 & &|\; \cdot \frac{1}{\log 0.5} \\
        x &< \frac{\log 0.95}{\log 0.5} \approx 0.074
    \end{align*}
\end{boxx}

\subsection{Trigonometrische Funktionen}

Gegeben sei die Funktion $f$ mit $f(x) = \sin x$.
Der Graph von der Funktion $g$ mit
\begin{align*}
    g(x) = a \sin\left(b(x-c)\right) + d
\end{align*}
ist gegenüber dem Graph von $f$
\begin{itemize}
    \item um $|a|$ Einheiten in $y$-Richtung gestreckt,
    \item um $d$ Einheiten in y-Richtung verschoben,
    \item besitzt die Periode 
    $\displaystyle p = \frac{2 \pi}{b}$ 
    (Streckung in $x$-Richtung) und
    \item um $c$ Einheiten in $x$-Richtung verschoben.
\end{itemize}

Für $a<0$ wird der Graph zusätzlich an der $x$-Achse
gespiegelt.

\begin{boxx}[DarkBlue]{Beispiel}
    $a)$\hspace{3mm} Gib im Intervall $I=[0;2\pi]$
    zwei Lösungen der Gleichung $\sin x = -0.6$ an.
    \begin{figure}[H]
        \centering
        \begin{tikzpicture}
        \begin{axis}[
            clip = false,
            height=5cm,
            width=11cm,
            axis lines = middle,
            xlabel = {\(x\)},
            ylabel = {\(y\)},
            xtick={-6.28318, -4.7123889, -3.14159, -1.5708, 1.5708, 3.14159, 4.7123889, 6.28318},
            xticklabels={$-2\pi$, $-\frac{3\pi}{2}$,$-\pi$, $-\frac{\pi}{2}$, 
             $\frac{\pi}{2}$,$\pi$, $\frac{3\pi}{2}$, $2\pi$},
            xmin=-4.5,
            xmax=8,
            ymin=-1.2,
            ymax=1.2, 
        ]
        \addplot[thick, DarkBlue, domain=-3.14159:6.28318, samples =100]{sin(deg(x))};
        \addplot[LightRed, thick, domain=-4:7]{-0.6} node[anchor=west, pos = 1] {\(y=-0.6\)};
        \node[label={315:{\(x_2\)}},circle,fill,inner sep=1.5pt] at (axis cs:5.64,-0.6) {};
        \node[label={225:{\(x_1\)}},circle,fill,inner sep=1.5pt] at (axis cs:3.78,-0.6) {};
        \addplot[] coordinates {
            (-0.64,0) (-0.64,-0.6)
        } node[anchor = south, pos = 0] {\(-0.64\)};
        \end{axis}
        \end{tikzpicture}
    \end{figure}
    \begin{align*}
        \sin^{-1} (-0.6) &\approx -0.64 \\
        x_2 &= -0.64 + 2\pi \approx 5.64 \\ 
        x_1 &= \pi + 0.64 \approx 3.78
    \end{align*}

    $b)$\hspace{3mm} Skizziere den Graphen von
    $f(x) = 2\sin\left(2\pi(x - 1)\right) - 1$.
    \begin{figure}[H]
        \centering
        \begin{tikzpicture}
        \begin{axis}[
            height=6cm,
            width=12cm,
            axis lines = middle,
            xlabel = {\(x\)},
            ylabel = {\(y\)},
            ytick = {1, -1, -2, -3},
            xmin=-3.5,
            xmax=3.5,
            ymin=-3.5,
            ymax=1.6,
            grid = major,
            clip = false
        ]
        \addplot[thick, DarkBlue, domain=-3.2:3.2, samples=200, smooth]{2*sin(deg(2*pi*(x-1)))-1} node[pos=0.01, anchor = east] {\(G_f\)};
        \end{axis}
        \end{tikzpicture}
    \end{figure}
\end{boxx}

\subsection{Senkrechte und waagerechte Asymptoten}

Gegeben sind die ganzrationalen Funktionen $g(x)$ und
$h(x)$. Die Funktion $f$ mit $\displaystyle f(x) = \frac{g(x)}{h(x)}\;\;\;(h(x)\not= 0)$
nennt man \textbf{gebrochenrationale Funktion}.

Wenn $g\left(x_0\right) \not= 0$ und $h\left(x_0\right) = 0$ gilt,
dann ist $x_0$ eine \textbf{Polstelle} von $f$ und die Gerade mit
$x=x_0$ ist eine \textbf{senkrechte Asymptote} des Graphen von $f$.

Gilt $g\left(x_0\right) = 0$ und $h\left(x_0\right) = 0$,
dann liegt keine senkrechte Asymptote, sondern eine \textbf{hebbare Definitionslücke} vor.

Für die waagerechte Asymptote gilt:
\begin{enumerate}
    \item Zählergrad > Nennergrad: keine waagerechte Asymptote
    \item Zählergrad = Nennergrad: höchste Potenz von $x$ ausklammern und 
    $\displaystyle \lim_{x\to\pm\infty} f(x)$ bilden. $\displaystyle \left(y = \frac{a}{b}\right)$
    \item Zählergrad < Nennergrad: waagerechte Asymptote bei $y = 0$.
\end{enumerate}
\begin{boxx}[DarkBlue]{Beispiel}
    $a)$\hspace{3mm} Bestimme die senkrechte und waagerechte Asymptote.\\\\
    $\displaystyle f(x) = \frac{6x^2+3}{5x^2- \sfrac{1}{2}}$ 
    \begin{align*}
        5x^2 - \frac{1}{2} = 0 \Leftrightarrow x &= \pm\frac{1}{\sqrt{10}} & &\rightarrow \text{Senk. Asymp. bei } x = \pm\frac{1}{\sqrt{10}}\\
        \lim_{x\to\pm\infty} \frac{\cancel{x^2}\left(6 + \frac{3}{x^2}\right)}{\cancel{x^2}\left(5 - \frac{\sfrac{1}{2}}{x^2}\right)} &= \frac{6}{5}
        & &\rightarrow \text{Wag. Asymp. bei } y = \frac{6}{5}
    \end{align*}
    $\displaystyle g(x) = \frac{7x}{x^2 - 1}$
    \begin{align*}
        x^2 - 1 = 0 \Leftrightarrow x &= \pm 1 & &\rightarrow \text{Senk. Asymp. bei } x = \pm 1\\
        & & &\rightarrow \text{Wag. Asymp. bei } y = 0
    \end{align*}
    $\displaystyle h(x) = \frac{3x^2 + 2}{7x - 2}$ \\
    \begin{align*}
        7x - 2 = 0 \Leftrightarrow x &= \frac{2}{7} & &\rightarrow \text{Senk. Asymp. bei } x = \frac{2}{7} \\
        & & &\rightarrow \text{keine wag. Asymp.}
    \end{align*}
    \newpage
    $b)$\hspace{3mm} Skizziere den Graphen von $t$ mit 
    $\displaystyle t(x) = \frac{3x^2 + 1}{x^2 + 4x + 4}$.
    \begin{figure}[H]
        \centering
        \begin{tikzpicture}
        \begin{axis}[
            height=7cm,
            width=10cm,
            axis lines = middle,
            xlabel = {\(x\)},
            ylabel = {\(y\)},
            xmax= 15,
            xmin = -15,
            restrict y to domain = -20:20
        ]
        \addplot[thick, DarkBlue, domain=-20:20, samples = 500]{(3*x^2+1)/(x^2 +4*x + 4)} node[pos=0.1, anchor = south] {\(G_f\)};
        \addplot[thick,dashed, LightPurple, domain=-15:15]{3} node[pos=0.57, anchor=south]{\(y = 3\)};
        \addplot[thick,dashed, LightPurple, domain=-15:15] coordinates {(-2,20) (-2, 0)} node[pos=0.92,anchor=east] {\(x=-2\)};
        \end{axis}
        \end{tikzpicture}
    \end{figure}
\end{boxx}
\subsection{Vollständige Kurvendiskussion}
Strategie zum "`Zeichnen"' eines Graphen:
\begin{enumerate}
    \item Nullstellen bestimmen $\displaystyle \left(f(x) = 0\right)$,
    \item senkrechte und waagerechte Asymptoten bestimmen,
    \item Symmetrie zur $y$-Achse und zum Ursprung,
    \item Hoch-, Tief- und Wendepunkte,
    \item Verhalten für $\displaystyle x \to \pm \infty$, bzw. Vorzeichenwechsel an einer Polstelle,
    \item Verhalten für $x \to 0$ (kleinste Potenz von $x$ betrachten).
\end{enumerate}
\begin{boxx}[DarkBlue]{Beispiel}
    Skizziere den Graphen von $f$ mit $\displaystyle f(x) = \frac{(x+1)^2}{2x^2}$

    \begin{wrapfigure}{R}{0pt}
        \begin{tikzpicture}
        \begin{axis}[
            axis lines = middle,
            width = 9cm,
            height = 8cm,
            xlabel = {\(x\)},
            ylabel = {\(y\)},
            restrict y to domain = 0:8,
            ymax = 6
        ]
            \addplot[DarkBlue, thick, domain=-6:6, samples=500]{((x+1)^2)/(2*x^2)};
            \addplot[LightPurple, dashed, domain = -6:6]{0.5};
        \end{axis}
        \end{tikzpicture}
    \end{wrapfigure}
    \begin{align*}
        \\
        \\
        f(x) &= 0 \\
        \frac{(x+1)^2}{2x^2} &= 0 \Leftrightarrow x = -1 \\\\
        \text{Senk. Asymp.}&\text{ bei }  x = 0\\
        \text{Wag. Asymp.}&\text{ bei }  y = \frac{1}{2}
    \end{align*}
    \vspace{3cm}
\end{boxx}
\newpage
\subsection{Funktionenscharen -- Ortskurven}
Eine Funktion mit Parameter nennt man \textbf{Funktionenschar}.
Die Menge aller Punkte, die alle eine gemeinsame Eigenschaft teilen, z. B.
gemeinsame Hochpunkte, bilden eine \textbf{Ortskurve}.
Um die Gleichung der Ortskurve zu bestimmen geht man wie folgt vor:
\begin{enumerate}
    \item Die notwendige Bedingung aufstellen und nach $x$ auflösen.
    \item $x$ in $f_t(x)$ einsetzen $\rightarrow$ $y$-Koordinate
    \item Gleichung aus 1. nach $t$ auflösen und in die $y$-Koordinate aus 2. einsetzen.
\end{enumerate}
\begin{boxx}[DarkBlue]{Beispiel}
    $a)$\hspace{3mm} Bestimme die Ortskurve aller Tiefpunkte von $\displaystyle f_t(x) = (x-t)e^x + 1$
    \begin{align*}
        f'_t(x) = e^x + (x-t)e^x
    \end{align*}
    \begin{align*}
        f'_t(x) &= 0 \\
        e^x + (x-t)e^x &= 0 \\
        e^x\left(1 + (x-t)\right) &= 0 & &|\; e^x \not = 0 \;\;\forall x\\
        1 + (x-t) &= 0 \\
        x &= t-1
    \end{align*}
    \begin{align*}
        f(t-1) &= \left((t-1)-t\right)e^{t-1} + 1 \\
        &= -e^{t-1} + 1 & &|\; t = x+1 \\
        \Rightarrow y &= -e^x + 1
    \end{align*}

    $b)$\hspace{3mm} Untersuche $\displaystyle f_t(x) = 2(x+5)e^{tx}$ auf gemeinsame Punkte.
    \begin{align*}
        f_0(x) &= f_1(x) \\
        2(x+5) &= 2(x+5)e^{x} \\
        2(x+5) - 2(x+5)e^{x} &= 0 \\
        2(x+5)(1 - e^x) &= 0 \\
        \Rightarrow x_1 = -5;\; x_2 &= 0
    \end{align*}
    \begin{align*}
        f_t(-5) &= 2(-5+5)e^{-5t} = 0 \\
        f_t(0) &= 2(0+5)e^{0} = 10
    \end{align*}
    \begin{align*}
        \Rightarrow P_1(-5,0);\; P_2(0,10)
    \end{align*}
\end{boxx}
\newpage
\section{Lineare Gleichungssysteme}
\subsection{Der Gauß-Algorithmus}
Mehrere Gleichungen mit gemeinsamen (linearen) Variablen bilden ein
\textbf{lineares Gleichungssystem (LGS).}

Ein LGS mit 3 oder mehr Variablen löst man meistens mit dem 
\textbf{Gauß-Algorithmus} am sinnvollsten. Dabei werden durch
\begin{enumerate}
    \item vertauschen von Gleichungen,
    \item Multiplikation von einer oder mehrerer Gleichungen mit einer Zahl $\not= 0$,
    \item Addition mehrerer Gleichungen und
    \item Einsetzen einer Variablen in eine andere Gleichung
\end{enumerate}
so lange Variablen eliminiert und dadurch bestimmt bis eine sogenannte
\textbf{Stufenform} vorliegt.
\[\left( \begin{array}{ccc|c} x & x & x & x \\ \tikzmark{d1}0 & x & x  & x \\ \tikzmark{d2}0 & 0\tikzmark{d3} & x & x \end{array} \right) \begin{array}{cc}
    &\\
    \longleftarrow \text{Matrix-Schreibweise} &\\
    &
\end{array}\]
\begin{tikzpicture}[overlay, remember picture]
    \draw ([xshift=-1pt,yshift=11pt]{pic cs:d1}) -- ([xshift=-1pt,yshift=-1pt]{pic cs:d2});
    \draw ([xshift=-1pt,yshift=-1pt]{pic cs:d2}) -- ([xshift=6pt,yshift=-1pt]{pic cs:d3});
    \draw ([xshift=6pt,yshift=-1pt]{pic cs:d3}) -- ([xshift=-1pt,yshift=11pt]{pic cs:d1});
\end{tikzpicture}
\begin{boxx}[DarkBlue]{Beispiel}
    Löse das LGS.
    \begin{align*}
        &&&&2x_1 - 4x_2 + 5x_3 &= 3 &&&&|\; \cdot (-3) &&|\; \cdot (-2) &&&&\\
        &&&&3x_1 + 3x_2 + 7x_3 &= 13 &&&&|\; \cdot 2 &&&&\\
        &&&&4x_1 - 2x_2 - 3x_3 &= -1 &&&&&&|\; \cdot 1&&&&
    \end{align*}
    \begin{align*}
        &\sim \left( \begin{array}{ccc|c} 2 & -4 & 5 & 3 \\ 0 & 18 & -1  & 17 \\ 0 & 6 & -13 & -7 \end{array} \right) \begin{array}{ll}
             &  \\
             |\cdot 1& \\
             |\cdot (-3)&
        \end{array} \\
        &\sim \left( \begin{array}{ccc|c} 2 & -4 & 5 & 3 \\ 0 & 18 & -1  & 17 \\ 0 & 0 & 38 & 38 \end{array} \right) \begin{array}{ll}
             &  \\
             & \\
             \Rightarrow x_3 = 1&
        \end{array} 
    \end{align*}
    \begin{align*}
        18x_2 - 1 &= 17 \\
        x_2 &= 1
    \end{align*}
    \begin{align*}
        2x_1 - 4 + 5 &= 3 \\
        2x_1 &= 2 \\
        x_1 &= 1
    \end{align*}
    \[\Rightarrow \mathbb{L} = \lbrace(1;\;1;\;1)\rbrace\]
\end{boxx}
\newpage
\subsection{Lösungsmengen von LGS}
Man bringt ein LGS wie gewohnt un Stufenform und
erkennt dann schnell, dass ein LGS entweder
\begin{itemize}
    \item \textbf{keine}
    \item \textbf{eine} oder
    \item \textbf{unendlich} viele Lösungen haben kann.
\end{itemize}
Bei keiner Lösung erhält man eine Zeile der Form
$\displaystyle 0 \cdot x_3 = 1$, bei unendlich vielen
Lösungen erhält man bspw. $0 \cdot x_3 = 0$.
Dann wählt man für $x_3$ einen beliebigen Parameter und gibt die anderen Variablen in Abhängigkeit von diesem an.
\begin{boxx}[DarkBlue]{Beispiel}
    Bestimme die Lösungsmenge des LGS. \\
    $a)$\hspace{3mm}
    \begin{align*}
        &\left( \begin{array}{ccc|c} 1 & 2 & 2 & 3 \\ 2 & 4 & 4  & 6 \\ -1 & 1 & 5 & 9 \end{array} \right) \begin{array}{ll}
             |\cdot (-2) & |\cdot 1 \\
             |\cdot 1& \\
             &|\cdot 1
        \end{array} \\
        \sim&\left( \begin{array}{ccc|c} 1 & 2 & 2 & 3 \\ 0 & 0 & 0 & 0 \\ 0 & 3 & 7 & 12 \end{array} \right) \begin{array}{ll}
             & \\
             \Rightarrow x_3 = t& \\
             &
        \end{array} \\
    \end{align*}
    \begin{align*}
        3 x_2 + 7t &= 12 \\
        x_2 &= - \frac{7}{3}t + 4
    \end{align*}
    \begin{align*}
        x_1 + 2\left(-\frac{7}{3}t + 4\right) + 2t &= 3 \\
        x_1 - \frac{14}{3}t + 8 + 2t &= 3
        x_1 &= \frac{8}{3}t - 5
    \end{align*}
    \[\mathbb{L} = \biggl\{\left(\frac{8}{3}t - 5;\; -\frac{7}{3}t + 4;\; t\right)\biggl\}\]
    $b)$\hspace{3mm}
    \begin{align*}
        &\left( \begin{array}{ccc|c} -2 & 1 & 4 & 3 \\ -4 & 2 & 8  & 8 \\ -3 & 2 & 5 & 9 \end{array} \right) \begin{array}{ll}
             |\cdot (-2) & \\
             |\cdot 1& \\
             &
        \end{array} \\
        \sim&\left( \begin{array}{ccc|c} -2 & 1 & 4 & 3 \\ 0 & 0 & 0 & 2 \\ -3 & 2 & 5 & 9 \end{array} \right) \begin{array}{ll}
             & \\
             \Rightarrow \mathbb{L} = \{\}& \\
             &
        \end{array} \\
    \end{align*}
\end{boxx}
\newpage
\subsection{Bestimmen ganzrationaler Funktionen}
Zum Bestimmen eines Funktionsterms einer ganzrationalen
Funktion bietet sich folgende Strategie:
\begin{enumerate}
    \item Aufstellen eines allgemeinen Funktionsterms und ggf. dessen Ableitungen.
    \item Bedingungen für $f$, $f'$, $f''$, ... formulieren.
    \item LGS aufstellen und lösen.
    \item Überprüfung ob ein Hochpunkt wirklich ein Hockpunkt ist, etc.
\end{enumerate}
\begin{boxx}[DarkBlue]{Beispiel}
    Bestimme die ganzrationale Funktion vom Grad 4, 
    dessen Graph symmetrisch zur $y$-Achse ist,
    den Punkt $A(2, 0)$ enthält und den Tiefpunkt $T(1, 0)$ hat.
    \begin{enumerate}
        \item \begin{align*}
            f(x) &= ax^4+ bx^3 + cx^2 + dx + e \\
            f'(x) &= 4ax^3 + 3bx^2 +2cx + d \\
            f''(x) &= 12x^2 + 6bx + 2c
        \end{align*}
        \item \hspace*{1mm}
        \begin{figure}[H]
            \centering
            \begin{tabular}{rl}
                Symmetrie zur $y$-Achse: & $b=0;\;d=0$ \\
                $A(2, 0)$: & $f(0) = e = 2$ \\
                $T(1, 0)$: & $f'(1) = 4a + 2c = 0$ \\
                & $f(1) = a + c + 2 = 0$
            \end{tabular}
        \end{figure}
        \item 
        \begin{align*}
            &\left( \begin{array}{cc|c} 4 & 2 & 0 \\ 1 & 1 & -2 \end{array} \right) \begin{array}{ll}
                |\cdot 1 & \\
                |\cdot -(4)&
            \end{array} \\
            \sim&\left( \begin{array}{cc|c} 4 & 2 & 0 \\ 0 & -2 & 8 \end{array} \right) \begin{array}{ll}
                & \\
                \Rightarrow c = -4&
            \end{array}\\
            &\hspace*{4mm}4a -8 = 0 \hspace*{6mm}\Rightarrow a = 2
        \end{align*}
        \[\Rightarrow f(x) = 2x^4 - 4x^2 + 2\]
        \item \begin{align*}
            f''(1) &= 24-8 = 16 > 0 \hspace*{3mm}\Rightarrow T(1, 0)
        \end{align*}
    \end{enumerate}
\end{boxx}
\end{document}