\documentclass{article}
\usepackage[ngerman]{babel} % |
\usepackage[utf8]{inputenc} % | Language and special characters
\usepackage[T1]{fontenc} %    |
\usepackage{amsmath} % |
\usepackage{amssymb} % | Math symbols and environments
\usepackage{amsthm} %  |
\usepackage{siunitx} %  Typesetting (SI) units
\DeclareSIUnit\litre{l}
%\usepackage{physics} % Advanced Mathematics of Physics, conflicts with siunitx
\usepackage{multirow} % Tables with cells that span more than one row
\usepackage{fancyhdr} % Custom page layout
\usepackage{graphicx} % Insert images
\usepackage{gensymb} % Generic units of measurement in math and text typeface
\usepackage{xfrac} % Split-level fractions
\usepackage{xcolor} % Handling colors
\usepackage{float} % Improved floating objects such as figures and tables
\usepackage{tikz} % Graphics and figures
\usetikzlibrary{arrows} % -> customizable arrow tips
\usetikzlibrary{tikzmark}
\usepackage{pgfplots} % Plotting functions and data 
\usepgfplotslibrary{fillbetween} % -> filling areas in pgfplots
\pgfplotsset{compat=newest}
\usepackage{systeme} % Systems of equations
\usepackage{romanbar} % Roman numbers with bars
\usepackage{titlepic} % Putting picture on the title page
\usepackage{ragged2e} % Commands and environments for setting ragged text
\usepackage{multicol} % Multicolumn formating
\usepackage[a4paper, ignoreheadfoot, left=2.5cm, right=2.5cm, top=2cm, bottom=3.5cm, headsep=1cm]{geometry} % Margins etc.
\usepackage{cancel} % Cancel terms in equations
\usepackage[version=4]{mhchem} % Write chemical equations
\usepackage{wrapfig} % Figures with text wrapped around them
\usepackage{hyperref} % Extensive support for hypertext
\usepackage{sidecap} % Typeset captions sideways
\sidecaptionvpos{figure}{c} % -> Vertical alignment of caption
\usepackage{pdfpages} % Include PDF documents
\usepackage{setspace} % Set space between lines 
% (\singlespacing, \onehalfspacing, \doublespacing)
\usepackage{subcaption}
\usepackage[scale=0.96]{XCharter} % Use the XCharter text font
% \usepackage[xcharter]{newtxmath} % Set the math font

% Hyperref setup:
\hypersetup{
    colorlinks=true,
    linkcolor=black,
    filecolor=magenta,      
    urlcolor=blue,
    citecolor=black
}


\definecolor{page}{HTML}{FFFFFF} % white
\definecolor{text}{HTML}{000000} % black
\definecolor{primary}{HTML}{019875}
\definecolor{contrastColour}{HTML}{E8F1F2} % white
\definecolor{tertiary}{HTML}{C47238} % yellow
\definecolor{secondary}{HTML}{C6B53C} % orange
\definecolor{quaternary}{HTML}{BD3E4C} % red
\definecolor{alternativePrimary}{HTML}{13293D} % blue
\colorlet{infoBulleBackground}{page!98!text}
% Shades
\definecolor{LightGrey}{HTML}{a5b1c2}
\definecolor{Grey}{HTML}{778ca3}
\definecolor{DarkGrey}{HTML}{4b6584}
\definecolor{Black}{HTML}{231F20}
% Primary Colours
\definecolor{Red}{HTML}{eb3b47}
\definecolor{LightRed}{HTML}{fc5c65}
\definecolor{DarkRed}{HTML}{d91c38}

\definecolor{Yellow}{HTML}{f7b731}
\definecolor{LightYellow}{HTML}{fed330}
\definecolor{DarkYellow}{HTML}{f0a132}

\definecolor{Blue}{HTML}{3867d6}
\definecolor{LightBlue}{HTML}{4b7bec}
\definecolor{DarkBlue}{HTML}{2654bf}
% tertiary Colours
\definecolor{Green}{HTML}{20bf6b}
\definecolor{LightGreen}{HTML}{26de81}
\definecolor{DarkGreen}{HTML}{1ba155}

\definecolor{Orange}{HTML}{fa8231}
\definecolor{LightOrange}{HTML}{fd9644}
\definecolor{DarkOrange}{HTML}{f76b20}

\definecolor{Purple}{HTML}{8854d0}
\definecolor{LightPurple}{HTML}{a55eea}
\definecolor{DarkPurple}{HTML}{6e49b8}
% secondary+ Colours
% \definecolor{Brown}{HTML}{}
\definecolor{Cyan}{HTML}{0fb9b1}
\definecolor{LightCyan}{HTML}{2bcbba}
\definecolor{DarkCyan}{HTML}{00a8a8}


\usetikzlibrary{calc}

\usepackage[many]{tcolorbox}

\newtcolorbox{boxx}[2][]{
    enhanced,
    fonttitle=\fontsize{11}{13.2}\selectfont\bfseries,
    coltitle=text,
    colback=infoBulleBackground,
    colbacktitle=infoBulleBackground,
    colframe=infoBulleBackground,
    toprule=4pt,titlerule=0pt,bottomrule=0pt,rightrule=0pt,leftrule=0pt,
    segmentation hidden,
    sharpish corners,
    overlay={\draw[line width=2.5pt,#1] (frame.north west)--(frame.south west);},
    % margins & padding
    before skip=\baselineskip,
    after skip=\baselineskip,
    boxsep=2pt,
    top=6pt,
    bottom=4pt,
    left=8pt,
    right=6pt,
    breakable,
    title=#2,
    extras middle and last={%
        %top=20pt,
        overlay={%
            \draw[line width=2.5pt,#1] (frame.north west)--(frame.south west);
            % \draw[line width=0.5pt,dashed,#1] (frame.north west)--(frame.north east);
            % \draw[] {([yshift=-12pt,xshift=4.5pt]frame.north west)} node[anchor=west] {#2};
            % \draw[] {([yshift=-12pt,xshift=-4.5pt]frame.north east)} node[anchor=east] {eg};
        }
    },
}

\definecolor{codegreen}{HTML}{369432}
\definecolor{codegray}{HTML}{3a3d41}
\definecolor{codepurple}{HTML}{c586c0}
\definecolor{backcolour}{HTML}{1e1e1e}
\definecolor{standartcolor}{HTML}{cccccc}
\definecolor{codeorange}{HTML}{f48771}

\setlength{\parindent}{0pt}
\definecolor{bg}{rgb}{0.13, 0.13, 0.13}
%\pagecolor{bg}
%\color{white}
\begin{document}
%---COLORS---
\definecolor{pRed}{RGB}{255, 55, 55}
\definecolor{pOrange}{RGB}{219, 132, 36}

%\pagestyle{fancy}
%\fancyhf{}
%\lhead{Moathekuchenaufgabe}
%\rhead{Lösung}
%\cfoot{\thepage}

\thispagestyle{empty}

\section*{Partielle Integration}

\begin{boxx}[Red]{Definition}
    Um das Produkt zweier Funktionen $u$ und $v'$ zu integrieren,
    kann man dazu partielle Integration verwenden.

    \begin{align} \label{ibp_eq}
        \int uv' \, dx = uv - \int u'v\, dx
    \end{align}

    Dabei müssen $u$ und $v'$ so gewählt werden, dass das Integral von $u'v$
    einfacher zu lösen ist.
    \\\\
    Mit Integrationsgrenzen lautet die Formel:
    \begin{align*}
        \int_a^b uv' \, dx = \left[uv\right]_a^b - \int_a^b u'v\, dx
    \end{align*}
\end{boxx}

\begin{boxx}[DarkBlue]{Beispiel}
    $\displaystyle \int x \cos(x) \, dx$ \\

    Wählt man $u(x) = x$ und $v'(x) = \cos(x)$, 
    so erhält man für $u'$ und $v$:
    \begin{align*}
        u'(x) &= 1 & v(x) &= \sin(x)
    \end{align*}
    Mit der Formel \eqref{ibp_eq} folgt:
    \begin{flalign*}
        \int x \cos(x) \, dx &= x \sin(x) - \int \sin(x) \cdot 1 \, dx &\\
        &= x \sin(x) + \cos(x) + c&
    \end{flalign*}
\end{boxx}
\subsubsection*{Wiederholte partielle Integration}
Für Integrale wie $\displaystyle \int x^2 e^x \, dx$  
erhält man nach Anwendung der Formel $\displaystyle x^2 e^x - \int 2x e^x \, dx$.

Um das entstandene Integral $\displaystyle\int 2x e^x \, dx$ zu lösen, 
muss man partielle Integration erneut anwenden.

\subsubsection*{Phoenix-Integration}
Bei Integralen wie $\displaystyle \int \sin(x) \cos(x) \, dx$
erhält man nach Anwenden der Formel wieder dasselbe Integral.
Dieses kann dann mittels einer Äquivalenzumformung auf die andere Seite gebracht werden:
\begin{flalign*}
    \int \sin(x) \cos(x) \, dx &= \sin^2(x) - \int \sin(x) \cos(x) \, dx & &|\; +\int \sin(x) \cos(x) \, dx &\\
    2 \int \sin(x) \cos(x) \, dx &= \sin^2(x) & &|\; :2 &\\
    \int \sin(x) \cos(x) \, dx &= \frac{\sin^2(x)}{2} + c
\end{flalign*}
Es kann auch vorkommen, dass partielle Integration zweimal durchgeführt 
werden muss, bevor man zum ursprünglichen Integral zurückkommt.
Ebenso so ist zu beachten, dass ein konstanter Faktor vor das 
Integral gezogen werden kann: 
$\displaystyle \int c \cdot f(x) \, dx = c \int f(x) \, dx$

\newpage
\thispagestyle{empty}
\subsection*{DI-Methode}

\begin{wraptable}{r}{0.2\textwidth}
    \centering
    \begin{tabular}{ccc}
        ~ & $D$ & $I$ \\ 
        $+$ & $u$ & $v'$ \\ 
        $-$ & $u'$ & $v$ \\ 
        $+$ & $\vdots$ & $\vdots$ \\ 
    \end{tabular}
\end{wraptable}
Um partielle Integration etwas übersichtlicher und unkomplizierter 
zu gestalten, insbesondere bei wiederholter Anwendung, 
können wir die DI-Methode verwenden.
\\\\
Bei der DI-Methode erstellt man eine Tabelle mit zwei Spalten;
eine \textit{D}-Spalte, in der man die zu differenzierende Funktion wählt, 
und eine \textit{I}-Spalte, in der man die zu integrierende Funktion wählt.
Links daneben schreibt man abwechselnde $+$ und $-$ Zeichen.

Jetzt differenziert man die Funktion in der \textit{D}-Spalte und integriert Funktion in der \textit{I}-Spalte.
Man hört auf, wenn einer der folgenden 3 Haltepunkte aufkommt:
\\\\
\textbf{1. Eine 0 in der \textit{D}-Spalte}

Erreicht man eine 0 in der \textit{D}-Spalte, hört man auf.
Das Ergebnis erhält man nun durch das multiplizieren der Diagonalen
vom  Eintrag in der \textit{D}-Spalte (mit Vorzeichen) zum darunterliegenden Eintrag in der 
\textit{I}-Spalte für jede Zeile.
\begin{multicols}{2}

    \begin{flalign*}
        & \int x^2 \sin(3x) \,dx &\\
        & = \color{Purple} -\frac{1}{3} x^2 \cos(3x) \color{Orange}+ \frac{2}{9}x\sin(3x) \color{Green}+ \frac{2}{27}\cos(3x) \color{black} + c &
    \end{flalign*}
    \\\\
    \begin{table}[H]
        \centering
        \begin{tabular}{ccc}
            ~ & $D$ & $I$ \\ 
            \\[-.8em]
            $\color{Purple}+$ & $\color{Purple} x^2$\tikzmark{11a} & $\displaystyle\sin(3x)$ \\ 
            \\[-.8em]
            $\color{Orange}-$ & $\color{Orange}2x$\tikzmark{12a} & \tikzmark{11b}$\color{Purple}\displaystyle-\frac{1}{3}\cos(3x)$ \\ 
            \\[-.8em]
            $\color{Green}+$ & $\color{Green}2$\tikzmark{13a} & \tikzmark{12b}$\color{Orange}\displaystyle-\frac{1}{9}\sin(3x)$ \\ 
            \\[-.8em]
            $-$ & $0$ & \tikzmark{13b}$\color{Green}\displaystyle \frac{1}{27}\sin(3x)$ \\ 
        \end{tabular}
    \end{table}

\end{multicols}


\begin{tikzpicture}[overlay, remember picture, shorten >=7pt, shorten <=7pt, transform canvas={yshift=.25\baselineskip}]
    \color{Purple} \draw [->] ([xshift=-3pt]{pic cs:11a}) -- ([xshift=5pt]{pic cs:11b});
    \color{Orange} \draw [->] ([xshift=-2pt]{pic cs:12a}) -- ([xshift=5pt]{pic cs:12b});
    \color{Green}\draw [->] ([xshift=-1pt]{pic cs:13a}) -- ([xshift=5pt]{pic cs:13b});
\end{tikzpicture}


\textbf{2. Man kann das Produkt einer Zeile integrieren}

Kann man das Produkt einer Zeile integrieren, hört man auf.
Für das Ergebnis multipliziert man die Diagonalen wie beim ersten Haltepunkte
und addiert bzw. subtrahiert dann (je nach Vorzeichen) das Integral des Produkts der letzten Zeile.

\begin{multicols}{2}
    \begin{flalign*}
        & \int x^4 \ln(x) \,dx &\\
        & = \frac{1}{5}x^5 \ln(x) \color{Yellow} - \int \frac{1}{x} \cdot \frac{1}{5}x^5 \, dx & \\
        & = \frac{1}{5}x^5 \ln(x) - \int \frac{1}{5} x^4 \, dx \\
        & = \frac{1}{5}x^5 \ln(x) - \frac{1}{25} x^5 + c
    \end{flalign*}
    \\
    
    \begin{table}[H]
        \centering
        \begin{tabular}{ccc}
            ~ & $D$ & $I$ \\
            \\[-.8em]
            $+$ & $\displaystyle\ln(x)$\tikzmark{21a} & $\displaystyle x^4$ \\ 
            \\[-.8em]
            $\color{Yellow}-$ & $\color{Yellow}\displaystyle\frac{1}{x}$ & \tikzmark{21b}$\color{Yellow}\displaystyle \frac{1}{5}x^5$ \\ 
            \\[-.8em]
        \end{tabular}
    \end{table}
\end{multicols}

\begin{tikzpicture}[overlay, remember picture, shorten >=7pt, shorten <=7pt, transform canvas={yshift=.25\baselineskip}]
    \draw [->] ([xshift=-3pt]{pic cs:21a}) -- ([xshift=1pt]{pic cs:21b});
\end{tikzpicture}


\textbf{3. Eine Zeile wiederholt sich}

Wiederholt sich eine Zeile (abgesehen von Konstanten), hört man auf. 
Man verfährt nun so wie beim 2. Haltepunkt.
Dieser Fall entspricht dann der Phoenix-Integration und ist dementsprechend zu lösen.

\begin{multicols}{2}

    \begin{flalign*}
        \int e^x \sin(x) \,dx &= - e^x \cos(x) + e^x \sin(x) \color{Yellow} - \int e^x \sin(x)\,dx &\\
        2\int e^x \sin(x) \,dx &= - e^x \cos(x) + e^x \sin(x) &\\
        \int e^x \sin(x) \,dx &= - \frac{1}{2}e^x \cos(x) + \frac{1}{2}e^x \sin(x) + c&\\
    \end{flalign*}
    \\
    \begin{table}[H]
        \centering
        \begin{tabular}{ccc}
            ~ & $D$ & $I$ \\ 
            \\[-.8em]
            $+$ & $\displaystyle e^x$\tikzmark{31a} & $\displaystyle \sin(x)$ \\ 
            \\[-.8em]
            $-$ & $\displaystyle e^x$\tikzmark{32a} & \tikzmark{31b}$\displaystyle -\cos(x)$ \\ 
            \\[-.8em]
            $\color{Yellow}+$ & $\color{Yellow}\displaystyle e^x$ & \tikzmark{32b}$\color{Yellow}\displaystyle -\sin(x)$ \\ 
        \end{tabular}
        \begin{tikzpicture}[overlay, remember picture, shorten >=7pt, shorten <=7pt, transform canvas={yshift=.25\baselineskip}]
            \draw [->] ([xshift=-3pt]{pic cs:31a}) -- ([xshift=4pt]{pic cs:31b});
            \draw [->] ([xshift=-3pt]{pic cs:32a}) -- ([xshift=4pt]{pic cs:32b});
        \end{tikzpicture}
    \end{table}
\end{multicols}

\end{document}